% Options for packages loaded elsewhere
\PassOptionsToPackage{unicode}{hyperref}
\PassOptionsToPackage{hyphens}{url}
\PassOptionsToPackage{dvipsnames,svgnames,x11names}{xcolor}
%
\documentclass[
  letterpaper,
]{scrreprt}

\usepackage{amsmath,amssymb}
\usepackage{iftex}
\ifPDFTeX
  \usepackage[T1]{fontenc}
  \usepackage[utf8]{inputenc}
  \usepackage{textcomp} % provide euro and other symbols
\else % if luatex or xetex
  \usepackage{unicode-math}
  \defaultfontfeatures{Scale=MatchLowercase}
  \defaultfontfeatures[\rmfamily]{Ligatures=TeX,Scale=1}
\fi
\usepackage{lmodern}
\ifPDFTeX\else  
    % xetex/luatex font selection
\fi
% Use upquote if available, for straight quotes in verbatim environments
\IfFileExists{upquote.sty}{\usepackage{upquote}}{}
\IfFileExists{microtype.sty}{% use microtype if available
  \usepackage[]{microtype}
  \UseMicrotypeSet[protrusion]{basicmath} % disable protrusion for tt fonts
}{}
\makeatletter
\@ifundefined{KOMAClassName}{% if non-KOMA class
  \IfFileExists{parskip.sty}{%
    \usepackage{parskip}
  }{% else
    \setlength{\parindent}{0pt}
    \setlength{\parskip}{6pt plus 2pt minus 1pt}}
}{% if KOMA class
  \KOMAoptions{parskip=half}}
\makeatother
\usepackage{xcolor}
\usepackage[lmargin=2cm,rmargin=2cm,tmargin=3cm,bmargin=3cm]{geometry}
\setlength{\emergencystretch}{3em} % prevent overfull lines
\setcounter{secnumdepth}{5}
% Make \paragraph and \subparagraph free-standing
\makeatletter
\ifx\paragraph\undefined\else
  \let\oldparagraph\paragraph
  \renewcommand{\paragraph}{
    \@ifstar
      \xxxParagraphStar
      \xxxParagraphNoStar
  }
  \newcommand{\xxxParagraphStar}[1]{\oldparagraph*{#1}\mbox{}}
  \newcommand{\xxxParagraphNoStar}[1]{\oldparagraph{#1}\mbox{}}
\fi
\ifx\subparagraph\undefined\else
  \let\oldsubparagraph\subparagraph
  \renewcommand{\subparagraph}{
    \@ifstar
      \xxxSubParagraphStar
      \xxxSubParagraphNoStar
  }
  \newcommand{\xxxSubParagraphStar}[1]{\oldsubparagraph*{#1}\mbox{}}
  \newcommand{\xxxSubParagraphNoStar}[1]{\oldsubparagraph{#1}\mbox{}}
\fi
\makeatother


\providecommand{\tightlist}{%
  \setlength{\itemsep}{0pt}\setlength{\parskip}{0pt}}\usepackage{longtable,booktabs,array}
\usepackage{calc} % for calculating minipage widths
% Correct order of tables after \paragraph or \subparagraph
\usepackage{etoolbox}
\makeatletter
\patchcmd\longtable{\par}{\if@noskipsec\mbox{}\fi\par}{}{}
\makeatother
% Allow footnotes in longtable head/foot
\IfFileExists{footnotehyper.sty}{\usepackage{footnotehyper}}{\usepackage{footnote}}
\makesavenoteenv{longtable}
\usepackage{graphicx}
\makeatletter
\def\maxwidth{\ifdim\Gin@nat@width>\linewidth\linewidth\else\Gin@nat@width\fi}
\def\maxheight{\ifdim\Gin@nat@height>\textheight\textheight\else\Gin@nat@height\fi}
\makeatother
% Scale images if necessary, so that they will not overflow the page
% margins by default, and it is still possible to overwrite the defaults
% using explicit options in \includegraphics[width, height, ...]{}
\setkeys{Gin}{width=\maxwidth,height=\maxheight,keepaspectratio}
% Set default figure placement to htbp
\makeatletter
\def\fps@figure{htbp}
\makeatother
% definitions for citeproc citations
\NewDocumentCommand\citeproctext{}{}
\NewDocumentCommand\citeproc{mm}{%
  \begingroup\def\citeproctext{#2}\cite{#1}\endgroup}
\makeatletter
 % allow citations to break across lines
 \let\@cite@ofmt\@firstofone
 % avoid brackets around text for \cite:
 \def\@biblabel#1{}
 \def\@cite#1#2{{#1\if@tempswa , #2\fi}}
\makeatother
\newlength{\cslhangindent}
\setlength{\cslhangindent}{1.5em}
\newlength{\csllabelwidth}
\setlength{\csllabelwidth}{3em}
\newenvironment{CSLReferences}[2] % #1 hanging-indent, #2 entry-spacing
 {\begin{list}{}{%
  \setlength{\itemindent}{0pt}
  \setlength{\leftmargin}{0pt}
  \setlength{\parsep}{0pt}
  % turn on hanging indent if param 1 is 1
  \ifodd #1
   \setlength{\leftmargin}{\cslhangindent}
   \setlength{\itemindent}{-1\cslhangindent}
  \fi
  % set entry spacing
  \setlength{\itemsep}{#2\baselineskip}}}
 {\end{list}}
\usepackage{calc}
\newcommand{\CSLBlock}[1]{\hfill\break\parbox[t]{\linewidth}{\strut\ignorespaces#1\strut}}
\newcommand{\CSLLeftMargin}[1]{\parbox[t]{\csllabelwidth}{\strut#1\strut}}
\newcommand{\CSLRightInline}[1]{\parbox[t]{\linewidth - \csllabelwidth}{\strut#1\strut}}
\newcommand{\CSLIndent}[1]{\hspace{\cslhangindent}#1}

\makeatletter
\@ifpackageloaded{bookmark}{}{\usepackage{bookmark}}
\makeatother
\makeatletter
\@ifpackageloaded{caption}{}{\usepackage{caption}}
\AtBeginDocument{%
\ifdefined\contentsname
  \renewcommand*\contentsname{Table of contents}
\else
  \newcommand\contentsname{Table of contents}
\fi
\ifdefined\listfigurename
  \renewcommand*\listfigurename{List of Figures}
\else
  \newcommand\listfigurename{List of Figures}
\fi
\ifdefined\listtablename
  \renewcommand*\listtablename{List of Tables}
\else
  \newcommand\listtablename{List of Tables}
\fi
\ifdefined\figurename
  \renewcommand*\figurename{Figure}
\else
  \newcommand\figurename{Figure}
\fi
\ifdefined\tablename
  \renewcommand*\tablename{Table}
\else
  \newcommand\tablename{Table}
\fi
}
\@ifpackageloaded{float}{}{\usepackage{float}}
\floatstyle{ruled}
\@ifundefined{c@chapter}{\newfloat{codelisting}{h}{lop}}{\newfloat{codelisting}{h}{lop}[chapter]}
\floatname{codelisting}{Listing}
\newcommand*\listoflistings{\listof{codelisting}{List of Listings}}
\usepackage{amsthm}
\theoremstyle{definition}
\newtheorem{definition}{Definition}[chapter]
\theoremstyle{remark}
\AtBeginDocument{\renewcommand*{\proofname}{Proof}}
\newtheorem*{remark}{Remark}
\newtheorem*{solution}{Solution}
\newtheorem{refremark}{Remark}[chapter]
\newtheorem{refsolution}{Solution}[chapter]
\makeatother
\makeatletter
\makeatother
\makeatletter
\@ifpackageloaded{caption}{}{\usepackage{caption}}
\@ifpackageloaded{subcaption}{}{\usepackage{subcaption}}
\makeatother

\ifLuaTeX
  \usepackage{selnolig}  % disable illegal ligatures
\fi
\usepackage{bookmark}

\IfFileExists{xurl.sty}{\usepackage{xurl}}{} % add URL line breaks if available
\urlstyle{same} % disable monospaced font for URLs
\hypersetup{
  pdftitle={Second year progress review},
  pdfauthor={Nicholas Harbour},
  colorlinks=true,
  linkcolor={blue},
  filecolor={Maroon},
  citecolor={Blue},
  urlcolor={Blue},
  pdfcreator={LaTeX via pandoc}}


\title{Second year progress review}
\author{Nicholas Harbour}
\date{September 5, 2024}

\begin{document}
\cleardoublepage
\thispagestyle{empty}
{\centering
{\Huge\bfseries Second year progress review \par}
\vspace{12ex}
{\Large\bfseries Nicholas Harbour \par}
{\Large Student ID: 14330088 \par}
{\Large ORCID: 0009-0008-2424-4516 \par}
{\Large Supervised by: Markus Owen and Matthew Hubbard\par}
{\bfseries\large September 5, 2024 \par}
\vspace{12ex}
%
%
{\bfseries\large Centre for Mathematical Medicine and Biology, School of
Mathematical Sciences, University of Nottingham, Nottingham, NG7 2RD,
UK \par}
%
}

\renewcommand*\contentsname{Table of contents}
{
\hypersetup{linkcolor=}
\setcounter{tocdepth}{2}
\tableofcontents
}

\bookmarksetup{startatroot}

\chapter*{Preface}\label{preface}
\addcontentsline{toc}{chapter}{Preface}

\markboth{Preface}{Preface}

This is a Quarto book, to view it as HTML you can go here:
\url{https://harbour-n.github.io/Second-year-report/}.

The report is structured as follows. In the first chapter I give a
summary of the main results in my PhD over the first two years I then
outline a brief plan for the remaining time of my PhD. In the second
chapter I present a literature review specifically focusing on cancer
stem cell modelling, the primary focus of my second year. In the third
chapter I present our preprint
\href{https://www.biorxiv.org/content/10.1101/2024.08.22.609156v1.abstract}{Virtual
Clinical Trials of BMP4 Differentiation Therapy: Digital Twins to Aid
Glioblastoma Trial Design}, which will form a thesis chapter.

\bookmarksetup{startatroot}

\chapter{Summary}\label{summary}

This chapter provides an overview of the progress in my PhD research and
outlines the future direction of my thesis.

My PhD research centres on the mathematical modelling of cellular
subpopulations in glioblastoma (GBM), the most common and aggressive
primary malignant brain caner in adults. During the first year, I
focused on analysing single-cell and bulk RNA-seq data, aiming to
extract dynamic information about tumour progression from these static
snapshots, to motivate mathematical modelling. This lead to many
conference talks/posters and a published abstract
\href{https://doi.org/10.1093/neuonc/noad179.0150}{Inference of cell
cycle regulation between glioblastoma subpopulations in vivo to drive
computational and mathematical models of the cancer complex system} from
the Society of Neuro-Oncology annual meeting, although it may not form a
significant part of my thesis.

Following my initial visit to the Mathematical Neuro-Oncology Lab, run
by Kristin Swanson at the Mayo Clinic in May 2023, I became involved in
a project investigating differentiation therapy as a novel treatment for
GBM, this is a joint project with Dr Quinones lab in the department of
Neurosurgery (Mayo Clinic Florida). This has become the primary focus of
my second year. GBM remains almost universally fatal, partly due to its
resistance to radiation therapy. Research has identified a subpopulation
of cancer cells in GBM (and other cancers) with stem cell-like
properties that make them highly resistant to conventional cytotoxic
treatments, such as radiotherapy and chemotherapy. These cancer stem
cells (CSCs), or glioma stem cells (GSC) in the case of glioma, are
believed to drive tumour initiation, growth, and recurrence, making
their eradication crucial for effective treatment. Dr Quinones' lab is
developing new therapies for GBM using novel delivery mechanisms that
aim to create a less favourable microenvironment for GBM to survive. One
therapy they are particularly interested in is BMP4; this is a protein
that has been shown to drive differentiation of GSCs towards a
predominantly glial (astrocytic) fate, to reduce GBM tumor burden
\emph{in vivo} and to improve survival in a mouse model of GBM.
Throughout my second year, we have been developing a mathematical model
that simulates GSC-driven tumor growth in GBM, its response to BMP4
therapy and standard radiotherapy. Using data from five glioma stem cell
lines provided by Dr.~Quinones' lab, we have parameterized this model,
enabling us to estimate the sensitivity of these cell lines to BMP4. We
explore the model for a range of different parameters developing a
virtual clinical trial approach to see how BMP4 impacts simulated GBM
growth across a range of virtual cohorts. Our findings suggest that
tumour proliferation rate is also a critical factor that must be
accounted for when assessing BMP4 efficacy. Additionally, we have used
this model to explore various BMP4 treatment schedules, with the
ultimate goal of informing future clinical trial design.

In addition to this primary project, I have been involved in several
other initiatives during my second year, which are not detailed anywhere
else in this report. In November 2023 I took part in the Integrated
Mathematical Oncology (IMO) workshop at the Moffit Cancer Center,
Florida. Our team developed a mathematical model for evolutionary
steering in breast cancer, came second place and won a pilot fund of
\$50,000 to further this research project
(\url{https://imoworkshop.org/IMO11/index.html}). In March 2024, I
undertook a second visit to the Mathematical Neuro-Oncology Lab at Mayo
Clinic Arizona. During this visit I took part in an academic retreat
focusing on sex differences and the the immune landscape in GBM. This
visit strengthened my collaboration with the group, resulting in an
official research internship position. From May to June I led a team of
PhD students and research staff in a
\href{(\%3Chttps://wolfbyttner.github.io/exeter-quantitative-modelling/\%3E)}{datathon}
run by the University of Exeter and EPSRC Hub for Quantitative Modelling
in Healthcare. The goal was to quantify heterogeneity in human daily
rhythms using time series data of hormone concentrations in healthy
individuals. Our team won the prize for best negative result, using
functional data analysis and machine learning techniques to demonstrate
that circadian rhythm signals from sleep-wake cycles dominate over other
signals that could differentiate patient metadata such as sex, age or
weight.

The plan for my thesis includes several chapters based on a series of
papers, a draft of the anticipated structure is as follows:

\begin{itemize}
\item
  \textbf{Introduction}: A short text explaining the context and
  structure of the thesis.
\item
  \textbf{Outreach paper}: We aim to publish a paper in Frontiers for
  Young Minds, a journal targeting a young audience (ages 9-15). This
  could serve as an engaging and non-traditional introductory chapter to
  my thesis, explaining the basics of modelling in GBM.
\item
  \textbf{Literature review}: An overall literature review, encompassing
  all of my work. This will be slightly broader than the literature
  review presented in this report (Chapter~\ref{sec-lit-review}) that
  focuses specifically on the cancer stem cell literature that is
  particularly useful for understanding the context for my preprint.
\item
  \textbf{BMP4 Virtual Clinical Trials}: This will be based on the
  preprint
  \href{https://www.biorxiv.org/content/10.1101/2024.08.22.609156v1.abstract}{Virtual
  Clinical Trials of BMP4 Differentiation Therapy: Digital Twins to Aid
  Glioblastoma Trial Design}. We plan to submit this to a broad
  interdisciplinary science journal, focusing on the integration of data
  and modelling, as well as model-informed clinical trial design.
\item
  \textbf{Mathematical Analysis of the model}: Following on from this we
  are planing on writing a paper that will consider the model from a
  more mathematical perspective. We have already begun some of this
  analysis such as steady states of the model, nullclines, model
  reductions, although this analysis is not included in this report.
\item
  \textbf{Digital Twins and Alan Turing Institute}: Beginning in January
  2025, I will participate in the Alan Turing Institute's PhD enrichment
  placement, which is designed to help PhD students deepen their
  research in machine learning and data science. The Institute hosts a
  center dedicated to digital twin research, where I plan to tackle some
  of the fundamental challenges in developing digital twins for
  healthcare, such as creating robust, uncertainty-aware models that can
  be effectively used in clinical settings. In collaboration with the
  Mathematical Neuro-Oncology Lab, we will apply these ideas to digital
  twin models for GBM.
\item
  \textbf{Conclusion}: Drawing together the strands of the thesis.
\end{itemize}

\bookmarksetup{startatroot}

\chapter{Literature review}\label{sec-lit-review}

In this chapter we present a brief literature review particularly
focused of stem cells in cancer and the different modelling approaches
that have been used.

\section{Introduction}\label{sec-introduction-lit}

Stem cells are defined as cells that have the ability to perpetuate
themselves through self-renewal and to generate mature cells of a
particular tissue through differentiation (\citeproc{ref-reya2001}{Reya
et al. 2001}). Stem cells are fundamental to tissue maintenance and
repair; they also play a critical role in cancer development and in
determining the outcomes of cancer treatment
(\citeproc{ref-weiss2017a}{Weiss, Komarova, and Rodriguez-Brenes 2017}).

\section{Stem cells in cancer}\label{sec-stem-cells-in-cancer}

Perhaps the most important and useful property of stem cells is that of
self-renewal. Self-renewal is crucial to stem cell function, because it
is required by the majority of stem cells to persist for the lifetime of
the animal. Moreover, whereas stem cells from different organs may vary
in their developmental potential, all stem cells must self-renew and
regulate the relative balance between self-renewal and differentiation.
Understanding the regulation of normal stem cell self-renewal is also
fundamental to understanding the regulation of cancer cell
proliferation, because cancer can be considered to be a disease of
unregulated self-renewal (\citeproc{ref-reya2001}{Reya et al. 2001}).
Another distinguishing hallmark of stem cells is the ability to undergo
asymmetric division, during which stem cells give rise to daughter cells
of different fates, proliferative potential, size or other
characteristics (\citeproc{ref-majumdar2020}{Majumdar and Liu 2020};
\citeproc{ref-hitomi2021}{Hitomi et al. 2021a}). Cancer stem cells
(CSCs) generate such diverse progeny by executing multiple modes of cell
division. Lineage-tracing experiments in glioma stem cells (GSCs)
revealed that CSC undergo three main types of cell division: 1)
Symmetric CSC self-renewing division, where a CSC produces two daughter
CSCs; 2) symmetric differentiating division, where a CSC gives rise to
two non-CSC daughter cells; 3) asymmetric division, where a CSC produces
one CSC and one non-CSC. Additionally less than 1\% of cell divisions
resulted in cell death (\citeproc{ref-lathia2011}{Lathia et al. 2011}).
The types of CSC cell division are summarized in
Figure~\ref{fig-types-csc-div}.

\begin{figure}

\centering{

\includegraphics{images/lit_review/types_csc_div.png}

}

\caption{\label{fig-types-csc-div}The three main types of CSC cell
division. Symmetric self-renewal results in two daughter CSCs. Symmetric
differentiation results in two non-CSC daughter cells. Asymmetric
differentiation results in one CSC and one non-CSC. Non-CSCs are denoted
CC for cancer cells (created with BioRender.com).}

\end{figure}%

Numerous arguments suggest a stem-cell origin for human cancers. First,
it is worth noting that stem cells possess many of the features that
constitute the tumour phenotype, including self-renewal and essentially
unlimited replicative potential (\citeproc{ref-hanahan2000}{Hanahan and
Weinberg 2000}). Second, the mutations that initiate tumour formation
seem to accumulate in cells that persist throughout a person's life, as
suggested by the exponential increase of cancer incidence with age
(\citeproc{ref-meza2008}{Meza et al. 2008}). This is thought to reflect
a requirement for between four and seven mutations in a single cell (and
its progeny) to effect malignant transformation
(\citeproc{ref-hanahan2000}{Hanahan and Weinberg 2000}). Similarly,
cancer formation from cells that persist throughout life is suggested by
an increased incidence in adults of skin tumours such as melanoma after
higher childhood exposure to a mutagenic agent such as ultraviolet solar
radiation (\citeproc{ref-balk2011}{Balk 2011}). Normal somatic stem
cells are strong candidates for such persistent cells, an alternative
explanation would be that a more mature cell undergoes a
dedifferentiation event, reverting to a more primitive stem cell
phenotype (\citeproc{ref-sell1993}{Sell 1993};
\citeproc{ref-reya2001}{Reya et al. 2001}).

A stem cell origin for human cancers was first identified in leukaemia,
perhaps due to the high fraction of stem cells in the haematopoietic
system, when it was discovered that some, but not all, cancer cells were
able to initiate tumours of the blood
(\citeproc{ref-taipale2001}{Taipale and Beachy 2001};
\citeproc{ref-lapidot1994}{Lapidot et al. 1994};
\citeproc{ref-bonnet1997}{Bonnet and Dick 1997}). More recently CSCs
have been identified in many solid tumours including breast, colon and
brain (\citeproc{ref-al-hajj2003}{Al-Hajj et al. 2003};
\citeproc{ref-ricci-vitiani2007}{Ricci-Vitiani et al. 2007};
\citeproc{ref-ignatova2002}{Ignatova et al. 2002};
\citeproc{ref-hemmati2003}{Hemmati et al. 2003};
\citeproc{ref-singh2004}{Singh et al. 2004};
\citeproc{ref-galli2004}{Galli et al. 2004}). In GBM, cells expressing
the CD133 cell surface protein marker (also found on neural stem cells)
have been identified as having stem cell properties \emph{in vitro}
(\citeproc{ref-singh2003}{Singh et al. 2003}). Furthermore, when tested
using a xenograft assay, it was found that injection of as few as 100
CD133+ cells produced a tumour that could be serially transplanted and
was phenotypically similar to the patient's original tumour, while
injection of \(10^5\) CD133- cells survived in the host but did not
cause a tumour (\citeproc{ref-singh2004}{Singh et al. 2004}). This
provides strong evidence that there is a small subpopulation of glioma
stem cells that have the unique ability to initiate tumours, while the
majority of cells cannot.

\section{Cancer stem cells and treatment
resistance}\label{sec-cancer-stem-cells-and-treatment-resistance}

Radiation therapy is the most common form of treatment across all
cancers, with around 50\% of all cancer patients receiving radiotherapy
at some point in their treatment (\citeproc{ref-baskar2012}{Baskar et
al. 2012}). However, in addition to being tumour-initiating, CSCs are
highly resistant to both radio- and chemo-therapy through preferential
activation of the DNA damage checkpoint response and an increase in DNA
repair capacity (\citeproc{ref-bao2006}{Bao et al. 2006};
\citeproc{ref-tang2021}{Tang et al. 2021a}; \citeproc{ref-rich2007}{Rich
2007}; \citeproc{ref-schonberg2014}{Schonberg et al. 2014}). In glioma,
experimental results have shown that both in culture and mouse models
CD133-expressing stem cells survive radiation in larger proportions than
the majority of tumour cells which lack CD133 expression
(\citeproc{ref-bao2006}{Bao et al. 2006}; \citeproc{ref-gao2013}{Gao et
al. 2013}); these results suggest that CSC confer radio-resistance in
GBM and ultimately are the source of tumour recurrence after radiation.

In addition to being resistant to treatment CSCs also engage in a
synergistic relationship with the surrounding tumour microenvironment to
promote angiogenesis, proliferation, migration, tumour survival, and
immune evasion (\citeproc{ref-ma2018}{Ma et al. 2018};
\citeproc{ref-rich2007}{Rich 2007}). Taken together this highlights the
important role CSCs play in determining tumour response to therapy.
There is a desperate need for targeted therapies that either directly
kill CSCs or sensitize CSCs to standard cytotoxic therapies in order to
improve treatment outcomes.

\section{Mathematical models of cancer stem cell
dynamics}\label{sec-mathematical-models-of-cancer-stem-cell-dynamics}

Many different mathematical models have been developed to model stem
cell dynamics. Understanding CSC kinetics and interaction with their
non-stem counterparts is still limited; theoretical and mathematical
models may help elucidate their role in cancer progression and treatment
response. Here we focus on a subset of models used in the literature
that cover a wide range of modelling techniques and have particularly
inspired our modelling used in the preprint presented at the end of this
report.

Many of the following models use slightly different terminology to refer
to the non-stem cell population such as cancer cell, progenitor cells or
tumour cells, for clarity we will refer to non-stem cells always as
cancer cells (CCs) throughout this review.

\subsection{Agent-based models}\label{sec-agent-based-model}

In (\citeproc{ref-enderling2009}{Enderling et al. 2009}) and
(\citeproc{ref-gao2013}{Gao et al. 2013}) the authors develop an
agent-based model (ABM) to study the dynamics of CSCs and CCs in a
tumour. It is assumed that tumours are a heterogenous mix of CSCs and
CCs. Cells are considered as individual entities with a cell cycle and
limited proliferation capacity \(\rho \in [0,\rho_\text{max}]\). CSCs
have unlimited self-renewal, hence \(\rho_\text{max} = \infty\). At each
cell division CSCs can undergo symmetric self-renewing division with
probability \(\delta\) or asymmetric division with probability
\(1-\delta\). The proliferation capacity \(\rho\) is decremented at each
CC division and inherited by both daughter cells.

Simulations of the ABM model revealed the following key results:

\begin{itemize}
\tightlist
\item
  Tumours developing solely from CCs will inevitably die out, due to
  their limited proliferation capacity. Hence, CSCs are necessary for
  malignant tumour growth. This is consistent with experimental results
  showing only CSCs can initiate tumours
  (\citeproc{ref-lapidot1994}{Lapidot et al. 1994};
  \citeproc{ref-singh2004}{Singh et al. 2004}).
\item
  Tumours started from a single CC could still persist for a long time
  as long-term dormant lesions, but due to space-limited growth remain
  small -- well below the potential maximum size of
  \(2^{\rho_\text{max}}\). Significant growth only occurs once a CSC is
  initiated. In fact, even with a single CSC, the tumour may remain
  small for an extended period due to space-limited growth. This is
  consistent with the observation that many tumours remain dormant for
  many years before they start to grow
  (\citeproc{ref-sweeney1995}{Sweeney 1995};
  \citeproc{ref-neves-e-castro2006}{Neves-E-Castro 2006};
  \citeproc{ref-folkman2004}{Folkman and Kalluri 2004}).
\item
  A high rate of spontaneous death of CCs actually enables room for
  sufficient stem cell divisions to enrich the stem cell pool and drive
  tumour growth. This leads to what they call the ``tumour growth
  paradox'', where counterintuitively an increase in the death rate of
  CCs decreases the total tumour size in the short term, in the long run
  it leads to an increase in the total tumour size as the tumour
  contains more CSCs.
\end{itemize}

Mathematically the tumour growth paradox is defined as follows.

\begin{definition}[Tumour growth
paradox]\protect\hypertarget{def-tumor-growth-paradox}{}\label{def-tumor-growth-paradox}

Let \(N_\alpha (t)\) denote a total tumour population with death rate
\(\alpha\) for CCs. The population exhibits a tumour growth paradox if
there exist death rates \(\alpha_1 < \alpha_2\) and times \(t_1,t_2, T\)
and \(T_0\) such that

\begin{equation}\phantomsection\label{eq-tumour-growth-paradox}{
\begin{aligned}
N_{\alpha_1}(t_1) = N_{\alpha_2}(t_2)& \quad \text{and} \quad N_{\alpha_1}(t_1 + T) < N_{\alpha_2}(t_2 + T) \\
&\text{for} \quad (0<T<T_0),
\end{aligned}
}\end{equation}

\end{definition}

\subsection{Integro-differential
model}\label{sec-integro-differential-model}

Following on from the ABM developed in
(\citeproc{ref-enderling2009}{Enderling et al. 2009};
\citeproc{ref-gao2013}{Gao et al. 2013}), in
(\citeproc{ref-hillen2013}{Hillen, Enderling, and Hahnfeldt 2013}) the
authors develop an integro-differential equation version of the model,
based on the same assumptions as in
(\citeproc{ref-enderling2009}{Enderling et al. 2009};
\citeproc{ref-gao2013}{Gao et al. 2013}) outlined in
Section~\ref{sec-agent-based-model}. Let \(u(x,t)\) denote the density,
in cells per unit space, and let \(v(x,t)\) denote the density of CCs.
Hence, the total tumour density is denoted \(N(x,t) = u(x,t) + v(x,t)\).
For this analysis cells are assumed to be very small compared to the
size of the tissue domain \(\Omega\) (which we take without loss of
generality to have unit volume). It is also assumed that cells cannot
pile on top of each other so there is a maximum density of one cell per
unit space, this implies \(N(x,t) \leq 1\). Cells can only proliferate
if there is space to place the daughter cells, otherwise reproduction is
inhibited (cellular quiescence). To model the spatial search for space,
they define a nonlinear integral term, consistent with the ABM
(\citeproc{ref-enderling2009}{Enderling et al. 2009};
\citeproc{ref-gao2013}{Gao et al. 2013}) they assume that all cells can
migrate randomly, which is modelled by simple diffusion. These
assumptions lead to the following system of equations to describe CSC
and CC dynamics:

\begin{equation}\phantomsection\label{eq-integro-differential-CSC-model}{
\begin{aligned}
    \underbrace{\frac{\partial u(x,t)}{\partial t}}_\text{Rate of change CSCs} &= \underbrace{D_u \nabla^2 u}_\text{Diffusion of CSCs} + \underbrace{\delta \gamma \int_{\Omega} k(x,y,N(x,t))u(y,t) dy}_\text{Self-renewal of CSCs} \\
    \underbrace{\frac{\partial v(x,t)}{\partial t}}_\text{Rate of change CCs} &= \underbrace{D_v \nabla^2 v}_\text{Diffusion of CCs} + \underbrace{(1-\delta) \gamma \int_{\Omega} k(x,y,N(x,t))u(y,t) dy}_\text{Differentiation of CSCs} + \\
    & \underbrace{\rho \int_{\Omega} k(x,y,N(x,t))v(y,t) dy}_\text{Proliferation of CCs} - \underbrace{\alpha v}_\text{Apoptosis of CCs}.
\end{aligned}
}\end{equation}

The spatial distribution kernel \(k(x,y,N)\) describes the rate of
progeny contribution to location x for a cell at location y per ``cell
cycle time'' i.e., the defined period between divisions of a freely
cycling cell. Since greater density at \(x\) would be expected to hinder
progeny occupation it is assumed that \(k\) is monotonically decreasing
in \(N\), with \(k(x,y,N(x,t))=0\) at \(N=1\). The number of cell cycle
times per unit time of CSCs and CCs are denoted by \(\gamma\) and
\(\rho\), respectively, and for simplicity it is assumed that
\(\gamma = \rho = 1\) throughout. The parameter \(\delta\) with
\(0 \leq \delta \leq 1\), as in the ABM
(\citeproc{ref-enderling2009}{Enderling et al. 2009};
\citeproc{ref-gao2013}{Gao et al. 2013}), denotes the fraction of CSC
divisions that are symmetric self-renewal, while \(1-\delta\) is the
fraction of asymmetric divisions. The parameter \(\alpha\) denotes the
spontaneous death rate of CCs. Background cell motility is modelled by
the diffusion coefficients \(D_u\) and \(D_v\) for CSCs and CCs,
respectively. The system is considered to hold in a smooth bounded
domain \(\Omega\), with homogeneous Neumann or Dirichlet boundary
conditions.

Homogeneous Neumann boundary conditions correspond to a boundary that is
impenetrable by cells, this could for example represent a tissue
surrounded by membranes, smooth muscle tissue, or bone, and are given by

\begin{equation}\phantomsection\label{eq-homogeneous-neumann-boundary-conditions}{
\frac{\partial u}{\partial n} = 0, \quad \frac{\partial v}{\partial n} = 0 \quad \text{on} \quad \partial \Omega,
}\end{equation}

where \(\partial / \partial n\) is the normal derivative at the
boundary. The redistribution kernel can only redistribute cells within
this domain \(\Omega\), hence we impose

\begin{equation}\phantomsection\label{eq-redistribution-kernel-neumann-BC}{
k(x,y,N) = 0 \quad \text{for} \quad x \notin \Omega. 
}\end{equation}

Homogeneous Dirichlet boundary conditions correspond to a boundary that
cells can freely leave but not re-enter again, for example this could
represent intravasation into adjacent blood vessels, and are given by

\begin{equation}\phantomsection\label{eq-homogeneous-dirichlet-boundary-conditions}{
u = 0, \quad v = 0 \quad \text{on} \quad \partial \Omega.
}\end{equation}

The redistribution kernel describes transport of cells out of the domain
but does not allow entering from the outside in, hence

\begin{equation}\phantomsection\label{eq-redistribution-kernel-dirichlet-BC}{
k(x,y,N) = 0 \quad \text{for} \quad y \notin \Omega.
}\end{equation}

Based on these two boundary conditions we can model any combination of
domains such as partially covered by membranes, partially permeable
membranes and adjacent blood vessels.

\subsection{ODE model reduction}\label{sec-ode-model-reduction}

In order to analyse this model analytically
(\citeproc{ref-hillen2013}{Hillen, Enderling, and Hahnfeldt 2013})
reduce the system of integro-differential equations
(Equation~\ref{eq-integro-differential-CSC-model}) to a system of
ordinary differential equations in the following way.

\textbf{Reduction 1: Progeny placement depends only on the density at
the destination}\\
In this case \(k(x,y,N(x,t)) = k(N(x,t))\). Introducing mean densities
which, given that the domain \(\Omega\) has unit volume, can be written
as

\begin{equation}\phantomsection\label{eq-density-mean}{
\bar{u}(t) = \int_{\Omega} u(y,t) dy, \quad \bar{v}(t) = \int_{\Omega} v(y,t) dy, \quad \bar{N}(t) = \bar{u}(t) + \bar{v}(t).
}\end{equation}

Then Equation~\ref{eq-integro-differential-CSC-model} becomes:

\begin{equation}\phantomsection\label{eq-first-reduction-ODE-CSC-model}{
\begin{aligned}
u_t(x,t) &= D_u \nabla^2 u(x,t) + \delta   k(N(x,t))\bar{u}(t), \\
v_t(x,t) &= D_v \nabla^2 v(x,t) + (1-\delta)   k(N(x,t))\bar{u}(t) + k(N(x,t))\bar{v}(t) - \alpha v(x,t).
\end{aligned}
}\end{equation}

\textbf{Reduction 2: Density is uniform across the domain}\\
If tumour growth is assumed uniform across the domain then,
\(k(N(x,t)) = k(\bar{N}(t))\) and \(u(x,t)\) and \(v(x,t)\) can be
replaced with their spatially averaged values (\(\bar{u}(t)\) and
\(\bar{v}(t)\)) and diffusion is zero everywhere. Therefore,
Equation~\ref{eq-first-reduction-ODE-CSC-model} becomes:

\begin{equation}\phantomsection\label{eq-ODE-CSC-model}{
\begin{aligned}
    \frac{d \bar{u}}{dt} &= \delta  k(\bar{N}(t)) \bar{u}, \\
    \frac{d \bar{v}}{dt} &= (1-\delta)  k(\bar{N}(t)) \bar{u} + k(\bar{N}(t)) \bar{v} - \alpha \bar{v}(t),
\end{aligned}
}\end{equation}

where the volume filling constraint \(k(\bar{N})\) is taken to be

\begin{equation}\phantomsection\label{eq-volume-filling-constraint}{
k(\bar{N}) = \text{max} \left\{0, 1 - \bar{N}^\sigma \right\}, \quad \text{for} \quad \sigma > 1.
}\end{equation}

An exponent of \(\sigma = 1\) corresponds to a linearly decreasing rate
of occupancy for newborn cells as the total density \(\bar{N}\)
increases. Since cells are nonrigid, deformable and able to squeeze into
available spaces (\citeproc{ref-hillen2013}{Hillen, Enderling, and
Hahnfeldt 2013}) argue that \(\sigma > 1\) is appropriate and take it to
be \(\sigma = 4\), in all their simulations.

Without a CSC population \(\bar{u}(t)\), the density of CCs
\(\bar{v}(t)\) satisfies the equation

\begin{equation}\phantomsection\label{eq-ODE-TC-only-model}{
\frac{d\bar{v}}{dt}  =K(\bar{v}(t))\bar{v}(t) - \alpha \bar{v}(t).
}\end{equation}

Since \(K(\bar{v}(t))\) is a decreasing function of \(\bar{v}(t)\) the
CC population will die out when \(\alpha > k(0)\). Note that this does
not specifically set a limited proliferation capacity for CCs, as was
the case in (\citeproc{ref-enderling2009}{Enderling et al. 2009}),
rather if \(\alpha > k(0)\) then \(\alpha > k(N)\) for all \(N\) hence
the CC population will never survive on its own.

This simpler ODE model allows for analysis of the steady states, from
which it can be shown that the pure stem cell steady state
\((u,v)= (1,0)\) is a global attractor. Therefore, this model predicts
that for long times the tumour will consist of only stem cells.
Intermediate tumour composition and the time at which the steady state
is achieved are dependent on cell death rate \(\alpha\). The convergence
to \((1,0)\) is somewhat surprising as typically the CSC compartment is
considered small, comprising only 1-3\% of the total tumour
(\citeproc{ref-bao2006}{Bao et al. 2006}). This suggests that the model
may miss key biological dynamics of the CSCs such as some stem cell
death and feedback regulation of symmetric / asymmetric division.
However, (\citeproc{ref-hillen2013}{Hillen, Enderling, and Hahnfeldt
2013}) argue that this does not interfere with their analysis, as they
are interested in the intermediate time dynamics of tumour initiation
and growth, rather than the long-term behaviour as
\(t \rightarrow \infty\).

\subsection{Stochastic model of CSC
dynamics}\label{sec-stochastic-model}

In (\citeproc{ref-turner2009}{Turner et al. 2009}), the authors develop
a stochastic model for the dynamics of CSCs and CCs, particularly for
the case of brain cancer. This stochastic model is particularly
appropriate for situations in which small numbers of cells are present
such as \emph{in vitro} or in the early stages of tumour formation. In
these cases stochastic fluctuations may have significant effects and
cannot be neglected. To study larger populations the authors then derive
a deterministic ODE model, based on the stochastic master equation, that
describes the average number of CSCs and CCs.

The model assumptions on CSCs and CCs are largely similar as those given
previously in Section~\ref{sec-agent-based-model}. However, one key
difference is that CSCs are assumed not to be immortal so have some
probability of death. Defining \(p(n_s, n_p,t)\) as the probability that
there are exactly \(n_s\) CSCs and \(n_p\) CCs at time \(t\), the
stochastic master equation is given by

\begin{equation}\phantomsection\label{eq-stochastic-master-equation}{
\begin{aligned}
\frac{dp(n_s,n_p,t)}{dt} = & \ \rho_s \left[ \underbrace{r_1 (n_s - 1) p(n_s - 1, n_p, t)}_{\text{symmetric self renewal of CSCs}} \right. \\
& + \underbrace{r_2 n_s p(n_s, n_p - 1, t)}_{\text{asymmetric division of CSCs}} \\
& + \underbrace{r_3 (n_s + 1) p(n_s + 1, n_p - 2, t)}_{\text{symmetric differentiation of CSCs}} \\
& \left. - \underbrace{n_s p(n_s, n_p, t)}_\text{Overall division} \right] \\
& + \underbrace{\Gamma_s \left[ (n_s + 1) p(n_s + 1, n_p, t) - n_s p(n_s, n_p, t) \right]}_\text{Apoptosis of CSCs} \\
& + \underbrace{\Gamma_p \left[ (n_p + 1) p(n_s, n_p + 1, t) - n_p p(n_s, n_p, t) \right]}_\text{Apoptosis of CCs}.
\end{aligned}
}\end{equation}

This model explicitly accounts for all three types of CSC division shown
in Figure~\ref{fig-types-csc-div} where \(r_1=\) symmetric self renewal
probability, \(r_2=\) asymmetric division probability, \(r_3=\)
symmetric differentiation probability, and \(\rho_s\) represents the
overall CSC division rate. While the model in
Section~\ref{sec-integro-differential-model} only accounts for
asymmetric division it can be shown that these two formulations are
equivalent for a large set of parameter values
(\citeproc{ref-hillen2013}{Hillen, Enderling, and Hahnfeldt 2013}). The
parameters \(\Gamma_s\) and \(\Gamma_p\) represent the rate of apoptosis
for CSCs and CCs respectively. Due to the model's stochastic nature, and
the inclusion of a death probability for CSCs, it can be shown that the
occurrence of a single CSC will not necessarily result in a tumour, even
if the probability of self-renewal is greater than the probability of
differentiation. This is in contrast to the previous models discussed
(\citeproc{ref-enderling2009}{Enderling et al. 2009};
\citeproc{ref-hillen2013}{Hillen, Enderling, and Hahnfeldt 2013}) and to
a deterministic version of this model
(Equation~\ref{eq-deterministic-master-equation}) that would predict
exponential growth of the tumour from a single CSC.

For larger cellular populations it becomes more challenging to simulate
such a stochastic model and it becomes pertinent to consider the
equations for the average number of CSCs and CCs. Defining the mean
cellular populations \(S = <n_s>\), \(P = <n_p>\) and \(r = r_1 - r_3\)
(i.e., the difference in symmetric self-renewal division and asymmetric
differentiation division), the deterministic model is given by

\begin{equation}\phantomsection\label{eq-deterministic-master-equation}{
\begin{aligned}
\frac{dS}{dt} &= \rho_s r S - \Gamma_s S, \\
\frac{dP}{dt} &= \rho_s (1-r) S - \Gamma_p P.
\end{aligned}
}\end{equation}

This model is largely similar to the ODE model presented in
(\citeproc{ref-hillen2013}{Hillen, Enderling, and Hahnfeldt 2013}),
Equation~\ref{eq-ODE-CSC-model}. The main differences are that
Equation~\ref{eq-deterministic-master-equation} includes a rate of CSC
apoptosis \(\Gamma_s\) and
Equation~\ref{eq-deterministic-master-equation} does not account for
completion for space. Although (\citeproc{ref-turner2009}{Turner et al.
2009}) introduce this into a later version of the model, where they take
\(\rho_s\) to be given by \(\rho_s(S,P) = \rho_s(1 - c_s S - c_p P)\),
where \(1/c_s\) and \(1/c_p\) are the limiting populations of stem and
CCs respectively. However, a limitation of this approach is that it does
not account for competition between CSCs and CCs for space and
resources.

\subsection{A multispecies PDE model of cell
lineages}\label{sec-multispecies-model-of-cell-lineages}

In (\citeproc{ref-youssefpour2012}{Youssefpour et al. 2012}) the authors
develop a multispecies PDE model for CSC lineage dynamics. This is the
first model we have looked at that considers more than the two cell
types, CSCs and CCs. Here the model is more complex and accounts for
CSCs, committed progenitor cells, terminal cells, and dead cells. As
with the previous models we have looked at it is assumes that
differentiation and feedback processes link the cells' lineage through
self-renewal fractions and mitosis rates. The dependent variables in the
model are the local volume fractions of the cell species denoted
\(\phi_\text{CSC}, \phi_\text{CP}, \phi_\text{TC}, \phi_\text{DC}\), as
well as healthy cells and water \(\phi_\text{H}, \phi_\text{W}\).
Assuming there are no voids the sum of the volume fractions equals 1 and
each cell type follows a conservation equation of the form

\begin{equation}\phantomsection\label{eq-youssefpour-multispecies}{
\frac{\partial \phi}{\partial t} = - \nabla \cdot J - \nabla \cdot (u_s \phi) + S
}\end{equation}

where \(\phi\) denotes the volume fraction of the cell type, \(J\) is
generalized diffusion, \(u_s\) is the mass-averaged velocity of the
solid components, \(S\) denotes the mass-exchange terms. This type of
multiphase volume fraction approach has been widely used in cancer
modelling (\citeproc{ref-lowengrub2010}{Lowengrub et al. 2010};
\citeproc{ref-breward2003}{Breward 2003};
\citeproc{ref-hubbard2013}{Hubbard and Byrne 2013}). Simulation of this
model show that the distributions of cell populations obtained from ABM
model (\citeproc{ref-enderling2009}{Enderling et al. 2009}) and if this
mutlspecies model are different. In the ABM CSCs tend to be located
either at the center of small tumour clusters or distributed relatively
uniformly throughout the tumour, some CSCs may migrate past the tumour
boundary and generate new tumour clusters that eventually join the
primary cluster crating an irregular boundary. On the other hand, the
continuum model finds that CSCs tend to be located at the boundary of
sufficient large tumour spheroids, this is consistent with some
experiments (\citeproc{ref-vlashi2009}{Vlashi et al. 2009}). For smaller
spheroids the CSCs may be more uniformly distributed. The patterns of
CSCs and CCs are due to the cell signaling model implemented, detailed
in Section~\ref{sec-differentiation-promoter-and-self-renewal-promoter},
where the CSC self-renewal fraction is regulated by feedback factors
produced by the different cells types
(\citeproc{ref-youssefpour2012}{Youssefpour et al. 2012}).

\section{Models of differentiation
therapy}\label{sec-models-of-differentiation-therapy}

If, as with normal tissues, cellular phenotypic heterogeneity within
tumours can be explained by a hierarchy of differentiation, with only a
subset of stem-like cells capable of long-term self-renewal, this raises
the prospect that signals promoting differentiation could be effective
at driving malignant cells to a less aggressive and ideally post-mitotic
differentiated state (\citeproc{ref-caruxe9n2016}{Carén, Beck, and
Pollard 2016}). This differentiation therapy approach has seen success
in acute promyelocytic leukemia (APL) where all-trans-retinoic acid
(ATRA) can promote differentiation of CSCs and lead to complete
remission (\citeproc{ref-yan2016}{Yan and Liu 2016};
\citeproc{ref-dethuxe92018}{De Thé 2018}). In GBM, bone morphogenetic
protein 4 (BMP4), a member of the TGF-\(\beta\) superfamily, has shown
potential as a differentiation therapeutic agent. BMP4 has been shown to
drive differentiation of GSCs towards a predominantly glial (astrocytic)
fate, to reduce GBM tumour burden \emph{in vivo} and to improve survival
in a mouse model of GBM (\citeproc{ref-nayak2020}{Nayak et al. 2020};
\citeproc{ref-piccirillo2006}{Piccirillo et al. 2006}). Despite its
potential as a treatment option relatively few mathematical models have
considered its possible effects on tumour growth
(\citeproc{ref-youssefpour2012}{Youssefpour et al. 2012};
\citeproc{ref-bachman2013}{Bachman and Hillen 2013};
\citeproc{ref-turner2009}{Turner et al. 2009}).

\subsection{Differentiation promoter and self-renewal
promoter}\label{sec-differentiation-promoter-and-self-renewal-promoter}

In (\citeproc{ref-youssefpour2012}{Youssefpour et al. 2012}) they follow
(\citeproc{ref-lander2009}{Lander et al. 2009}) and assume that the
proliferation and differentiation of CSCs are regulated by factors in
the tumour microenvironemnt that feedback on self-renewal fractions and
mitosis rates. In particular, they denote the differentiation promoter
\(T\) (for TGF-beta superfamily members) that reduces the probability of
self-renewal for CSCs. They also account for self-renewal promoter \(W\)
which increases the probability of self-renewal of CSCs, as well as an
inhibitor of \(W\) denoted \(WI\). Possible self-renewal promoters
include WNTs, Notch and Shh (\citeproc{ref-pannuti2010}{Pannuti et al.
2010}; \citeproc{ref-bailey2007}{Bailey, Singh, and Hollingsworth
2007}). Using the notation introduced in
Section~\ref{sec-multispecies-model-of-cell-lineages}, they define the
CSC self-renewal fraction as

\begin{equation}\phantomsection\label{eq-youssefpour-Ps}{
P_s = P_\text{min} + (P_\text{max} - P_\text{min}) \left( \frac{\xi C_w}{1 + \xi C_W} \right)\left( \frac{1}{1 + \psi C_T} \right),
}\end{equation}

where \(P_\text{min}\) and \(P_\text{max}\) are the minimum and maximum
probability of CSC self-renewal, taken to be \(0.2\) and \(1\)
respectively. \(C_W\) and \(C_T\) represent the concentrations of the
self-renewal promoter and differentiation promoter respectively. The
parameters \(\xi\) and \(\psi\) quantify the sensitivity of CSCs to the
regulating proteins.

The concentrations of differentiation promoter and self-renewal promoter
are then modelled as follows. It is assumed that \(T\) is more diffuse
than either \(W\) or \(WI\). Therefore, on the time scale of cellular
proliferation they make the quasi-steady-state assumption that time
derivatives and advection of \(T\) can be neglected. Thus, the
quasi-steady reaction-diffusion equation for \(C_T\) is given by

\begin{equation}\phantomsection\label{eq-youssefpour-T}{
0 = \nabla \cdot (D_T \nabla C_T) - \left( \nu_{UT} \phi + \nu_{DT} \right) C_T + \nu_{PT}\phi_{TC}
}\end{equation}

where \(D_T\) is the diffusion coefficient, \(\nu_{UT}\), \(\nu_{DT}\)
and \(\nu_\text{PT}\) are the uptake rate by CSCs, the rate of decay and
the rate of production by the terminal cells, respectively.

To model the self-renewal promoter \(W\) and its inhibitor \(WI\) a
generalized Gierer-Meinhard-Turing system of reaction advection
diffusion equations is used, given by

\begin{equation}\phantomsection\label{eq-youssefpour-W-WI}{
\begin{aligned}
&\frac{\partial C_W}{\partial t} + \nabla \cdot (u_s C_W) = \nabla \cdot (D_W \nabla C_W) + f(C_W, C_{WI}), \\
&\frac{\partial C_{WI}}{\partial t} + \nabla \cdot (u_s C_{WI}) = \nabla \cdot (D_{WI} \nabla C_{WI}) + g(C_W, C_{WI}),
\end{aligned}
}\end{equation}

where

\begin{equation}\phantomsection\label{eq-youssefpour-f-g}{
\begin{aligned}
f(C_W, C_{WI}) &= \nu_{PW} \frac{C^2_W}{C_{WI}}C_0 \phi_\text{CSC} - \nu_{DW} C_W + u_0C_0 (\phi_\text{CSC} + \phi_\text{CP} + \phi_\text{TC}), \\
g(C_W, C_{WI}) &= \nu_\text{PWI}C^2_W C_0 \phi_\text{CSC} - \nu_\text{DWI} C_{WI}.
\end{aligned}
}\end{equation}

The parameters \(D_W\) and \(D_{WI}\) are the diffusion coefficients,
\(\nu_\text{PW}\), \(\nu_\text{DW}\) and \(\nu_\text{PWI}\),
\(\nu_\text{DWI}\) are the production and decay rates of \(W\) and
\(WI\) respectively. The parameter \(u_0\) represents a low-level source
of \(W\) from all the tumour cells.

To model differentiation therapy
(\citeproc{ref-youssefpour2012}{Youssefpour et al. 2012}) fix
\(\psi = 0.5\) and introduce an external source of \(T\) i.e., a
constant source term is added to Equation~\ref{eq-youssefpour-T}.

\subsection{No self-renewal promoter}\label{no-self-renewal-promoter}

In (\citeproc{ref-bachman2013}{Bachman and Hillen 2013}) they follow
(\citeproc{ref-youssefpour2012}{Youssefpour et al. 2012}) and model
differentiation therapy through a relationship between the average level
of differentiation promoter, which they denote \(C_F\) and the
probability of CSC self-renewal \(P_s\). However, they do not include
the effects of a CSC self-renewal promoting factor, thus

\begin{equation}\phantomsection\label{eq-bachman-Ps}{
P_s(t) = P_\text{min} + (P_\text{max} - P_\text{min}) \left( \frac{1}{1 + \psi C_F(t)} \right),
}\end{equation}

where the notation is the same as in Equation~\ref{eq-youssefpour-Ps}.
Since (\citeproc{ref-bachman2013}{Bachman and Hillen 2013}) do not model
endogenous production of differentiation promoter, \(C_F\) solely
represents the level of differentiation promoter prescribed during
differentiation therapy. To address this lack of endogenous
differentiation promoters they take \(P_\text{max} = 0.505\) (which is
equivalent to setting \(\delta = 0.01\) as was done in
(\citeproc{ref-hillen2013}{Hillen, Enderling, and Hahnfeldt 2013})) and
\(P_\text{min} = 0.2\), as is done in
(\citeproc{ref-youssefpour2012}{Youssefpour et al. 2012}).

As (\citeproc{ref-bachman2013}{Bachman and Hillen 2013}) use the ODE
model developed in (\citeproc{ref-hillen2013}{Hillen, Enderling, and
Hahnfeldt 2013}) they must also develop a submodel for the average level
of differentiation promoter. As it is an ODE model they consider the
average level of differentiation promoter within a spatially homogeous
tumour \(C_F(t)\). It is assumed that the tumour resides within a
spherical region of tissue and that differentiation promoter enters this
area through the boundary. The differentiation promoter enters the
region from the boundary and will diffuse very quickly and attain a
steady state distribution over this region. To compute the value of
\(C_F(t)\) they solve the problem of diffusion over a sphere of radius
\(R\) and average the solution over the volume of the sphere. We use the
lower case letter to describe the radial symmetric solution \(c_F(r,t)\)
of the following boundary value problem

\begin{equation}\phantomsection\label{eq-bachman-CF}{
\begin{aligned}
\frac{\partial c_F}{\partial t} &= \omega \left( \frac{\partial}{\partial r} \left(\frac{\partial c_F}{\partial r} \right) + \frac{2}{r} \frac{\partial c_f}{\partial r}   \right) \\
c_F(R,t) &= C_{F0}(t),
\end{aligned}
}\end{equation}

where \(\omega\) is the effective diffusivity of the differentiation
promoter. Before differentiation therapy begins \(C_{F0}(t) = 0\), when
differentiation therapy begins the boundary condition on the sphere is
set to \(C_{F0}(t) = 1\), and the promoter diffuses into the sphere.
When differentiation therapy ends, the boundary condition is simply set
to 0 and the promoter diffuses out of the sphere. They then set

\begin{equation}\phantomsection\label{eq-bachman-CF2}{
C_F(t) = \frac{3}{R^3} = \int^R_0 c_F (r,t) r^2 dr.
}\end{equation}

\subsection{BMP4 in glioma}\label{bmp4-in-glioma}

In the previous models (\citeproc{ref-youssefpour2012}{Youssefpour et
al. 2012}; \citeproc{ref-hillen2013}{Hillen, Enderling, and Hahnfeldt
2013}) they considered a general differentiation promoter, in
(\citeproc{ref-turner2009}{Turner et al. 2009}) they consider the
specific differentiation promoter BMP4 in GBM. As in the case for the
general differentiation promoter they interpret the effects of BMP4 as
decreasing the net symmetric division rate \(r\) (following the notation
used in Section~\ref{sec-stochastic-model}). Based on
(\citeproc{ref-piccirillo2006}{Piccirillo et al. 2006}) they estimate
that from a pre-treatment value of \(r = 0.1\) the effect of BMP4 is to
reduce \(r\) to \(-0.1\), note that following the notation used in
Section~\ref{sec-stochastic-model} \(r\) is defined as \(r = r_1-r_3\)
so a change of \(r\) to negative represents a both an increase in the
proportion of symmetric differentiation divisions and a decrease in
symmetric self-renewal divisions. To model differentiation therapy the
parameter value \(r\) is simply switched between these two values for
the duration of BMP4 exposure.

\subsection{Summary of differentiation therapy
results}\label{summary-of-differentiation-therapy-results}

All the models compare 3 different treatment cases radiation alone,
differentiation therapy alone and combination therapy. Importantly, as
has been shown, all models assume that CSCs are less sensitive to
radiation than CCs (\citeproc{ref-bao2006}{Bao et al. 2006};
\citeproc{ref-tang2021a}{Tang et al. 2021b};
\citeproc{ref-hitomi2021a}{Hitomi et al. 2021b}). Despite the slight
differences in assumptions and implementation, all models find similar
results. Radiotherapy alone fails as some CSCs survive and are able to
repopulate the tumour. In fact, all models show an extension of the
``tumour growth paradox'' which we term the ``tumour treatment
paradox''. When treating with radiation alone the fraction of CSCs
increases, since the CCs are more susceptible they significantly reduce
their numbers leaving more room for CSCs. This CSC enriched
post-treatment state allows much more rapid re-growth of the tumour.
This suggests that current standard of care treatment selects for the
more resistant CSCs. Thus treatment often facilitates more rapid and
aggressive tumour recurrence. Differentiation therapy alone can
successfully eradicate the tumour, however, given all models assume that
CSCs can only transform into CCs through cellular division, rather than
direct transition, large intermediate values of total tumour size may be
reached using this approach. Combination of differentiation and
radiotherapy out performed either single therapies, often showing that
the tumour can be driven to much smaller sizes and potentially
extinction. This is because the differentiation agent induces CSCs to
turn into CCs which then can be killed by traditional radiation therapy.
This combination therapy can be considered a new class of strategies for
cancer therapy known as evolutionary steering approaches. Rather than
reactively altering treatment as resistance is acquired we proactively
select our treatment to minimise resistance and increase chance of
extinction (\citeproc{ref-enriquez-navas2015}{Enriquez-Navas,
Wojtkowiak, and Gatenby 2015}; \citeproc{ref-acar2020}{Acar et al.
2020}).

\section{Conclusion}\label{conclusion}

In summary, mathematical models of CSC dynamics provide valuable
insights into tumour growth, dormancy, and response to treatment. While
the models reviewed here vary in their complexity and underlying
assumptions, they all highlight the critical role of CSCs in tumour
maintenance, treatment resistance and tumour recurrence. Mathematical
modelling provides a unique tool for analysing the non-trivial
interaction between CSCs and CCs that can arise from fairly simple
assumptions, allowing us to verify or generate new hypothesis in order
to better understand the role of CSCs.

The differentiation therapy approach aims to drive CSCs within tumours
toward a differentiated state, potentially mitigating their self-renewal
capacity and aggressiveness. Despite this, many challenges remain in
developing differentiation therapies that can be used in a clinical
setting (\citeproc{ref-caruxe9n2016}{Carén, Beck, and Pollard 2016}).
One key challenge particularly in GBM is delivery of differentiation
therapy to the tumour microenvironment, due to the blood brain barrier
(BBB) which limits transport into the brain. New delivery mechanism to
combat this are being developed such as adipose derived mesenchymal stem
cells, containing BMP4 in nanoparticles. These can be delivered directly
into the tumour at the time of resection or systemically as they have
been shown to cross the BBB (\citeproc{ref-li2014a}{Li et al. 2014a};
\citeproc{ref-mangraviti2016a}{Mangraviti et al. 2016a};
\citeproc{ref-pendleton2013a}{Pendleton et al. 2013a}). Additionally as
with all therapies there is vast heterogeneity between patient response
to differentiation therapy, indicating the importance of identifying
biomarkers for sensitivity to such therapies. In work in preparation it
has been shown that GBM cell lines that did not express pRBS were
unresponsive to differentiation therapy (\citeproc{ref-fariasa}{Farias
et al., n.d.a}). Mathematical modelling may also help improve the
effectiveness of differentiation therapy strategies optimising its
delivery in combination with standard cytotoxic therapies as well as
elucidating the mechanism through which CSCs drive tumour growth.

In the next chapter I present our preprint, looking at a model for BMP4
induced differentiation therapy in GBM, which is available on
bioR\(\chi\)iv \href{https://doi.org/10.1101/2024.08.22.609156}{Virtual
Clinical Trials of BMP4 Differentiation Therapy: Digital Twins to Aid
Glioblastoma Trial Design}.

\bookmarksetup{startatroot}

\chapter{Virtual Clinical Trials of BMP4 Differentiation Therapy:
Digital Twins to Aid Glioblastoma Trial
Design}\label{virtual-clinical-trials-of-bmp4-differentiation-therapy-digital-twins-to-aid-glioblastoma-trial-design}

We leverage an integrative mathematical modeling framework to translate
the impact of preclinical findings in virtual clinical trials. We
develop a virtual clinical trial pipeline to face the real-world problem
of numerous of actual early phase clinical trials that have failed for
glioma/glioblastoma, the most common primary brain tumor. Even with the
most promising preclinical data, designing clinical trials is fraught
with challenges, including controlling for the many parameters used to
inform patient selection criteria. Here, we introduce a virtual trial
pipeline that allows us to consider the variability from some of these
criteria that can be used for future trials of novel therapies. As an
example, we apply this to the proposed delivery of BMP4 to stem cell
niches present in glioblastoma, the most aggressive glioma, known for
its inter- and intra-patient heterogeneity. The proposed approach of
BMP4 treatment, delivered through adipose-derived mesenchymal stem
cells, aims to promote cellular differentiation away from the
treatment-resistance stem cell niches towards a more
treatment-vulnerable state. This pipeline will help us narrow down
strategies for future trials, optimize timing of treatments relative to
key standard-of-care treatments, and predict synergy amongst the
developed treatments.

\hfill\break

\section{Introduction}\label{sec-introduction}

Glioblastoma (GBM) is the most commonly diagnosed primary malignant
brain cancer in adults. Current standard of care consists of maximal
safe surgical resection followed by concurrent radiotherapy and
chemotherapy (with temozolomide) (\citeproc{ref-stupp2005}{Stupp et al.
2005}). Despite this aggressive treatment, outcomes remain poor with
median survival of only 15 months (\citeproc{ref-ostrom2019}{Ostrom et
al. 2019}). The aggressiveness and fatal outcomes of GBM can be
attributed to its highly infiltrative nature and to its vast intra- and
inter-patient heterogeneity. Malignant cells can be found infiltrating
far into the peritumoral areas of the brain and are undetectable by
conventional imaging and operative techniques. Furthermore, tumors
contain an array of different cell types with distinct molecular and
phenotypic characteristics, hindering therapy efficacy. In particular,
previous studies have identified a sub-population of malignant glioma
cells with stem-like characteristics known as glioma stem cells (GSCs)
or brain tumor initiating cells (BTICs). These cells are highly
resistant to both radio- and chemo-therapy and therefore contribute to
treatment failure (\citeproc{ref-bao2006}{Bao et al. 2006};
\citeproc{ref-singh2004}{Singh et al. 2004}). Thus, if treatment
outcomes are to be improved for patients with GBM, it is vital that
therapeutics that specifically target these GSCs are developed.

The existence of a population of cancer stem cells (CSCs) was first
established in leukemia (\citeproc{ref-lapidot1994}{Lapidot et al.
1994}; \citeproc{ref-bonnet1997}{Bonnet and Dick 1997}). More recently
CSCs have been identified in many solid tumors including breast, colon
and brain (\citeproc{ref-al-hajj2003}{Al-Hajj et al. 2003};
\citeproc{ref-ricci-vitiani2007}{Ricci-Vitiani et al. 2007};
\citeproc{ref-ignatova2002}{Ignatova et al. 2002};
\citeproc{ref-hemmati2003}{Hemmati et al. 2003};
\citeproc{ref-singh2004}{Singh et al. 2004};
\citeproc{ref-galli2004}{Galli et al. 2004}). In GBM, cells expressing
the CD133 cell surface protein marker (also found on neural stem cells)
have been identified as GSCs based on their exclusive ability to
commence and support tumor growth (\citeproc{ref-singh2004}{Singh et al.
2004}). In addition to being tumor initiating, GSCs are highly resistant
to both radio- and chemo-therapy as they are more efficient at inducing
repair of damaged DNA (\citeproc{ref-bao2006}{Bao et al. 2006};
\citeproc{ref-tang2021}{Tang et al. 2021a}; \citeproc{ref-rich2007}{Rich
2007}; \citeproc{ref-stiles2008}{Stiles and Rowitch 2008};
\citeproc{ref-schonberg2014}{Schonberg et al. 2014};
\citeproc{ref-turner2009}{Turner et al. 2009};
\citeproc{ref-dirks2006}{Dirks 2006}). Furthermore, they also engage in
a synergistic relationship with the surrounding tumor microenvironment
(TME) to promote angiogenesis, proliferation, migration, tumor survival,
and immune evasion (\citeproc{ref-ma2018}{Ma et al. 2018}). Under the
brain cancer stem cell hypothesis, it is becoming increasingly clear
that while radiation may be transiently effective, treatment ultimately
fails in the long run if any GSCs survive
(\citeproc{ref-dingli2006}{Dingli and Michor 2006}).

If, as with normal tissues, cellular phenotypic heterogeneity within
tumors can be explained by a hierarchy of differentiation, with only a
subset of stem-like cells capable of long-term self-renewal
(\citeproc{ref-caruxe9n2016}{Carén, Beck, and Pollard 2016}), this
raises the prospect that signals promoting differentiation could be
effective at driving malignant cells to a less aggressive and ideally
post-mitotic differentiated state. This differentiation therapy approach
has seen success in acute promyelocytic leukemia (APL) where
all-trans-retinoic acid (ATRA) can promote differentiation of CSCs and
lead to complete remission (\citeproc{ref-dethuxe92018}{De Thé 2018};
\citeproc{ref-yan2016}{Yan and Liu 2016}). In GBM, bone morphogenetic
protein 4 (BMP4), a member of the TGF-\(\beta\) superfamily, has shown
potential as a differentiation therapeutic agent. BMP4 has been shown to
drive differentiation of GSCs towards a predominantly glial (astrocytic)
fate, to reduce GBM tumor burden \emph{in vivo} and to improve survival
in a mouse model of GBM (\citeproc{ref-nayak2020}{Nayak et al. 2020};
\citeproc{ref-caruxe9n2016}{Carén, Beck, and Pollard 2016};
\citeproc{ref-piccirillo2006}{Piccirillo et al. 2006}). The precise
mechanism through which BMP4 acts is unknown but a possible explanation
is that it reduces the frequency of symmetric divisions of GSCs
(\citeproc{ref-guerrero-cuxe1zares2014}{Guerrero-Cázares et al. 2014}),
and previous mathematical models of differentiation therapy have assumed
that differentiation promoters act in this way
(\citeproc{ref-youssefpour2012}{Youssefpour et al. 2012};
\citeproc{ref-bachman2013}{Bachman and Hillen 2013}).

Adipose derived mesenchymal stem cells (AMSCs) provide a possible
alternative to traditional treatment, as these cells preferentially
migrate toward areas of malignancy, and can be utilized to home towards
the infiltrating glioma cells (\citeproc{ref-li2014}{Li et al. 2014b};
\citeproc{ref-mangraviti2016}{Mangraviti et al. 2016b};
\citeproc{ref-pendleton2013}{Pendleton et al. 2013b};
\citeproc{ref-smith2015}{Smith et al. 2015};
\citeproc{ref-doucette2011}{Doucette et al. 2011}). Additionally AMSCs
can be engineered to secrete BMP4 for GBM therapy, and a migration assay
showed that compared to controls, engineered AMSCs significantly
retained their tropism and preferential migration towards GBM factors
(\citeproc{ref-mangraviti2016}{Mangraviti et al. 2016b};
\citeproc{ref-kim2020}{J. Kim et al. 2020};
\citeproc{ref-guerrero-cuxe1zares2014}{Guerrero-Cázares et al. 2014};
\citeproc{ref-tzeng2011}{Tzeng et al. 2011}). It has been shown that
nanoparticle-engineered AMSCs maintain their multi-potency
characteristics and release their therapeutic cargo progressively,
furthermore systemically-delivered engineered AMSCs can cross the
blood-brain barrier and retain their preferential tropism towards
gliomas (\citeproc{ref-mangraviti2016}{Mangraviti et al. 2016b}).
Therefore, AMSCs delivery promises the ability to hone in on GSC niches
and locally release treatment to these regions. BMP4-loaded AMSCS
(BMP4-AMSCs) would theoretically migrate to GSCs, release BMP4, and
increase the rate of their differentiation, reducing the number of GSCs
and increasing sensitivity to standard treatments.

Clinical trials typically consist of four phases in which a new
intervention is investigated in human subjects to determine its safety
and efficacy. This system is notoriously resource intensive and
inefficient. The average cost per patient is \$59,500 and takes more
than 10 years, with only 10\% of drugs in phase 1 studies eventually
approved (\citeproc{ref-dowden2019}{Dowden and Munro 2019};
\citeproc{ref-yankeelov2024}{Yankeelov et al. 2024}). In GBM, the
statistics are even worse; multiple recent phase 3 trials have failed to
meet their prespecified primary endpoints
(\citeproc{ref-bagley2022}{Bagley et al. 2022};
\citeproc{ref-reardon2020}{Reardon et al. 2020};
\citeproc{ref-cloughesy2020}{Cloughesy et al. 2020};
\citeproc{ref-weller2017}{Weller et al. 2017};
\citeproc{ref-narita2019}{Narita et al. 2019};
\citeproc{ref-roth2021}{Roth et al. 2021}). Of course, there are many
reasons for these failures including inter- and intra-patient
variability, drug delivery limitations, paucity of control arms, overly
stringent clinical eligibility, and beyond
(\citeproc{ref-bagley2022}{Bagley et al. 2022}).

Increasingly, it is being realised that mathematical modelling and
\emph{in silico} clinical trials can assist in addressing a number of
these problems. Subfields such as virtual clinical trials
(\citeproc{ref-craig2023}{Craig et al. 2023}), phase \emph{i} trials
(\citeproc{ref-kim2016}{E. Kim et al. 2016}), and digital twins
(\citeproc{ref-wu2022}{Wu et al. 2022}) have evolved all with the aim to
create simulated patients or patient cohorts that can be used as a
surrogate to predict the effects of treatment on a more personalized
level. This approach has been applied to a variety of cancers, including
breast cancer (\citeproc{ref-wang2019}{Wang et al. 2019};
\citeproc{ref-wang2020}{Wang et al. 2020}), head and neck cancer
(\citeproc{ref-zahid2021}{Zahid et al. 2021}), melanoma
(\citeproc{ref-kim2016}{E. Kim et al. 2016}), and ovarian cancer
(\citeproc{ref-cardinal2022}{Cardinal et al. 2022}). Here, we take a
similar \emph{in silico} approach to glioblastoma and its proposed
treatment with BMP4-AMSCs.

\section{Materials and methods}\label{sec-materials-methods}

We use mathematical modelling to simulate GSC-driven brain tumor growth.
Our model describes the interplay of GSCs (\(s(t)\)) and non-GSCs
(\(v(t)\)). To describe radiation therapy we use the well-established
linear quadratic model (\citeproc{ref-rockne2009}{R. Rockne et al.
2009}; \citeproc{ref-mcmahon2018}{McMahon 2018}) with realistic standard
treatments (five treatments per week, weekends off, for six weeks) and
with tissue-specific radiosensitivity parameters. Following
(\citeproc{ref-bachman2013}{Bachman and Hillen 2013};
\citeproc{ref-youssefpour2012}{Youssefpour et al. 2012}) we model
differentiation therapy as decreasing the propensity for self-renewal of
GSCs.

\subsection{Pre-clinical data}\label{sec-pre-clinical-data}

To quantify the effect of BMP4 on sensitising GSCs, we used data
collected from a clonogenic radiotherapy (RT) assay
(\citeproc{ref-farias}{Farias et al., n.d.b}). Five GSC cell lines were
treated with either complete media (untreated control), BMP4 (100
ng/ml), BMP4-AMSCs conditioned media (100 ng/ml BMP4), or a similar
volume of non-engineered AMSC conditioned media for a period of 48
hours, then seeded at 250-1250 cells per well in 96 well plates and
treated with 0, 2, 4, 6 Gy of radiation, with six replicates in each
case. After 14 days the colonies with more than 100 nm diameter were
manually counted.

Typically, the linear quadratic (LQ) model is used to describe the
surviving fraction of cells after radiation. However, this assumes only
one cell type is present (\citeproc{ref-rockne2009}{R. Rockne et al.
2009}; \citeproc{ref-mcmahon2018}{McMahon 2018};
\citeproc{ref-yu2015}{Yu et al. 2015}). In the control case, GSC lines
are cultured in stem cell media, so we assume that cells will remain as
GSCs unless any BMP4 or other differentiation promoter is present, thus
the LQ model is applicable. However, when BMP4 is applied some amount of
differentiation of GSCs will take place that explains the increased
sensitivity to RT. To get an estimate for the amount of differentiation
that took place over 48hr exposure, we fit a dual-linear quadratic (DLQ)
model similar to that of (\citeproc{ref-yu2015}{Yu et al. 2015}), given
by:

\begin{equation}\phantomsection\label{eq-DLQ}{
  \gamma(d) = F \text{exp}(\eta(-\alpha d - \beta d^2)) + (1-F)\text{exp}(-\alpha d - \beta d^2)
}\end{equation}

where \(d\) is the RT dose, \(F\) is the fraction of GSCs and \(\eta\)
is the protection factor for GSCs that accounts for the fact that GSCs
are less sensitive to radiation than other cancer cells; previous
experiments have estimated GSC radio-protection to be \(\eta = 0.1376\)
(\citeproc{ref-gao2013}{Gao et al. 2013}). We infer the radiobiological
parameter from the cell line data in the following way. First, we assume
that since the GSC lines are cultured in stem cell media no
differentiation will take place, therefore we fit the LQ model to the
control case for each cell line, assuming the ratio
\(\alpha / \beta = 10\) (\citeproc{ref-rockne2010}{R. Rockne et al.
2010}), to get an estimate for \(\alpha\). These form the
cell-line-specific radiosensitivity parameters. Using these parameters
we fit the DLQ model to the BMP4 treated survival data, treating
\(\alpha\) and \(\beta\) as fixed constants for each cell line from the
control case so that we are only fitting for \(F\). This allows us to
get an estimate of the fraction of GSCs after 48hrs exposure to BMP4 in
each cell line. The fitted \(\alpha\) and \(F\) for all the cell lines
are shown in Table~\ref{tbl-RT-data}.

\subsection{Model assumptions and
equations}\label{sec-model-assumptions}

\begin{itemize}
\item
  GSCs have unlimited replicative potential.
\item
  GSCs are capable of both self-renewing and differentiating. Using the
  same notation as (\citeproc{ref-youssefpour2012}{Youssefpour et al.
  2012}) we denote the proportion of GSCs that self-renew by \(P_s\).
  This does not distinguish between symmetric and asymmetric division,
  but rather just considers the overall fraction of proliferating cells
  that self-renew (\(P_s\)) and differentiate (\(1 - P_s\)).
\item
  Progenitor cells (PCs) have a limited replicative potential, this
  ensures that they are not tumor-initiating. Following
  (\citeproc{ref-enderling2009}{Enderling et al. 2009};
  \citeproc{ref-gao2013}{Gao et al. 2013}), the proliferative potential
  of PCs is set to a maximum of \(n = 10\) divisions. Once PCs have
  divided \(n=10\) times they become terminally differentiated (TC) and
  can no longer proliferate.
\end{itemize}

Figure~\ref{fig-model_schematic} shows a schematic of this model.
Following these assumptions, we derive the following equations,

\begin{equation}\phantomsection\label{eq-gsc-model}{
\begin{aligned}
  \underbrace{\frac{d s}{d t}}_{
    \substack{\text{Rate of} \\ \text{change GSCs}}} &=  \underbrace{(2P_s -1)  m_s s \left( 1 - \frac{N}{k}  \right)}_\text{Self-renewal of GSCs} - \underbrace{\delta_{s} s}_\text{Apoptosis}, \\
  \underbrace{\frac{d v_1}{d t}}_{
    \substack{\text{Rate of} \\ \text{change PCs}}} &= \underbrace{2(1- P_s)  m_s s \left( 1 - \frac{N}{k}  \right)}_{
    \substack{\text{Differentiation} \\ \text{of GSCs}}} - \underbrace{m_1 v_1 \left( 1 - \frac{N}{k} \right)}_\text{Proliferation of PCs} - \underbrace{\delta_{1} v_1}_\text{Apoptosis}, \\
  \underbrace{\frac{d v_i}{d t}}_{
    \substack{\text{Rate of} \\ \text{change PCs}}} &= \underbrace{2 m_{i-1} v_{i-1} \left( 1 - \frac{N}{k}  \right)}_{
    \substack{\text{Proliferation of} \\ \text{PCs}}} - \underbrace{m_i v_i \left( 1 - \frac{N}{k} \right)}_\text{Proliferation of PCs} - \underbrace{\delta_{i} v_i}_\text{Apoptosis}, \; i=2,...,n-1 \\
  \underbrace{\frac{d v_n}{d t}}_{
    \substack{\text{Rate of} \\ \text{change PCs}}} &= \underbrace{2 m_{n-1} v_{n-1} \left( 1 - \frac{N}{k}  \right)}_{
    \substack{\text{Differentiation} \\ \text{of PCs}}} - \underbrace{\delta_{n} v_n}_\text{Apoptosis}, \\
\end{aligned} 
}\end{equation}

where \(s(t)\) is the density of GSCs, \(v_i(t)\) is the density of PCs
(at proliferation level \(i\)) and \(v_n(t)\) are terminally
differentiated cells. The total density of the tumor is given by
\(N(t) = s + \sum_{i=1}^{n} v_i\). The apoptosis rate for GSCs is
\(\delta_s\) and for PCs are given by \(\delta_i\). The proliferation
rates of GSCs and PCs are given by \(m_s\), \(m_i\) respectively, where
we assume all \(m_i\) are equal. These proliferation rates are subject
to a crowding response, represented by the term \(1 - N/k\) where \(k\)
is the carrying capacity of the tumor.

\begin{figure}

\centering{

\includegraphics{images/model_schema.png}

}

\caption{\label{fig-model_schematic}Model schematic, showing how GSCs
differentiate into progenitor cancer cells which can then divide a
number of times, doubling in number each time
(Equation~\ref{eq-gsc-model}). Eventually, PCs become terminally
differentiated and are no longer able to proliferate. Introduced AMSCs
gradually die and release BMP4 (Equation~\ref{eq-AMSC-BMP4-model}) which
modifies the self-renewal rate \(P_s\) according to
Equation~\ref{eq-Ps}. (created with BioRender.com)}

\end{figure}%

\subsubsection{Differentiation therapy with mesenchymal stem cell
delivery model}\label{sec-differentiation-therapy}

A possible delivery mechanism for BMP4 is via AMSCs. These could be
implanted at the time of resection, or systemically with the ability to
cross the blood-brain barrier and subsequently diffuse with preferential
tropism towards glioma cells. We model AMSCs as simply decaying
exponentially from an initial concentration at implantation. BMP4 is
released from these AMSCs and taken up by GSCs. The equations for both
the AMSC and BMP4 are thus given by.

\begin{equation}\phantomsection\label{eq-AMSC-BMP4-model}{
\begin{aligned}
    \underbrace{\frac{d m}{d t}}_{
    \substack{\text{Rate of} \\ \text{change AMSCs}}} &= \underbrace{-\delta_m m}_\text{Decay of AMSCs} \\ 
    \underbrace{\frac{d B}{d t}}_{
    \substack{\text{Rate of} \\ \text{change BMP4}}} &=  \underbrace{C m}_\text{Release of BMP4} - \underbrace{u_sBs}_\text{uptake by GSCs} - \underbrace{\delta_B B}_\text{Decay of BMP4}, 
\end{aligned} 
}\end{equation}

where \(\delta_m\) is the decay rate of AMSCs, \(C\) is the rate at
which AMSCs release BMP4, \(u_s\) is the uptake rate of BMP4 by GSCs and
\(\delta_B\) is the decay rate of BMP4. \emph{In vivo} experiments have
shown that AMSCs can survive for around 14 days in a rodent model and
that BMP4 reaches its peak concentration at around 48hrs after initial
implantation of AMSC (\citeproc{ref-li2014}{Li et al. 2014b}).

\subsubsection{BMP4 model}\label{sec-BMP4-model}

Following (\citeproc{ref-bachman2013}{Bachman and Hillen 2013}), we
model differentiation therapy through a simple relationship between the
level of differentiation promoter (BMP4), and the probability of GSC
self-renewal, \(P_s\), and do not consider the effect of a GSC promoter
such as WNT (\citeproc{ref-lee2016}{Lee et al. 2016};
\citeproc{ref-youssefpour2012}{Youssefpour et al. 2012}). Therefore, the
relationship between \(P_s\) and BMP4 is given by

\begin{equation}\phantomsection\label{eq-Ps}{
    P_s(t) = P_\text{min} + (P_\text{max} - P_\text{min})  \left( \frac{1}{1 +  \psi B(t)} \right),
}\end{equation}

where \(P_\text{min}\) and \(P_\text{max}\) are the minimum and maximum
self-renewal probabilities and \(B(t)\) represents the concentration of
BMP4. \(P_\text{max}\) is attained if there is no BMP4 present, while
\(P_\text{min}\) is approached as \(B \rightarrow \infty\). We do not
consider any endogenous production of BMP4 (or other differentiation
promoter), therefore it is only present during differentiation therapy.
The parameter \(\psi\) represents the sensitivity of GSCs to BMP4, as
\(\psi\) increases (for the same concentration of BMP4) \(P_\text{min}\)
is approached faster. Any other potential effects of BMP4, for example
on growth rates, are ignored as they are in
(\citeproc{ref-youssefpour2012}{Youssefpour et al. 2012}) and
(\citeproc{ref-bachman2013}{Bachman and Hillen 2013}).

\subsubsection{Radiotherapy model}\label{sec-RT-model}

We model the effects of radiotherapy using the linear quadratic (LQ)
model, which is widely used in mathematical modelling of RT
(\citeproc{ref-rockne2010}{R. Rockne et al. 2010};
\citeproc{ref-rockne2009}{R. Rockne et al. 2009};
\citeproc{ref-orourke2009}{O'Rourke, McAneney, and Hillen 2009};
\citeproc{ref-mcmahon2018}{McMahon 2018}). The fraction of cells that
survive, \(\gamma_{\text{rad}}(d)\) after a single fractional dose of,
\(d\) Grays (Gy) of radiation is given by

\begin{equation}\phantomsection\label{eq-LQ-model}{
    \gamma_{\text{rad}}(d) = \text{exp}\left(-\eta \mu (\alpha d + \beta d^2)\right),
}\end{equation}

where \(\alpha\) can be interpreted as death from single-stranded breaks
(linear component) and \(\beta\) can be interpreted as death from
double-strand breaks (quadratic component). It has been shown that the
value of \(\alpha\) correlates with the proliferation rate of the tumor.
We can use this linear relationship to estimate \(\alpha\) on a
patient-specific level as \(\alpha = 0.05m_v\)
(\citeproc{ref-rockne2010}{R. Rockne et al. 2010}). To account for the
fact that GSCs are less sensitive to radiation than other cancer cells,
we include the additional radio-protection parameter \(\eta\). Previous
experiments have estimated GSC radio-protection to be \(\eta = 0.1376\)
(\citeproc{ref-gao2013}{Gao et al. 2013}; \citeproc{ref-bao2006}{Bao et
al. 2006}). Additionally, the model contains cells that are terminally
differentiated and so do not proliferate. RT primarily targets actively
proliferating cells so is less effective against non-proliferating
cells. We include \(\mu\) to account for this radio-protection of
non-proliferating cells, previous experiments have found this to be
around \(\mu = 0.5\) (\citeproc{ref-gao2013}{Gao et al. 2013};
\citeproc{ref-griffin2006}{Griffin 2006};
\citeproc{ref-potten1981}{Potten 1981}).

It is assumed that all effects of radiation on tumor cell density are
instantaneous, and no delay or otherwise toxic effects of radiotherapy
are considered. Nor are any effects on proliferation rate as a result of
radiotherapy considered. Therefore, we model the effects of radiotherapy
as an instantaneous reduction of tumor density. Considering each
compartment of the model (Equation~\ref{eq-gsc-model}) separately we
modify them by

\begin{equation}\phantomsection\label{eq-radiation_model}{ 
  \begin{aligned}
    s_\text{post-rad} = s_\text{pre-rad} \text{exp}(-\eta(\alpha d + \beta d^2)) \\ 
    v_{i_\text{post-rad}} = v_{i_\text{pre-rad}} \text{exp}(-(\alpha d + \beta d^2)) \\ 
    v_{n_\text{post-rad}} = v_{n_\text{pre-rad}} \text{exp}(-\mu(\alpha d + \beta d^2))
  \end{aligned} 
}\end{equation}

\subsubsection{Resection model}\label{sec-resection-model}

Similarly to radiotherapy we model resection as an instantaneous loss of
density, applied to each model compartment separately as

\begin{equation}\phantomsection\label{eq-resection_model}{
  \begin{aligned}
    s_\text{post-resect} = s_\text{pre-resect} \gamma_\text{res} \\ 
    v_{i,\text{post-resect}} = v_{i,\text{pre-resect}} \gamma_\text{res} \\ 
    v_{n,\text{post-resect}} = v_{n,\text{pre-resect}} \gamma_\text{res},
  \end{aligned} 
}\end{equation}

where \(\gamma_{\text{res}}\) is the surviving fraction after resection.
Chaichana et al.~investigated the efficacy of resection found that on
average resection resulted in a \(91.7 \%\) reduction in tumor volume,
so we take \(\gamma_{\text{res}} = (1- 0.917)\)
(\citeproc{ref-chaichana2014}{Chaichana et al. 2014}).

\section{Results}\label{sec-results}

We explore the model for a range of different parameter values to help
us identify possible strategies for patient selection in early phase
clinical trials of BMP4 therapy, as well as explore different delivery
schedules, to increase the likelihood of observing successful clinical
trials. We parameterise our model to 5 GBM cell lines where we have
known doubling times, radiotherapy response, and exposure to BMP4. This
allows us to estimate reasonable values of sensitivity to BMP4. We
develop a virtual clinical trial pipeline that allows us to assess the
likelihood of observing a successful trial for different patient
populations.

\subsection{Simulating radiotherapy
experiments}\label{sec-simulating-RT-experiments}

To parameterise the model we simulate the radiotherapy assay described
in Section~\ref{sec-pre-clinical-data}. In the assay a small number of
cells were seeded and allowed to grow for 48hrs exposed to either CTRL
(GSC media) or BMP4 (100ng/ml) then radiotherapy at 0,2,4,6 Gy was
applied. The number of surviving colonies was then counted. To simulate
this we initialise our model with a small density of GSCs. The cells are
cultured in GSC media in the CTRL case so we assume no differentiation
takes place before treatment (i.e.~\(P_s = 1\)). We simulate the BMP4 as
a constant concentration for the 48hrs. From the DLQ model
(Equation~\ref{eq-DLQ}) we have an estimate for what the fraction of
GSCs should be after the 48hrs and given that we also have the doubling
time for each cell line we can fit the parameter \(\psi\), which tells
us the change in GSC self-renewal (\(P_s\)) over the 48hrs.

The fitted \(\psi\) values for each of the 5 cell lines are shown in
Figure~\ref{fig-psi_values_visualised} and in Table~\ref{tbl-RT-data}.
We find that each cell line has a distinct sensitivity to BMP4, in
agreement with work in preparation (\citeproc{ref-farias}{Farias et al.,
n.d.b}) which shows that GBM1a, QNS657 and QNS120 are sensitive to BMP4
while QNS315 and QNS108 are resistant to BMP4 treatment.

\begin{figure}

\begin{minipage}{0.50\linewidth}

\centering{

\includegraphics{images/png/simulated_RT_assay_GBM1a.png}

}

\subcaption{\label{fig-sim_RT_assay_GBM1a}}

\end{minipage}%
%
\begin{minipage}{0.50\linewidth}

\centering{

\includegraphics{images/png/simulated_dose_response_GBM1a.png}

}

\subcaption{\label{fig-sim_dose_response_GBM1a}}

\end{minipage}%
\newline
\begin{minipage}{\linewidth}

\centering{

\includegraphics{images/png/psi_values_visualised.png}

}

\subcaption{\label{fig-psi_values_visualised}}

\end{minipage}%

\caption{\label{fig-days_gained_example_sim}Simulated radiotherapy
assay. (a) Initially a small number of cells were seeded (represented by
a small density of tumor cells). These were grown for 2 days under
either control or 100ng/ml BMP4. After 2 days, radiotherapy was applied
at different dosages and the number of surviving colonies were counted.
The red star indicates when RT was applied (48 hrs) and the green stars
indicate the final measured density of tumor immediately after
radiotherapy for different doses of radiation. (b) Simulated dose
response to radiotherapy after fitting \(\psi\) in the ODE model to
provide the expected fraction of GSCs given the doubling time of the
cell line. (c) Each of the cell lines has a distinct sensitivity to BMP4
denoted \(\psi\).}

\end{figure}%

\begin{longtable}[]{@{}
  >{\raggedright\arraybackslash}p{(\columnwidth - 10\tabcolsep) * \real{0.1667}}
  >{\raggedright\arraybackslash}p{(\columnwidth - 10\tabcolsep) * \real{0.1667}}
  >{\raggedright\arraybackslash}p{(\columnwidth - 10\tabcolsep) * \real{0.1667}}
  >{\raggedright\arraybackslash}p{(\columnwidth - 10\tabcolsep) * \real{0.1667}}
  >{\raggedright\arraybackslash}p{(\columnwidth - 10\tabcolsep) * \real{0.1667}}
  >{\raggedright\arraybackslash}p{(\columnwidth - 10\tabcolsep) * \real{0.1667}}@{}}
\caption{Fitted parameter values and metadata from cell lines. All cell
lines are from primary tumor. It is assumed that \(\alpha/\beta = 10\)
is fixed for all cell lines.}\label{tbl-RT-data}\tabularnewline
\toprule\noalign{}
\begin{minipage}[b]{\linewidth}\raggedright
Cell line
\end{minipage} & \begin{minipage}[b]{\linewidth}\raggedright
\(\alpha\)
\end{minipage} & \begin{minipage}[b]{\linewidth}\raggedright
F
\end{minipage} & \begin{minipage}[b]{\linewidth}\raggedright
Doubling time (hrs)
\end{minipage} & \begin{minipage}[b]{\linewidth}\raggedright
Sensitivity to BMP4 (\(\psi\))
\end{minipage} & \begin{minipage}[b]{\linewidth}\raggedright
Sex
\end{minipage} \\
\midrule\noalign{}
\endfirsthead
\toprule\noalign{}
\begin{minipage}[b]{\linewidth}\raggedright
Cell line
\end{minipage} & \begin{minipage}[b]{\linewidth}\raggedright
\(\alpha\)
\end{minipage} & \begin{minipage}[b]{\linewidth}\raggedright
F
\end{minipage} & \begin{minipage}[b]{\linewidth}\raggedright
Doubling time (hrs)
\end{minipage} & \begin{minipage}[b]{\linewidth}\raggedright
Sensitivity to BMP4 (\(\psi\))
\end{minipage} & \begin{minipage}[b]{\linewidth}\raggedright
Sex
\end{minipage} \\
\midrule\noalign{}
\endhead
\bottomrule\noalign{}
\endlastfoot
GBM1a & 0.338 & 0.571 & 54.7 & 0.00850409 & Male \\
QNS120 & 0.116 & 0.770 & 43.5 & 0.002094342 & Male \\
QNS108 & 0.151 & 0.949 & 109 & 0.000918066 & Male \\
QNS315 & 0.0841 & 1 & 63.6 & 0 & Female \\
QNS657 & 0.104 & 0.707 & 75.6 & 0.006332659 & Female \\
\end{longtable}

\subsection{Model simulations}\label{sec-model-sims}

We simulate our model for a range of parameter values to explore the
effect of BMP4 on tumor growth. Based on our fitted values of \(\psi\)
we consider a range from \([0,0.1]\), the full parameter values are
given in Table~\ref{tbl-params}. To simulate treatment we allow the
model to run until the total tumor size reaches \(0.2\), then initiate
treatment. To evaluate the effect of BMP4, we compare two treatment
arms: standard of care - resection (at the time of detection) followed
by radiotherapy 30 days later, and BMP4 - standard of care with the
addition of BMP4. To observe the effect of BMP4, we assume a fixed
concentration of AMSCs, which release BMP4 at a constant rate, from the
time of resection until the end of radiotherapy. In each case we record
the number of days until the tumor reaches a size of \(0.6\) and refer
to this as time to progression. This allows us to calculate the fold
change in time to progression to directly compare for each set of
parameters how much the BMP4 prolonged survival. The results are shown
in Figure~\ref{fig-days_gained}.

Since GSCs are less sensitive to RT than other cells, the fraction of
GSCs increases during RT; this is shown in our model in
Figure~\ref{fig-example_sims}. In particular, radiotherapy is more
effective against faster proliferating tumors
(\citeproc{ref-rockne2010}{R. Rockne et al. 2010}) so this effect is
particularly pronounced in these tumors. Enrichment of GSCs during RT
not only highlights that there are some resistant cells not being
targeted by radiotherapy but also since these are the most tumorigenic
cells, with a large proportion of GSCs, a recurrent tumor is able to
form rapidly (as compared to not at all if no GSCs were present). When
we simulated the addition of BMP4 (the dashed lines in
Figure~\ref{fig-example_sims}) we see that this peak in stem-cell
fraction is reduced due to the induced differentiation of GSCs. This
means that not only is the RT more effective as there are fewer
resistant GSCs but also that the tumor will take longer to recur as
there are fewer GSCs to drive regrowth. As the increase in GSC fraction
is most pronounced in faster proliferating tumors, we see that the
relative effect of BMP4 in delaying progression is also largest for
these patients.

\begin{longtable}[]{@{}
  >{\raggedright\arraybackslash}p{(\columnwidth - 8\tabcolsep) * \real{0.2000}}
  >{\raggedright\arraybackslash}p{(\columnwidth - 8\tabcolsep) * \real{0.2000}}
  >{\raggedright\arraybackslash}p{(\columnwidth - 8\tabcolsep) * \real{0.2000}}
  >{\raggedright\arraybackslash}p{(\columnwidth - 8\tabcolsep) * \real{0.2000}}
  >{\raggedright\arraybackslash}p{(\columnwidth - 8\tabcolsep) * \real{0.2000}}@{}}
\caption{Table of parameter values used. When the value is fixed its
value is given, if it is sampled from a distribution then the
distribution is given.}\label{tbl-params}\tabularnewline
\toprule\noalign{}
\begin{minipage}[b]{\linewidth}\raggedright
Parameter
\end{minipage} & \begin{minipage}[b]{\linewidth}\raggedright
Meaning
\end{minipage} & \begin{minipage}[b]{\linewidth}\raggedright
Value / distribution
\end{minipage} & \begin{minipage}[b]{\linewidth}\raggedright
Units
\end{minipage} & \begin{minipage}[b]{\linewidth}\raggedright
Reference
\end{minipage} \\
\midrule\noalign{}
\endfirsthead
\toprule\noalign{}
\begin{minipage}[b]{\linewidth}\raggedright
Parameter
\end{minipage} & \begin{minipage}[b]{\linewidth}\raggedright
Meaning
\end{minipage} & \begin{minipage}[b]{\linewidth}\raggedright
Value / distribution
\end{minipage} & \begin{minipage}[b]{\linewidth}\raggedright
Units
\end{minipage} & \begin{minipage}[b]{\linewidth}\raggedright
Reference
\end{minipage} \\
\midrule\noalign{}
\endhead
\bottomrule\noalign{}
\endlastfoot
\(\delta_s\) & death rate of GSCs & 0.001 & 1 / year & Estimated \\
\(\delta_i\) & death rate of PCs & 0.01 & 1 / year & Estimated \\
\(\delta_n\) & death rate of TCs & 0.1 & 1 / year & Estimated \\
\(\delta_m\) & death rate of AMSCs & 0.5 & 1 / year &
(\citeproc{ref-li2014}{Li et al. 2014b}) \\
\(\delta_B\) & Decay rate of BMP4 & 0.5 & 1 / year &
(\citeproc{ref-li2014}{Li et al. 2014b}) \\
\(u_B\) & Uptake rate of BMP4 & 0.5 & 1 / year & Estimated \\
\(C\) & Release rate of BMP4 from AMSCs & 0.5 & 1 / year & Estimated \\
n & Number of division of PCs & 10 & - & (\citeproc{ref-gao2013}{Gao et
al. 2013}) \\
\(m_i\) & Proliferation rate of PCs & \(\text{log-N}(2.75,0.51)\) & 1 /
year & (\citeproc{ref-yang2019}{Yang et al. 2019}) \\
\(m_s\) & Proliferation rate of GSCs & \(0.0345 m_i\) & 1 / year &
Estimated \\
\(\alpha\) & Radiosensitivity & \(0.05 m_i\) & 1 / Gy &
(\citeproc{ref-rockne2010}{R. Rockne et al. 2010}) \\
\end{longtable}

\begin{figure}

\begin{minipage}{0.50\linewidth}

\centering{

\includegraphics{images/png/example_sims.png}

}

\subcaption{\label{fig-example_sims}}

\end{minipage}%
%
\begin{minipage}{0.50\linewidth}

\centering{

\includegraphics{images/png/days_gained.png}

}

\subcaption{\label{fig-days_gained}}

\end{minipage}%

\caption{\label{fig-effect_BMP4}Effect of BMP4 on tumor progression. (a)
Example model simulations showing the total tumor density and fraction
of GSCs. Red stars indicate when the tumor was detected at \(N=0.2\) and
the blue dot represents the start of radiotherapy (30 days later).
Simulated standard of care (resection and radiotherapy) is plotted in
solid lines and BMP4-AMSCs is in the dashed lines. During radiotherapy
the fraction of GSCs is greatly increased as they are less sensitive
than the non-GSCs. When BMP4 is added this enrichment in GSCs is
reduced, delaying time for tumor regrowth. (b) Days gained surface. As
we increase both the sensitivity to BMP4 (\(\psi\)) and the
proliferation rate the fold change in time to progression increases.}

\end{figure}%

\subsection{Virtual clinical trial
pipeline}\label{sec-virtual-trial-pipeline}

Firstly we simulate a large cohort of virtual patients with fix
sensitivity to BMP4 based on the parameters identified in
Section~\ref{sec-simulating-RT-experiments}. We compare our simulated
BMP4 patients to virtual controls to show that BMP4 can delay tumor
growth. We then develop a pipeline for simulating early-phase 2 clinical
trials and calculating the probability of observing a successful trial.

\subsubsection{Uncertainty in tumor detection and
death}\label{sec-unce-death-detect}

We assume that both detection of the tumor and death depend on tumor
size in a random fashion for each virtual patient. Detection (death) is
more likely the bigger a tumor is, but two tumors of equal size in two
patients do not necessarily lead to detection (death) at the same times.
Thus, we assume that the times of tumor detection and death, represented
by the random variables \(T_{\text{detect}}\) and \(T_{\text{death}}\),
depend on total tumor density \(N(t)\) according to the hazard function

\begin{equation}\phantomsection\label{eq-detect_death_hazard}{
  P(T_d \in [t,t+\Delta t) | T_d>t) = \lambda_d (N(t)) \Delta t + o(\Delta t), \quad d \in \{\text{detect},\text{death}\},
}\end{equation}

where

\begin{equation}\phantomsection\label{eq-lambda_N}{
  \lambda_d (N) = \frac{\lambda_{d,\text{max}}}{1 + e^{-m_d(N - N_d)}}.
}\end{equation}

Here, \(\lambda_d(N)\) is a shifted logistic function;
\(\lambda_{d,\text{max}}\) is the maximum rate of detection that we take
to be 1, \(m_d\) describes the steepness of the logistic function, which
we set to be 100 in the detection case and 20 in the death case. The
constant \(N_d\) is a threshold parameter at which the probability rate
of detection or death is half-maximal (each of these for
\(d\in\{\text{detect},\text{death}\}\)), we set this to be 0.2 and 0.7
in the detection and death cases respectively. \(\Delta t\) is the
timestep on which the model is solved numerically. This is similar to
the approach of (\citeproc{ref-bartoszynski2001}{Bartoszyński et al.
2001}; \citeproc{ref-plevritis2006}{Plevritis et al. 2006}), apart from
our choice of nonlinear dependence on tumor size. The shifted logistic
function acts as a switch mechanism, meaning once \(N > N_d\) the rate
of detection or death rapidly increases towards its maximum.
Figure~\ref{fig-detect_death} shows the form of these functions with the
parameters used in the virtual trials that follow.

\subsubsection{Simulated control virtual patient
population}\label{sec-simulation-controls}

To construct a group of control patients, we sample proliferation rates
from a distribution consistent with measured proliferation rates of
around 300 patients (\citeproc{ref-yang2019}{Yang et al. 2019}), from
this distribution we sample 200 patients. We then simulate these
patients undergoing resection and radiotherapy (standard of care); to
consider the effect of BMP4 on survival we assume a constant
concentration of 100ng/ml is maintained from detection until the end of
radiotherapy and we consider two different sensitivities
\(\psi = 0.0021\) and \(\psi = 0.0085\), these values correspond with
the cell lines QNS120 and GBM1a from Table~\ref{tbl-RT-data}. The
results are shown in Figure~\ref{fig-simulated-controls}. We see that
BMP4 improves simulated survival times and that as the sensitivity to
BMP4 is increased the response is more pronounced.

\begin{figure}

\begin{minipage}{0.50\linewidth}

\centering{

\includegraphics{images/png/detect_death.png}

}

\subcaption{\label{fig-detect_death}}

\end{minipage}%
%
\begin{minipage}{0.50\linewidth}

\centering{

\includegraphics{images/png/hist_data.png}

}

\subcaption{\label{fig-hist_data}}

\end{minipage}%
\newline
\begin{minipage}{0.50\linewidth}

\centering{

\includegraphics{images/png/sim_control_example_sims.png}

}

\subcaption{\label{fig-example_sims_control}}

\end{minipage}%
%
\begin{minipage}{0.50\linewidth}

\centering{

\includegraphics{images/png/simulate_control_KM.png}

}

\subcaption{\label{fig-simulated_control_KM}}

\end{minipage}%

\caption{\label{fig-simulated-controls}Simulated virtual control
patients. (a) Tumor detection and death are considered random events
which increase in probability as tumor size increases. (b) Histogram of
proliferation rates from (\citeproc{ref-yang2019}{Yang et al. 2019}),
red line shows fitted log normal distribution (c) Example model
simulation trajectories showing overall tumor density and fraction of
stem cells. The red stars indicate when the tumor was detected and the
blue dots when radiotherapy was started. (d) Comparison of survival
times for simulated control patients (resection and radiotherapy) with
BMP4 treatment, for two different sensitivities to BMP4.}

\end{figure}%

\subsubsection{Simulated phase 2 trial}\label{sec-phase-2-trial}

In practice a phase 2 trial will have only a small number of patients
(typically around 20-40) and the population will be heterogeneous with
respect to sensitivity to BMP4. We develop a virtual trial pipeline to
consider the chance of observing a successful trial for different
virtual populations. We have previously observed that proliferation rate
can impact the efficacy of BMP4 treatment
(Figure~\ref{fig-days_gained}). Therefore, we consider 3 stratifications
of the population based on proliferation rate. We construct these by
sampling twice as many virtual patients as we need for each trial (80)
and then splitting them into either the top 50\% fastest proliferators,
the middle 50\%, or the slowest 50\%, so that each group has 40 virtual
patients. We then split them via a stratified random split, so that both
groups have similar distributions of proliferation rates, into control
and treatment arms (20 in each), with parameters drawn from
Table~\ref{tbl-params} and \(\psi\) from a truncated normal
distribution. This assumes knowledge of growth rates for patients
enrolled in the trial, which is straightforward for our simulated
virtual patients. For each group (fast, medium, slow) we perform 20
virtual trials, each one comprising a distinct set of 40 patients,
giving us an idea of the chance of observing a successful trial.

We assume a distribution of sensitivities drawn from a truncated normal
distribution with mean sensitivity \(\psi_\text{mean} = 0.0085\), so
that, according to our model fits this would be a population that has
been identified as highly sensitive to BMP4. For this cohort of virtual
patients we consider a possible delivery schedule of BMP4-AMSCs which we
shall term `single-hit', this consists of a single dose of AMSCs-BMP4 at
the time of resection. This is a promising and practical option for
BMP4-AMSCs delivery as they can be implanted directly into the tumor at
the time of resection and means the patient is receiving some treatment
in the time between resection and radiotherapy, when typically they
receive none during this time. For this cohort of virtual patients, we
implement a series of different concentrations of AMSCs-BMP4; in
Figure~\ref{fig-Identical_mid_KM} we show average BMP4 concentration in
ng/ml, to compare this to the radiotherapy assay in
Section~\ref{sec-simulating-RT-experiments}. An average concentration of
8ng/ml corresponds to a peak of 100ng/ml at the time of resection, the
same concentration maintained for two days in the assay. We see that,
for peak concentrations of BMP4 similar to that in the \emph{in vitro}
assay, no trials show statistically significant differences in survival
curves (log-rank test, \(p>0.05\)). Indeed, according to these results a
considerably higher total dose of BMP4 is required in order to see
consistently successful trials (defined as rejecting the null hypothesis
of identical survival curves between control and treatment arms). This
may be because of the relatively rapid decay of AMSCs and BMP4; if ASMCs
are delivered at the time of resection very few remain and BMP4
concentrations are low by the time RT begins.

The results plotted in Figure~\ref{fig-Identical_mid_KM} are for the
group of fast proliferators only as this was the only group which saw a
significant response to BMP4 therapy.
Figure~\ref{fig-phase2_trial_summary} shows a summary of the results for
all the proliferation groups. This highlights the importance of
considering other patient specific parameters when selecting potential
patients for clinical trial.

\begin{figure}

\begin{minipage}{0.50\linewidth}

\centering{

\includegraphics{images/png/virtual_trial_BMP4_200_rho_case_5.png}

}

\subcaption{\label{fig-example2}}

\end{minipage}%
%
\begin{minipage}{0.50\linewidth}

\centering{

\includegraphics{images/png/virtual_trial_BMP4_500_rho_case_5.png}

}

\subcaption{\label{fig-example3}}

\end{minipage}%
\newline
\begin{minipage}{0.50\linewidth}

\centering{

\includegraphics{images/png/virtual_trial_BMP4_1000_rho_case_5.png}

}

\subcaption{\label{fig-ex-4}}

\end{minipage}%
%
\begin{minipage}{0.50\linewidth}

\centering{

\includegraphics{images/png/prob_succ_strat.png}

}

\subcaption{\label{fig-phase2_trial_summary}}

\end{minipage}%

\caption{\label{fig-Identical_mid_KM}Survival curves for 20 simulated
virtual trials for the fast proliferating stratification shaded by
success (if BMP4 treatment arm is significantly different to the control
arm) for four different concentrations of BMP4. (a) As expected, when
BMP4 is low the control and treatment arms are similar in all 20 trials.
(b) As BMP4 concentration increases, the fraction of successful trials
increases. (c) For sufficiently high BMP4 (and hence a sufficiently
strong treatment effect), almost all trials are successful. (d) Summary
of the successful trial from phase 2. Orange, green and blue represent
fast, medium and slow proliferation, respectively.}

\end{figure}%

\subsubsection{Different delivery schedules for
BMP4}\label{different-delivery-schedules-for-bmp4}

The high concentrations of BMP4 required in a single-hit approach
identified in Section~\ref{sec-phase-2-trial} motivates us to look at
alternative dosing strategies, where similar or greater benefit may be
observed for less average dose of BMP4. As the AMSCs only last for
around 2 weeks (and BMP4 slightly less) only a small concentration is
left by the time radiotherapy occurs in the single-hit approach. Since
BMP4 increases the radiosensitivity of the tumor and reduces the
enrichment of GSCs when radiotherapy occurs, delivering it in
combination with radiotherapy may provide significantly improved
responses even when a similar total dose of BMP4 is considered.
Therefore, we consider 3 alternative treatment schedules: i) a single
dose immediately before radiotherapy, ii) a continuous dose from
resection until the end of RT and iii) periodic doses in combination
with radiotherapy. These schedules are shown in
Figure~\ref{fig-delivery_schedule}. In all cases, the same total dose of
BMP4 (this corresponds to the same area under the curve in
Figure~\ref{fig-delivery_schedule}) will be administered so the
comparison is fair between delivery schedules.

We use the same virtual trial pipeline outlined previously in
Section~\ref{sec-virtual-trial-pipeline}, applied to these three
alternative dosing strategies. A summary of the results for the fast
proliferating group is plotted in
Figure~\ref{fig-delivery_schedule_frac_succ}. We see that the continuous
delivery schedule is the most effective, with the highest number of
successful trials with an average BMP4 concentration of around 10ng/ml
required for almost all trials to be successful. This is closely
followed by the periodic schedule. The shifted single dose appears to be
largely similar in efficacy as the single dose at resection. These show
that longer term exposure of BMP4 greatly increases its efficacy. This
highlights the importance of designing the delivery schedule of BMP4 to
maximise the effect of the treatment, and in future could be optimised
on a patient-specific level.

\begin{figure}

\begin{minipage}{0.50\linewidth}

\centering{

\includegraphics{images/png/BMP4_delivery_schedule.png}

}

\subcaption{\label{fig-delivery_schedule}}

\end{minipage}%
%
\begin{minipage}{0.50\linewidth}

\centering{

\includegraphics{images/png/diff_schedules_frac_succ.png}

}

\subcaption{\label{fig-delivery_schedule_frac_succ}}

\end{minipage}%

\caption{\label{fig-delivery_schedule_frac_succ_overall}Alternative
delivery schedules have improved response to BPM4 for same total dose.
(a) Different delivery schedules for BMP4. Here time 0 represents the
time of detection and radiotherapy occurs 30 days after detection. (b)
Summary of fraction of successful trails for different delivery
schedules.}

\end{figure}%

\section{Discussion}\label{sec-discussion}

In this study, we have developed a comprehensive mathematical model to
simulate GSC-driven tumor growth, specifically focusing on the impact of
BMP4 treatment. By parameterizing our model using experimental data from
five distinct glioma stem cell lines exposed to BMP4, we were able to
qualitatively estimate the sensitivity of these cell lines to BMP4. This
approach provides valuable insights into the potential variability in
treatment response among different glioma cell populations.

A key limitation of our model lies in the inherent assumptions necessary
for reducing the complex biology of glioblastoma/glioma to a system of
ordinary differential equations (ODEs). While our model successfully
encapsulates many aspects of tumor dynamics, it does not explicitly
account for the spatial heterogeneity of glioblastoma. Future work could
expand this model to incorporate spatial considerations, to better
capture the intricate microenvironment and its influence on tumor
behaviour. That said, we hope that capturing the essence of the proposed
BMP4 treatment in the current model has highlighted key mechanisms by
which impact on tumor growth may or may not be seen in the clinic.

Moreover, our study primarily concentrated on the impact of BMP4-induced
differentiation on radiosensitivity, leaving the direct effects of BMP4
on proliferation rates as an area for further exploration. Understanding
how BMP4 modulates proliferation, in addition to differentiation, could
provide a more comprehensive picture of its therapeutic potential and
guide the development of more effective treatment strategies.

Importantly, we have also demonstrated a virtual clinical trial pipeline
to evaluate the potential of BMP4-AMSCs treatment for patients with GBM.
Our simulations revealed that a significant amount of BMP4 would be
required to achieve successful outcomes in a substantial proportion of
patients. This finding underscores the importance of optimizing BMP4
dosage and delivery methods for future clinical applications.
Additionally, our results suggest that patient stratification based on
proliferation rates could enhance the likelihood of treatment success.
By selecting patients with higher proliferation rates, we could
potentially increase the observed efficacy of BMP4 in combination with
radiotherapy.

Furthermore, we explored various BMP4 delivery schedules and identified
that alternative strategies could enhance the therapeutic synergy
between BMP4 and radiotherapy. These findings indicate that the timing
and duration of BMP4 administration, in particular in relation to the
timing of radiotherapy, can be crucial factors that could be optimized
to improve clinical outcomes. We have shown that prolonged exposure to
BMP4 greatly increased it efficacy when compared to a `single-hit'
approach. Future studies could expand on this by investigating different
delivery modalities and schedules, potentially in combination with other
therapies, to maximize the therapeutic benefit of BMP4.

In conclusion, our work provides a robust framework for virtual clinical
trials, offering valuable predictions that can guide the clinical
translation of BMP4-based therapies for GBM. By highlighting the need
for high BMP4 levels, patient stratification, and optimized delivery
strategies, we have laid the groundwork for future studies that can
further refine and validate these approaches in a clinical context.

\bookmarksetup{startatroot}

\chapter*{References}\label{references}
\addcontentsline{toc}{chapter}{References}

\markboth{References}{References}

\phantomsection\label{refs}
\begin{CSLReferences}{1}{0}
\bibitem[\citeproctext]{ref-acar2020}
Acar, Ahmet, Daniel Nichol, Javier Fernandez-Mateos, George D.
Cresswell, Iros Barozzi, Sung Pil Hong, Nicholas Trahearn, et al. 2020.
{``Exploiting Evolutionary Steering to Induce Collateral Drug
Sensitivity in Cancer.''} \emph{Nature Communications} 11 (1): 1923.
\url{https://doi.org/10.1038/s41467-020-15596-z}.

\bibitem[\citeproctext]{ref-al-hajj2003}
Al-Hajj, Muhammad, Max S. Wicha, Adalberto Benito-Hernandez, Sean J.
Morrison, and Michael F. Clarke. 2003. {``Prospective Identification of
Tumorigenic Breast Cancer Cells.''} \emph{Proceedings of the National
Academy of Sciences} 100 (7): 3983--88.
\url{https://doi.org/10.1073/pnas.0530291100}.

\bibitem[\citeproctext]{ref-bachman2013}
Bachman, Jeff W. N., and Thomas Hillen. 2013. {``Mathematical
Optimization of the Combination of Radiation and Differentiation
Therapies for Cancer.''} \emph{Frontiers in Oncology} 3.
\url{https://doi.org/10.3389/fonc.2013.00052}.

\bibitem[\citeproctext]{ref-bagley2022}
Bagley, Stephen J., Shawn Kothari, Rifaquat Rahman, Eudocia Q. Lee,
Gavin P. Dunn, Evanthia Galanis, Susan M. Chang, et al. 2022.
{``Glioblastoma Clinical Trials: Current Landscape and Opportunities for
Improvement.''} \emph{Clinical Cancer Research} 28 (4): 594--602.
\url{https://doi.org/10.1158/1078-0432.CCR-21-2750}.

\bibitem[\citeproctext]{ref-bailey2007}
Bailey, Jennifer M., Pankaj K. Singh, and Michael A. Hollingsworth.
2007. {``Cancer Metastasis Facilitated by Developmental Pathways: Sonic
Hedgehog, Notch, and Bone Morphogenic Proteins.''} \emph{Journal of
Cellular Biochemistry} 102 (4): 829--39.
\url{https://doi.org/10.1002/jcb.21509}.

\bibitem[\citeproctext]{ref-balk2011}
Balk, Sophie J. 2011. {``Ultraviolet Radiation: A Hazard to Children and
Adolescents.''} \emph{Pediatrics} 127 (3): e791--817.
\url{https://doi.org/10.1542/peds.2010-3502}.

\bibitem[\citeproctext]{ref-bao2006}
Bao, Shideng, Qiulian Wu, Roger E. McLendon, Yueling Hao, Qing Shi,
Anita B. Hjelmeland, Mark W. Dewhirst, Darell D. Bigner, and Jeremy N.
Rich. 2006. {``Glioma Stem Cells Promote Radioresistance by Preferential
Activation of the DNA Damage Response.''} \emph{Nature} 444 (7120):
756--60. \url{https://doi.org/10.1038/nature05236}.

\bibitem[\citeproctext]{ref-bartoszynski2001}
Bartoszyński, Robert, Lutz Edler, Leonid Hanin, Annette Kopp-Schneider,
Lyudmila Pavlova, Alexander Tsodikov, Alexander Zorin, and Andrej Yu.
Yakovlev. 2001. {``Modeling Cancer Detection: Tumor Size as a Source of
Information on Unobservable Stages of Carcinogenesis.''}
\emph{Mathematical Biosciences} 171 (2): 113--42.
\url{https://doi.org/10.1016/s0025-5564(01)00058-x}.

\bibitem[\citeproctext]{ref-baskar2012}
Baskar, Rajamanickam, Kuo Ann Lee, Richard Yeo, and Kheng-Wei Yeoh.
2012. {``Cancer and Radiation Therapy: Current Advances and Future
Directions.''} \emph{International Journal of Medical Sciences} 9 (3):
193--99. \url{https://doi.org/10.7150/ijms.3635}.

\bibitem[\citeproctext]{ref-bonnet1997}
Bonnet, Dominique, and John E. Dick. 1997. {``Human Acute Myeloid
Leukemia Is Organized as a Hierarchy That Originates from a Primitive
Hematopoietic Cell.''} \emph{Nature Medicine} 3 (7): 730--37.
\url{https://doi.org/10.1038/nm0797-730}.

\bibitem[\citeproctext]{ref-breward2003}
Breward, C. 2003. {``A Multiphase Model Describing Vascular Tumour
Growth.''} \emph{Bulletin of Mathematical Biology} 65 (4): 609--40.
\url{https://doi.org/10.1016/S0092-8240(03)00027-2}.

\bibitem[\citeproctext]{ref-cardinal2022}
Cardinal, Olivia, Chloé Burlot, Yangxin Fu, Powel Crosley, Mary Hitt,
Morgan Craig, and Adrianne L. Jenner. 2022. {``Establishing Combination
PAC{-}1 and TRAIL Regimens for Treating Ovarian Cancer Based on
Patient{-}Specific Pharmacokinetic Profiles Using {\emph{in Silico}}
Clinical Trials.''} \emph{Computational and Systems Oncology} 2 (2).
\url{https://doi.org/10.1002/cso2.1035}.

\bibitem[\citeproctext]{ref-caruxe9n2016}
Carén, Helena, Stephan Beck, and Steven M. Pollard. 2016.
{``Differentiation Therapy for Glioblastoma {\textendash} Too Many
Obstacles?''} \emph{Molecular \& Cellular Oncology} 3 (2): e1124174.
\url{https://doi.org/10.1080/23723556.2015.1124174}.

\bibitem[\citeproctext]{ref-chaichana2014}
Chaichana, Kaisorn L., Eibar Ernesto Cabrera-Aldana, Ignacio
Jusue-Torres, Olindi Wijesekera, Alessandro Olivi, Maryam Rahman, and
Alfredo Quinones-Hinojosa. 2014. {``When Gross Total Resection of a
Glioblastoma Is Possible, How Much Resection Should Be Achieved?''}
\emph{World Neurosurgery} 82 (1-2): e257--65.
\url{https://doi.org/10.1016/j.wneu.2014.01.019}.

\bibitem[\citeproctext]{ref-cloughesy2020}
Cloughesy, Timothy F., Kevin Petrecca, Tobias Walbert, Nicholas
Butowski, Michael Salacz, James Perry, Denise Damek, et al. 2020.
{``Effect of Vocimagene Amiretrorepvec in Combination With Flucytosine
Vs Standard of Care on Survival Following Tumor Resection in Patients
With Recurrent High-Grade Glioma: A Randomized Clinical Trial.''}
\emph{JAMA Oncology} 6 (12): 1939.
\url{https://doi.org/10.1001/jamaoncol.2020.3161}.

\bibitem[\citeproctext]{ref-craig2023}
Craig, Morgan, Jana L. Gevertz, Irina Kareva, and Kathleen P. Wilkie.
2023. {``A Practical Guide for the Generation of Model-Based Virtual
Clinical Trials.''} \emph{Frontiers in Systems Biology} 3 (June).
\url{https://doi.org/10.3389/fsysb.2023.1174647}.

\bibitem[\citeproctext]{ref-dethuxe92018}
De Thé, Hugues. 2018. {``Differentiation Therapy Revisited.''}
\emph{Nature Reviews Cancer} 18 (2): 117--27.
\url{https://doi.org/10.1038/nrc.2017.103}.

\bibitem[\citeproctext]{ref-dingli2006}
Dingli, David, and Franziska Michor. 2006. {``Successful Therapy Must
Eradicate Cancer Stem Cells.''} \emph{STEM CELLS} 24 (12): 2603--10.
\url{https://doi.org/10.1634/stemcells.2006-0136}.

\bibitem[\citeproctext]{ref-dirks2006}
Dirks, Peter B. 2006. {``Stem Cells and Brain Tumours.''} \emph{Nature}
444 (7120): 687--88. \url{https://doi.org/10.1038/444687a}.

\bibitem[\citeproctext]{ref-doucette2011}
Doucette, Tiffany, Ganesh Rao, Yuhui Yang, Joy Gumin, Naoki Shinojima,
B. Nebiyou Bekele, Wei Qiao, Wei Zhang, and Frederick F. Lang. 2011.
{``Mesenchymal Stem Cells Display Tumor-Specific Tropism in an
RCAS/Ntv-a Glioma Model.''} \emph{Neoplasia} 13 (8): 716--25.
\url{https://doi.org/10.1593/neo.101680}.

\bibitem[\citeproctext]{ref-dowden2019}
Dowden, Helen, and Jamie Munro. 2019. {``Trends in Clinical Success
Rates and Therapeutic Focus.''} \emph{Nature Reviews Drug Discovery} 18
(7): 495--96. \url{https://doi.org/10.1038/d41573-019-00074-z}.

\bibitem[\citeproctext]{ref-enderling2009}
Enderling, Heiko, Alexander R. A. Anderson, Mark A. J. Chaplain, Afshin
Beheshti, Lynn Hlatky, and Philip Hahnfeldt. 2009. {``Paradoxical
Dependencies of Tumor Dormancy and Progression on Basic Cell
Kinetics.''} \emph{Cancer Research} 69 (22): 8814--21.
\url{https://doi.org/10.1158/0008-5472.CAN-09-2115}.

\bibitem[\citeproctext]{ref-enriquez-navas2015}
Enriquez-Navas, Pedro M., Jonathan W. Wojtkowiak, and Robert A. Gatenby.
2015. {``Application of Evolutionary Principles to Cancer Therapy.''}
\emph{Cancer Research} 75 (22): 4675--80.
\url{https://doi.org/10.1158/0008-5472.CAN-15-1337}.

\bibitem[\citeproctext]{ref-farias}
Farias, Virgina, Vinitha Rani, Natanael Zacro, Mieu Brooks, Naveen
Nagaiah, Henry Ruiz-Garcia, Rachel Sarabia-Estrada, et al. n.d.b.
{``BMP4-Secreting MSCs Derived from Human Adipose Tissue Radiosensitize
Glioblastoma Stem Cells via Downregulation of OLIG2 Signaling
Pathway.''} \emph{In Preparation}.

\bibitem[\citeproctext]{ref-fariasa}
---------, et al. n.d.a. {``BMP4-Secreting MSCs Derived from Human
Adipose Tissue Radiosensitize Glioblastoma Stem Cells via Downregulation
of OLIG2 Signaling Pathway.''} \emph{In Preparation}.

\bibitem[\citeproctext]{ref-folkman2004}
Folkman, Judah, and Raghu Kalluri. 2004. {``Cancer Without Disease.''}
\emph{Nature} 427 (6977): 787--87.
\url{https://doi.org/10.1038/427787a}.

\bibitem[\citeproctext]{ref-galli2004}
Galli, Rossella, Elena Binda, Ugo Orfanelli, Barbara Cipelletti, Angela
Gritti, Simona De Vitis, Roberta Fiocco, Chiara Foroni, Francesco
Dimeco, and Angelo Vescovi. 2004. {``Isolation and Characterization of
Tumorigenic, Stem-Like Neural Precursors from Human Glioblastoma.''}
\emph{Cancer Research} 64 (19): 7011--21.
\url{https://doi.org/10.1158/0008-5472.CAN-04-1364}.

\bibitem[\citeproctext]{ref-gao2013}
Gao, Xuefeng, J. Tyson McDonald, Lynn Hlatky, and Heiko Enderling. 2013.
{``Acute and Fractionated Irradiation Differentially Modulate Glioma
Stem Cell Division Kinetics.''} \emph{Cancer Research} 73 (5): 1481--90.
\url{https://doi.org/10.1158/0008-5472.CAN-12-3429}.

\bibitem[\citeproctext]{ref-griffin2006}
Griffin, Robert J. 2006. {``Radiobiology for the Radiologist, 6th
Edition.''} \emph{International Journal of Radiation
Oncology*Biology*Physics} 66 (2): 627.
\url{https://doi.org/10.1016/j.ijrobp.2006.06.027}.

\bibitem[\citeproctext]{ref-guerrero-cuxe1zares2014}
Guerrero-Cázares, Hugo, Stephany Y. Tzeng, Noah P. Young, Ameer O.
Abutaleb, Alfredo Quiñones-Hinojosa, and Jordan J. Green. 2014.
{``Biodegradable Polymeric Nanoparticles Show High Efficacy and
Specificity at DNA Delivery to Human Glioblastoma {\emph{in Vitro}} and
{\emph{in Vivo}}.''} \emph{ACS Nano} 8 (5): 5141--53.
\url{https://doi.org/10.1021/nn501197v}.

\bibitem[\citeproctext]{ref-hanahan2000}
Hanahan, Douglas, and Robert A Weinberg. 2000. {``The Hallmarks of
Cancer.''} \emph{Cell} 100 (1): 57--70.
\url{https://doi.org/10.1016/S0092-8674(00)81683-9}.

\bibitem[\citeproctext]{ref-hemmati2003}
Hemmati, Houman D., Ichiro Nakano, Jorge A. Lazareff, Michael
Masterman-Smith, Daniel H. Geschwind, Marianne Bronner-Fraser, and
Harley I. Kornblum. 2003. {``Cancerous Stem Cells Can Arise from
Pediatric Brain Tumors.''} \emph{Proceedings of the National Academy of
Sciences} 100 (25): 15178--83.
\url{https://doi.org/10.1073/pnas.2036535100}.

\bibitem[\citeproctext]{ref-hillen2013}
Hillen, Thomas, Heiko Enderling, and Philip Hahnfeldt. 2013. {``The
Tumor Growth Paradox and Immune System-Mediated Selection for Cancer
Stem Cells.''} \emph{Bulletin of Mathematical Biology} 75 (1): 161--84.
\url{https://doi.org/10.1007/s11538-012-9798-x}.

\bibitem[\citeproctext]{ref-hitomi2021}
Hitomi, Masahiro, Anastasia P. Chumakova, Daniel J. Silver, Arnon M.
Knudsen, W. Dean Pontius, Stephanie Murphy, Neha Anand, Bjarne W.
Kristensen, and Justin D. Lathia. 2021a. {``Asymmetric Cell Division
Promotes Therapeutic Resistance in Glioblastoma Stem Cells.''} \emph{JCI
Insight} 6 (3): e130510.
\url{https://doi.org/10.1172/jci.insight.130510}.

\bibitem[\citeproctext]{ref-hitomi2021a}
---------. 2021b. {``Asymmetric Cell Division Promotes Therapeutic
Resistance in Glioblastoma Stem Cells.''} \emph{JCI Insight} 6 (3):
e130510. \url{https://doi.org/10.1172/jci.insight.130510}.

\bibitem[\citeproctext]{ref-hubbard2013}
Hubbard, M. E., and H. M. Byrne. 2013. {``Multiphase Modelling of
Vascular Tumour Growth in Two Spatial Dimensions.''} \emph{Journal of
Theoretical Biology} 316 (January): 70--89.
\url{https://doi.org/10.1016/j.jtbi.2012.09.031}.

\bibitem[\citeproctext]{ref-ignatova2002}
Ignatova, Tatyana N., Valery G. Kukekov, Eric D. Laywell, Oleg N.
Suslov, Frank D. Vrionis, and Dennis A. Steindler. 2002. {``Human
Cortical Glial Tumors Contain Neural Stem{-}Like Cells Expressing
Astroglial and Neuronal Markers in Vitro.''} \emph{Glia} 39 (3):
193--206. \url{https://doi.org/10.1002/glia.10094}.

\bibitem[\citeproctext]{ref-kim2016}
Kim, Eunjung, Vito W. Rebecca, Keiran S. M. Smalley, and Alexander R. A.
Anderson. 2016. {``Phase i Trials in Melanoma: A Framework to Translate
Preclinical Findings to the Clinic.''} \emph{European Journal of Cancer}
67 (November): 213--22.
\url{https://doi.org/10.1016/j.ejca.2016.07.024}.

\bibitem[\citeproctext]{ref-kim2020}
Kim, Jayoung, Sujan K. Mondal, Stephany Y. Tzeng, Yuan Rui, Rawan
Al-kharboosh, Kristen K. Kozielski, Adip G. Bhargav, Cesar A. Garcia,
Alfredo Quiñones-Hinojosa, and Jordan J. Green. 2020. {``Poly(ethylene
Glycol){\textendash}Poly(beta-Amino Ester)-Based Nanoparticles for
Suicide Gene Therapy Enhance Brain Penetration and Extend Survival in a
Preclinical Human Glioblastoma Orthotopic Xenograft Model.''} \emph{ACS
Biomaterials Science \& Engineering} 6 (5): 2943--55.
\url{https://doi.org/10.1021/acsbiomaterials.0c00116}.

\bibitem[\citeproctext]{ref-lander2009}
Lander, Arthur D, Kimberly K Gokoffski, Frederic Y. M Wan, Qing Nie, and
Anne L Calof. 2009. {``Cell Lineages and the Logic of Proliferative
Control.''} Edited by Charles F Stevens. \emph{PLoS Biology} 7 (1):
e1000015. \url{https://doi.org/10.1371/journal.pbio.1000015}.

\bibitem[\citeproctext]{ref-lapidot1994}
Lapidot, Tsvee, Christian Sirard, Josef Vormoor, Barbara Murdoch, Trang
Hoang, Julio Caceres-Cortes, Mark Minden, Bruce Paterson, Michael A.
Caligiuri, and John E. Dick. 1994. {``A Cell Initiating Human Acute
Myeloid Leukaemia After Transplantation into SCID Mice.''} \emph{Nature}
367 (6464): 645--48. \url{https://doi.org/10.1038/367645a0}.

\bibitem[\citeproctext]{ref-lathia2011}
Lathia, J D, M Hitomi, J Gallagher, S P Gadani, J Adkins, A Vasanji, L
Liu, et al. 2011. {``Distribution of CD133 Reveals Glioma Stem Cells
Self-Renew Through Symmetric and Asymmetric Cell Divisions.''}
\emph{Cell Death \& Disease} 2 (9): e200--200.
\url{https://doi.org/10.1038/cddis.2011.80}.

\bibitem[\citeproctext]{ref-lee2016}
Lee, Yeri, Jin-Ku Lee, Sun Hee Ahn, Jeongwu Lee, and Do-Hyun Nam. 2016.
{``WNT Signaling in Glioblastoma and Therapeutic Opportunities.''}
\emph{Laboratory Investigation} 96 (2): 137--50.
\url{https://doi.org/10.1038/labinvest.2015.140}.

\bibitem[\citeproctext]{ref-li2014}
Li, Qian, Olindi Wijesekera, Sussan J. Salas, Joanna Y. Wang, Mingxin
Zhu, Colette Aprhys, Kaisorn L. Chaichana, et al. 2014b. {``Mesenchymal
Stem Cells from Human Fat Engineered to Secrete BMP4 Are Nononcogenic,
Suppress Brain Cancer, and Prolong Survival.''} \emph{Clinical Cancer
Research} 20 (9): 2375--87.
\url{https://doi.org/10.1158/1078-0432.CCR-13-1415}.

\bibitem[\citeproctext]{ref-li2014a}
---------, et al. 2014a. {``Mesenchymal Stem Cells from Human Fat
Engineered to Secrete BMP4 Are Nononcogenic, Suppress Brain Cancer, and
Prolong Survival.''} \emph{Clinical Cancer Research} 20 (9): 2375--87.
\url{https://doi.org/10.1158/1078-0432.CCR-13-1415}.

\bibitem[\citeproctext]{ref-lowengrub2010}
Lowengrub, J S, H B Frieboes, F Jin, Y-L Chuang, X Li, P Macklin, S M
Wise, and V Cristini. 2010. {``Nonlinear Modelling of Cancer: Bridging
the Gap Between Cells and Tumours.''} \emph{Nonlinearity} 23 (1):
R1--91. \url{https://doi.org/10.1088/0951-7715/23/1/R01}.

\bibitem[\citeproctext]{ref-ma2018}
Ma, Qianquan, Wenyong Long, Changsheng Xing, Junjun Chu, Mei Luo, Helen
Y. Wang, Qing Liu, and Rong-Fu Wang. 2018. {``Cancer Stem Cells and
Immunosuppressive Microenvironment in Glioma.''} \emph{Frontiers in
Immunology} 9 (December): 2924.
\url{https://doi.org/10.3389/fimmu.2018.02924}.

\bibitem[\citeproctext]{ref-majumdar2020}
Majumdar, Sreemita, and Song-Tao Liu. 2020. {``Cell Division Symmetry
Control and Cancer Stem Cells.''} \emph{AIMS Molecular Science} 7 (2):
82--101. \url{https://doi.org/10.3934/molsci.2020006}.

\bibitem[\citeproctext]{ref-mangraviti2016}
Mangraviti, Antonella, Stephany Y. Tzeng, David Gullotti, Kristen L.
Kozielski, Jennifer E. Kim, Michael Seng, Sara Abbadi, et al. 2016b.
{``Non-Virally Engineered Human Adipose Mesenchymal Stem Cells Produce
BMP4, Target Brain Tumors, and Extend Survival.''} \emph{Biomaterials}
100 (September): 53--66.
\url{https://doi.org/10.1016/j.biomaterials.2016.05.025}.

\bibitem[\citeproctext]{ref-mangraviti2016a}
---------, et al. 2016a. {``Non-Virally Engineered Human Adipose
Mesenchymal Stem Cells Produce BMP4, Target Brain Tumors, and Extend
Survival.''} \emph{Biomaterials} 100 (September): 53--66.
\url{https://doi.org/10.1016/j.biomaterials.2016.05.025}.

\bibitem[\citeproctext]{ref-mcmahon2018}
McMahon, Stephen Joseph. 2018. {``The Linear Quadratic Model: Usage,
Interpretation and Challenges.''} \emph{Physics in Medicine \& Biology}
64 (1): 01TR01. \url{https://doi.org/10.1088/1361-6560/aaf26a}.

\bibitem[\citeproctext]{ref-meza2008}
Meza, Rafael, Jihyoun Jeon, Suresh H. Moolgavkar, and E. Georg Luebeck.
2008. {``Age-Specific Incidence of Cancer: Phases, Transitions, and
Biological Implications.''} \emph{Proceedings of the National Academy of
Sciences} 105 (42): 16284--89.
\url{https://doi.org/10.1073/pnas.0801151105}.

\bibitem[\citeproctext]{ref-narita2019}
Narita, Yoshitaka, Yoshiki Arakawa, Fumiyuki Yamasaki, Ryo Nishikawa,
Tomokazu Aoki, Masayuki Kanamori, Motoo Nagane, et al. 2019. {``A
Randomized, Double-Blind, Phase III Trial of Personalized Peptide
Vaccination for Recurrent Glioblastoma.''} \emph{Neuro-Oncology} 21 (3):
348--59. \url{https://doi.org/10.1093/neuonc/noy200}.

\bibitem[\citeproctext]{ref-nayak2020}
Nayak, Sonali, Ashorne Mahenthiran, Yongyong Yang, Mark McClendon,
Barbara Mania-Farnell, Charles David James, John A. Kessler, et al.
2020. {``Bone Morphogenetic Protein 4 Targeting Glioma Stem-Like Cells
for Malignant Glioma Treatment: Latest Advances and Implications for
Clinical Application.''} \emph{Cancers} 12 (2): 516.
\url{https://doi.org/10.3390/cancers12020516}.

\bibitem[\citeproctext]{ref-neves-e-castro2006}
Neves-E-Castro, Manuel. 2006. {``Why Do Some Breast Cancer Cells Remain
Dormant?''} \emph{Gynecological Endocrinology} 22 (4): 190--97.
\url{https://doi.org/10.1080/09513590600624374}.

\bibitem[\citeproctext]{ref-orourke2009}
O'Rourke, S. F. C., H. McAneney, and T. Hillen. 2009. {``Linear
Quadratic and Tumour Control Probability Modelling in External Beam
Radiotherapy.''} \emph{Journal of Mathematical Biology} 58 (4-5):
799--817. \url{https://doi.org/10.1007/s00285-008-0222-y}.

\bibitem[\citeproctext]{ref-ostrom2019}
Ostrom, Quinn T, Gino Cioffi, Haley Gittleman, Nirav Patil, Kristin
Waite, Carol Kruchko, and Jill S Barnholtz-Sloan. 2019. {``CBTRUS
Statistical Report: Primary Brain and Other Central Nervous System
Tumors Diagnosed in the United States in 2012{\textendash}2016.''}
\emph{Neuro-Oncology} 21 (Supplement{\_}5): v1--100.
\url{https://doi.org/10.1093/neuonc/noz150}.

\bibitem[\citeproctext]{ref-pannuti2010}
Pannuti, Antonio, Kimberly Foreman, Paola Rizzo, Clodia Osipo, Todd
Golde, Barbara Osborne, and Lucio Miele. 2010. {``Targeting Notch to
Target Cancer Stem Cells.''} \emph{Clinical Cancer Research} 16 (12):
3141--52. \url{https://doi.org/10.1158/1078-0432.CCR-09-2823}.

\bibitem[\citeproctext]{ref-pendleton2013}
Pendleton, Courtney, Qian Li, David A. Chesler, Kristy Yuan, Hugo
Guerrero-Cazares, and Alfredo Quinones-Hinojosa. 2013b. {``Mesenchymal
Stem Cells Derived from Adipose Tissue Vs Bone Marrow: In Vitro
Comparison of Their Tropism Towards Gliomas.''} Edited by Joseph
Najbauer. \emph{PLoS ONE} 8 (3): e58198.
\url{https://doi.org/10.1371/journal.pone.0058198}.

\bibitem[\citeproctext]{ref-pendleton2013a}
---------. 2013a. {``Mesenchymal Stem Cells Derived from Adipose Tissue
Vs Bone Marrow: In Vitro Comparison of Their Tropism Towards Gliomas.''}
Edited by Joseph Najbauer. \emph{PLoS ONE} 8 (3): e58198.
\url{https://doi.org/10.1371/journal.pone.0058198}.

\bibitem[\citeproctext]{ref-piccirillo2006}
Piccirillo, S. G. M., B. A. Reynolds, N. Zanetti, G. Lamorte, E. Binda,
G. Broggi, H. Brem, A. Olivi, F. Dimeco, and A. L. Vescovi. 2006.
{``Bone Morphogenetic Proteins Inhibit the Tumorigenic Potential of
Human Brain Tumour-Initiating Cells.''} \emph{Nature} 444 (7120):
761--65. \url{https://doi.org/10.1038/nature05349}.

\bibitem[\citeproctext]{ref-plevritis2006}
Plevritis, Sylvia K., Peter Salzman, Bronislava M. Sigal, and Peter W.
Glynn. 2006. {``A Natural History Model of Stage Progression Applied to
Breast Cancer.''} \emph{Statistics in Medicine} 26 (3): 581--95.
\url{https://doi.org/10.1002/sim.2550}.

\bibitem[\citeproctext]{ref-potten1981}
Potten, C. S. 1981. {``The Cell Kinetic Mechanism for Radiation-Induced
Cellular Depletion of Epithelial Tissue Based on Hierarchical
Differences in Radiosensitivity.''} \emph{International Journal of
Radiation Biology and Related Studies in Physics, Chemistry and
Medicine} 40 (2): 217--25.
\url{https://doi.org/10.1080/09553008114551101}.

\bibitem[\citeproctext]{ref-reardon2020}
Reardon, David A., Alba A. Brandes, Antonio Omuro, Paul Mulholland,
Michael Lim, Antje Wick, Joachim Baehring, et al. 2020. {``Effect of
Nivolumab Vs Bevacizumab in Patients With Recurrent Glioblastoma: The
CheckMate 143 Phase 3 Randomized Clinical Trial.''} \emph{JAMA Oncology}
6 (7): 1003. \url{https://doi.org/10.1001/jamaoncol.2020.1024}.

\bibitem[\citeproctext]{ref-reya2001}
Reya, Tannishtha, Sean J. Morrison, Michael F. Clarke, and Irving L.
Weissman. 2001. {``Stem Cells, Cancer, and Cancer Stem Cells.''}
\emph{Nature} 414 (6859): 105--11.
\url{https://doi.org/10.1038/35102167}.

\bibitem[\citeproctext]{ref-ricci-vitiani2007}
Ricci-Vitiani, Lucia, Dario G. Lombardi, Emanuela Pilozzi, Mauro
Biffoni, Matilde Todaro, Cesare Peschle, and Ruggero De Maria. 2007.
{``Identification and Expansion of Human Colon-Cancer-Initiating
Cells.''} \emph{Nature} 445 (7123): 111--15.
\url{https://doi.org/10.1038/nature05384}.

\bibitem[\citeproctext]{ref-rich2007}
Rich, Jeremy N. 2007. {``Cancer Stem Cells in Radiation Resistance.''}
\emph{Cancer Research} 67 (19): 8980--84.
\url{https://doi.org/10.1158/0008-5472.CAN-07-0895}.

\bibitem[\citeproctext]{ref-rockne2009}
Rockne, R., E. C. Alvord, J. K. Rockhill, and K. R. Swanson. 2009. {``A
Mathematical Model for Brain Tumor Response to Radiation Therapy.''}
\emph{Journal of Mathematical Biology} 58 (4-5): 561.
\url{https://doi.org/10.1007/s00285-008-0219-6}.

\bibitem[\citeproctext]{ref-rockne2010}
Rockne, R, J K Rockhill, M Mrugala, A M Spence, I Kalet, K Hendrickson,
A Lai, T Cloughesy, E C Alvord, and K R Swanson. 2010. {``Predicting the
Efficacy of Radiotherapy in Individual Glioblastoma Patients {\emph{in
Vivo:}} A Mathematical Modeling Approach.''} \emph{Physics in Medicine
and Biology} 55 (12): 3271--85.
\url{https://doi.org/10.1088/0031-9155/55/12/001}.

\bibitem[\citeproctext]{ref-roth2021}
Roth, Patrick, Thierry Gorlia, Jaap C. Reijneveld, Filip Yves Francine
Leon De Vos, Ahmed Idbaih, Jean-Sebastien Frenel, Emilie Le Rhun, et al.
2021. {``EORTC 1709/CCTG CE.8: A Phase III Trial of Marizomib in
Combination with Temozolomide-Based Radiochemotherapy Versus
Temozolomide-Based Radiochemotherapy Alone in Patients with Newly
Diagnosed Glioblastoma.''} \emph{Journal of Clinical Oncology} 39
(15{\_}suppl): 2004--4.
\url{https://doi.org/10.1200/JCO.2021.39.15_suppl.2004}.

\bibitem[\citeproctext]{ref-schonberg2014}
Schonberg, David L., Daniel Lubelski, Tyler E. Miller, and Jeremy N.
Rich. 2014. {``Brain Tumor Stem Cells: Molecular Characteristics and
Their Impact on Therapy.''} \emph{Molecular Aspects of Medicine} 39
(October): 82--101. \url{https://doi.org/10.1016/j.mam.2013.06.004}.

\bibitem[\citeproctext]{ref-sell1993}
Sell, S. 1993. {``Cellular Origin of Cancer: Dedifferentiation or Stem
Cell Maturation Arrest?''} \emph{Environmental Health Perspectives} 101
(suppl 5): 15--26. \url{https://doi.org/10.1289/ehp.93101s515}.

\bibitem[\citeproctext]{ref-singh2003}
Singh, Sheila K., Ian D. Clarke, Mizuhiko Terasaki, Victoria E. Bonn,
Cynthia Hawkins, Jeremy Squire, and Peter B. Dirks. 2003.
{``Identification of a Cancer Stem Cell in Human Brain Tumors.''}
\emph{Cancer Research} 63 (18): 5821--28.

\bibitem[\citeproctext]{ref-singh2004}
Singh, Sheila K., Cynthia Hawkins, Ian D. Clarke, Jeremy A. Squire, Jane
Bayani, Takuichiro Hide, R. Mark Henkelman, Michael D. Cusimano, and
Peter B. Dirks. 2004. {``Identification of Human Brain Tumour Initiating
Cells.''} \emph{Nature} 432 (7015): 396--401.
\url{https://doi.org/10.1038/nature03128}.

\bibitem[\citeproctext]{ref-smith2015}
Smith, Chris L., Kaisorn L. Chaichana, Young M. Lee, Benjamin Lin, Kevin
M. Stanko, Thomas O'Donnell, Saksham Gupta, et al. 2015. {``Pre-Exposure
of Human Adipose Mesenchymal Stem Cells to Soluble Factors Enhances
Their Homing to Brain Cancer.''} \emph{Stem Cells Translational
Medicine} 4 (3): 239--51. \url{https://doi.org/10.5966/sctm.2014-0149}.

\bibitem[\citeproctext]{ref-stiles2008}
Stiles, Charles D., and David H. Rowitch. 2008. {``Glioma Stem Cells: A
Midterm Exam.''} \emph{Neuron} 58 (6): 832--46.
\url{https://doi.org/10.1016/j.neuron.2008.05.031}.

\bibitem[\citeproctext]{ref-stupp2005}
Stupp, Roger, Warren P. Mason, Martin J. Van Den Bent, Michael Weller,
Barbara Fisher, Martin J. B. Taphoorn, Karl Belanger, et al. 2005.
{``Radiotherapy Plus Concomitant and Adjuvant Temozolomide for
Glioblastoma.''} \emph{New England Journal of Medicine} 352 (10):
987--96. \url{https://doi.org/10.1056/NEJMoa043330}.

\bibitem[\citeproctext]{ref-sweeney1995}
Sweeney, Eamon. 1995. {``Dormant Cells in Columnar Cell Carcinoma of the
Thyroid.''} \emph{Human Pathology} 26 (6): 691--92.
\url{https://doi.org/10.1016/0046-8177(95)90180-9}.

\bibitem[\citeproctext]{ref-taipale2001}
Taipale, Jussi, and Philip A. Beachy. 2001. {``The Hedgehog and Wnt
Signalling Pathways in Cancer.''} \emph{Nature} 411 (6835): 349--54.
\url{https://doi.org/10.1038/35077219}.

\bibitem[\citeproctext]{ref-tang2021}
Tang, Xuejia, Chenghai Zuo, Pengchao Fang, Guojing Liu, Yongyi Qiu, Yi
Huang, and Rongrui Tang. 2021a. {``Targeting Glioblastoma Stem Cells: A
Review on Biomarkers, Signal Pathways and Targeted Therapy.''}
\emph{Frontiers in Oncology} 11 (July): 701291.
\url{https://doi.org/10.3389/fonc.2021.701291}.

\bibitem[\citeproctext]{ref-tang2021a}
---------. 2021b. {``Targeting Glioblastoma Stem Cells: A Review on
Biomarkers, Signal Pathways and Targeted Therapy.''} \emph{Frontiers in
Oncology} 11 (July): 701291.
\url{https://doi.org/10.3389/fonc.2021.701291}.

\bibitem[\citeproctext]{ref-turner2009}
Turner, C., A. R. Stinchcombe, M. Kohandel, S. Singh, and S.
Sivaloganathan. 2009. {``Characterization of Brain Cancer Stem Cells: A
Mathematical Approach.''} \emph{Cell Proliferation} 42 (4): 529--40.
\url{https://doi.org/10.1111/j.1365-2184.2009.00619.x}.

\bibitem[\citeproctext]{ref-tzeng2011}
Tzeng, Stephany Y., Hugo Guerrero-Cázares, Elliott E. Martinez, Joel C.
Sunshine, Alfredo Quiñones-Hinojosa, and Jordan J. Green. 2011.
{``Non-Viral Gene Delivery Nanoparticles Based on Poly(beta-Amino
Esters) for Treatment of Glioblastoma.''} \emph{Biomaterials} 32 (23):
5402--10. \url{https://doi.org/10.1016/j.biomaterials.2011.04.016}.

\bibitem[\citeproctext]{ref-vlashi2009}
Vlashi, Erina, Kwanghee Kim, Chann Lagadec, Lorenza Della Donna, John
Tyson McDonald, Mansoureh Eghbali, James W. Sayre, Encrico Stefani,
William McBride, and Frank Pajonk. 2009. {``In Vivo Imaging, Tracking,
and Targeting of Cancer Stem Cells.''} \emph{JNCI: Journal of the
National Cancer Institute} 101 (5): 350--59.
\url{https://doi.org/10.1093/jnci/djn509}.

\bibitem[\citeproctext]{ref-wang2019}
Wang, Hanwen, Oleg Milberg, Imke H. Bartelink, Paolo Vicini, Bing Wang,
Rajesh Narwal, Lorin Roskos, Cesar A. Santa-Maria, and Aleksander S.
Popel. 2019. {``{\emph{In Silico}} Simulation of a Clinical Trial with
Anti-CTLA-4 and Anti-PD-L1 Immunotherapies in Metastatic Breast Cancer
Using a Systems Pharmacology Model.''} \emph{Royal Society Open Science}
6 (5): 190366. \url{https://doi.org/10.1098/rsos.190366}.

\bibitem[\citeproctext]{ref-wang2020}
Wang, Hanwen, Richard J. Sové, Mohammad Jafarnejad, Sondra Rahmeh,
Elizabeth M. Jaffee, Vered Stearns, Evanthia T. Roussos Torres, Roisin
M. Connolly, and Aleksander S. Popel. 2020. {``Conducting a Virtual
Clinical Trial in HER2-Negative Breast Cancer Using a Quantitative
Systems Pharmacology Model with an Epigenetic Modulator and Immune
Checkpoint Inhibitors.''} \emph{Frontiers in Bioengineering and
Biotechnology} 8 (February).
\url{https://doi.org/10.3389/fbioe.2020.00141}.

\bibitem[\citeproctext]{ref-weiss2017a}
Weiss, Lora D., Natalia L. Komarova, and Ignacio A. Rodriguez-Brenes.
2017. {``Mathematical Modeling of Normal and Cancer Stem Cells.''}
\emph{Current Stem Cell Reports} 3 (3): 232--39.
\url{https://doi.org/10.1007/s40778-017-0094-4}.

\bibitem[\citeproctext]{ref-weller2017}
Weller, Michael, Nicholas Butowski, David D Tran, Lawrence D Recht,
Michael Lim, Hal Hirte, Lynn Ashby, et al. 2017. {``Rindopepimut with
Temozolomide for Patients with Newly Diagnosed, EGFRvIII-Expressing
Glioblastoma (ACT IV): A Randomised, Double-Blind, International Phase 3
Trial.''} \emph{The Lancet Oncology} 18 (10): 1373--85.
\url{https://doi.org/10.1016/S1470-2045(17)30517-X}.

\bibitem[\citeproctext]{ref-wu2022}
Wu, Chengyue, Guillermo Lorenzo, David A. Hormuth, Ernesto A. B. F.
Lima, Kalina P. Slavkova, Julie C. DiCarlo, John Virostko, et al. 2022.
{``Integrating Mechanism-Based Modeling with Biomedical Imaging to Build
Practical Digital Twins for Clinical Oncology.''} \emph{Biophysics
Reviews} 3 (2). \url{https://doi.org/10.1063/5.0086789}.

\bibitem[\citeproctext]{ref-yan2016}
Yan, Min, and Quentin Liu. 2016. {``Differentiation Therapy: A Promising
Strategy for Cancer Treatment.''} \emph{Chinese Journal of Cancer} 35
(1): 3. \url{https://doi.org/10.1186/s40880-015-0059-x}.

\bibitem[\citeproctext]{ref-yang2019}
Yang, Wei, Nicole M. Warrington, Sara J. Taylor, Paula Whitmire, Eduardo
Carrasco, Kyle W. Singleton, Ningying Wu, et al. 2019. {``Sex
Differences in GBM Revealed by Analysis of Patient Imaging,
Transcriptome, and Survival Data.''} \emph{Science Translational
Medicine} 11 (473): eaao5253.
\url{https://doi.org/10.1126/scitranslmed.aao5253}.

\bibitem[\citeproctext]{ref-yankeelov2024}
Yankeelov, Thomas E., David A. Hormuth, Ernesto A. B. F. Lima, Guillermo
Lorenzo, Chengyue Wu, Lois C. Okereke, Gaiane M. Rauch, Aradhana M.
Venkatesan, and Caroline Chung. 2024. {``Designing Clinical Trials for
Patients Who Are Not Average.''} \emph{iScience} 27 (1): 108589.
\url{https://doi.org/10.1016/j.isci.2023.108589}.

\bibitem[\citeproctext]{ref-youssefpour2012}
Youssefpour, H., X. Li, A. D. Lander, and J. S. Lowengrub. 2012.
{``Multispecies Model of Cell Lineages and Feedback Control in Solid
Tumors.''} \emph{Journal of Theoretical Biology} 304 (July): 39--59.
\url{https://doi.org/10.1016/j.jtbi.2012.02.030}.

\bibitem[\citeproctext]{ref-yu2015}
Yu, Victoria Y., Dan Nguyen, Frank Pajonk, Patrick Kupelian, Tania
Kaprealian, Michael Selch, Daniel A. Low, and Ke Sheng. 2015.
{``Incorporating Cancer Stem Cells in Radiation Therapy Treatment
Response Modeling and the Implication in Glioblastoma Multiforme
Treatment Resistance.''} \emph{International Journal of Radiation
Oncology*Biology*Physics} 91 (4): 866--75.
\url{https://doi.org/10.1016/j.ijrobp.2014.12.004}.

\bibitem[\citeproctext]{ref-zahid2021}
Zahid, Mohammad U., Abdallah S. R. Mohamed, Jimmy J. Caudell, Louis B.
Harrison, Clifton D. Fuller, Eduardo G. Moros, and Heiko Enderling.
2021. {``Dynamics-Adapted Radiotherapy Dose (DARD) for Head and Neck
Cancer Radiotherapy Dose Personalization.''} \emph{Journal of
Personalized Medicine} 11 (11): 1124.
\url{https://doi.org/10.3390/jpm11111124}.

\end{CSLReferences}




\end{document}
