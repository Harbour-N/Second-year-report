% Options for packages loaded elsewhere
\PassOptionsToPackage{unicode}{hyperref}
\PassOptionsToPackage{hyphens}{url}
\PassOptionsToPackage{dvipsnames,svgnames,x11names}{xcolor}
%
\documentclass[
  letterpaper,
]{scrreprt}

\usepackage{amsmath,amssymb}
\usepackage{iftex}
\ifPDFTeX
  \usepackage[T1]{fontenc}
  \usepackage[utf8]{inputenc}
  \usepackage{textcomp} % provide euro and other symbols
\else % if luatex or xetex
  \usepackage{unicode-math}
  \defaultfontfeatures{Scale=MatchLowercase}
  \defaultfontfeatures[\rmfamily]{Ligatures=TeX,Scale=1}
\fi
\usepackage{lmodern}
\ifPDFTeX\else  
    % xetex/luatex font selection
\fi
% Use upquote if available, for straight quotes in verbatim environments
\IfFileExists{upquote.sty}{\usepackage{upquote}}{}
\IfFileExists{microtype.sty}{% use microtype if available
  \usepackage[]{microtype}
  \UseMicrotypeSet[protrusion]{basicmath} % disable protrusion for tt fonts
}{}
\makeatletter
\@ifundefined{KOMAClassName}{% if non-KOMA class
  \IfFileExists{parskip.sty}{%
    \usepackage{parskip}
  }{% else
    \setlength{\parindent}{0pt}
    \setlength{\parskip}{6pt plus 2pt minus 1pt}}
}{% if KOMA class
  \KOMAoptions{parskip=half}}
\makeatother
\usepackage{xcolor}
\usepackage[lmargin=2cm,rmargin=2cm,tmargin=3cm,bmargin=3cm]{geometry}
\setlength{\emergencystretch}{3em} % prevent overfull lines
\setcounter{secnumdepth}{5}
% Make \paragraph and \subparagraph free-standing
\makeatletter
\ifx\paragraph\undefined\else
  \let\oldparagraph\paragraph
  \renewcommand{\paragraph}{
    \@ifstar
      \xxxParagraphStar
      \xxxParagraphNoStar
  }
  \newcommand{\xxxParagraphStar}[1]{\oldparagraph*{#1}\mbox{}}
  \newcommand{\xxxParagraphNoStar}[1]{\oldparagraph{#1}\mbox{}}
\fi
\ifx\subparagraph\undefined\else
  \let\oldsubparagraph\subparagraph
  \renewcommand{\subparagraph}{
    \@ifstar
      \xxxSubParagraphStar
      \xxxSubParagraphNoStar
  }
  \newcommand{\xxxSubParagraphStar}[1]{\oldsubparagraph*{#1}\mbox{}}
  \newcommand{\xxxSubParagraphNoStar}[1]{\oldsubparagraph{#1}\mbox{}}
\fi
\makeatother


\providecommand{\tightlist}{%
  \setlength{\itemsep}{0pt}\setlength{\parskip}{0pt}}\usepackage{longtable,booktabs,array}
\usepackage{calc} % for calculating minipage widths
% Correct order of tables after \paragraph or \subparagraph
\usepackage{etoolbox}
\makeatletter
\patchcmd\longtable{\par}{\if@noskipsec\mbox{}\fi\par}{}{}
\makeatother
% Allow footnotes in longtable head/foot
\IfFileExists{footnotehyper.sty}{\usepackage{footnotehyper}}{\usepackage{footnote}}
\makesavenoteenv{longtable}
\usepackage{graphicx}
\makeatletter
\def\maxwidth{\ifdim\Gin@nat@width>\linewidth\linewidth\else\Gin@nat@width\fi}
\def\maxheight{\ifdim\Gin@nat@height>\textheight\textheight\else\Gin@nat@height\fi}
\makeatother
% Scale images if necessary, so that they will not overflow the page
% margins by default, and it is still possible to overwrite the defaults
% using explicit options in \includegraphics[width, height, ...]{}
\setkeys{Gin}{width=\maxwidth,height=\maxheight,keepaspectratio}
% Set default figure placement to htbp
\makeatletter
\def\fps@figure{htbp}
\makeatother
% definitions for citeproc citations
\NewDocumentCommand\citeproctext{}{}
\NewDocumentCommand\citeproc{mm}{%
  \begingroup\def\citeproctext{#2}\cite{#1}\endgroup}
\makeatletter
 % allow citations to break across lines
 \let\@cite@ofmt\@firstofone
 % avoid brackets around text for \cite:
 \def\@biblabel#1{}
 \def\@cite#1#2{{#1\if@tempswa , #2\fi}}
\makeatother
\newlength{\cslhangindent}
\setlength{\cslhangindent}{1.5em}
\newlength{\csllabelwidth}
\setlength{\csllabelwidth}{3em}
\newenvironment{CSLReferences}[2] % #1 hanging-indent, #2 entry-spacing
 {\begin{list}{}{%
  \setlength{\itemindent}{0pt}
  \setlength{\leftmargin}{0pt}
  \setlength{\parsep}{0pt}
  % turn on hanging indent if param 1 is 1
  \ifodd #1
   \setlength{\leftmargin}{\cslhangindent}
   \setlength{\itemindent}{-1\cslhangindent}
  \fi
  % set entry spacing
  \setlength{\itemsep}{#2\baselineskip}}}
 {\end{list}}
\usepackage{calc}
\newcommand{\CSLBlock}[1]{\hfill\break\parbox[t]{\linewidth}{\strut\ignorespaces#1\strut}}
\newcommand{\CSLLeftMargin}[1]{\parbox[t]{\csllabelwidth}{\strut#1\strut}}
\newcommand{\CSLRightInline}[1]{\parbox[t]{\linewidth - \csllabelwidth}{\strut#1\strut}}
\newcommand{\CSLIndent}[1]{\hspace{\cslhangindent}#1}

\makeatletter
\@ifpackageloaded{bookmark}{}{\usepackage{bookmark}}
\makeatother
\makeatletter
\@ifpackageloaded{caption}{}{\usepackage{caption}}
\AtBeginDocument{%
\ifdefined\contentsname
  \renewcommand*\contentsname{Table of contents}
\else
  \newcommand\contentsname{Table of contents}
\fi
\ifdefined\listfigurename
  \renewcommand*\listfigurename{List of Figures}
\else
  \newcommand\listfigurename{List of Figures}
\fi
\ifdefined\listtablename
  \renewcommand*\listtablename{List of Tables}
\else
  \newcommand\listtablename{List of Tables}
\fi
\ifdefined\figurename
  \renewcommand*\figurename{Figure}
\else
  \newcommand\figurename{Figure}
\fi
\ifdefined\tablename
  \renewcommand*\tablename{Table}
\else
  \newcommand\tablename{Table}
\fi
}
\@ifpackageloaded{float}{}{\usepackage{float}}
\floatstyle{ruled}
\@ifundefined{c@chapter}{\newfloat{codelisting}{h}{lop}}{\newfloat{codelisting}{h}{lop}[chapter]}
\floatname{codelisting}{Listing}
\newcommand*\listoflistings{\listof{codelisting}{List of Listings}}
\usepackage{amsthm}
\theoremstyle{definition}
\newtheorem{definition}{Definition}[chapter]
\theoremstyle{remark}
\AtBeginDocument{\renewcommand*{\proofname}{Proof}}
\newtheorem*{remark}{Remark}
\newtheorem*{solution}{Solution}
\newtheorem{refremark}{Remark}[chapter]
\newtheorem{refsolution}{Solution}[chapter]
\makeatother
\makeatletter
\makeatother
\makeatletter
\@ifpackageloaded{caption}{}{\usepackage{caption}}
\@ifpackageloaded{subcaption}{}{\usepackage{subcaption}}
\makeatother

\ifLuaTeX
  \usepackage{selnolig}  % disable illegal ligatures
\fi
\usepackage{bookmark}

\IfFileExists{xurl.sty}{\usepackage{xurl}}{} % add URL line breaks if available
\urlstyle{same} % disable monospaced font for URLs
\hypersetup{
  pdftitle={Second year progress review},
  pdfauthor={Nicholas Harbour},
  colorlinks=true,
  linkcolor={blue},
  filecolor={Maroon},
  citecolor={Blue},
  urlcolor={Blue},
  pdfcreator={LaTeX via pandoc}}


\title{Second year progress review}
\author{Nicholas Harbour}
\date{2024-09-02}

\begin{document}
\cleardoublepage
\thispagestyle{empty}
{\centering
{\Huge\bfseries Second year progress review \par}
\vspace{12ex}
{\Large\bfseries Nicholas Harbour \par}
\vspace{3ex}
{\Large ORCID: 0009-0008-2424-4516 \par}
{\bfseries\large 2024-09-02 \par}
\vspace{12ex}
%
%
{\bfseries\large Centre for Mathematical Medicine and Biology, School of
Mathematical Sciences, University of Nottingham, Nottingham, NG7 2RD,
UK \par}
%
\vspace{12ex}
{\small Supervised by: Markus Owen and Matthew Hubbard\par}
}

\renewcommand*\contentsname{Table of contents}
{
\hypersetup{linkcolor=}
\setcounter{tocdepth}{2}
\tableofcontents
}

\bookmarksetup{startatroot}

\chapter*{Preface}\label{preface}
\addcontentsline{toc}{chapter}{Preface}

\markboth{Preface}{Preface}

Thanks to my supervisors: Markus Owen and Matthew Hubbard!!

This is a Quarto book, to view it as HTML you can go here:
\url{https://harbour-n.github.io/Second-year-report/}.

The report is structured as follows. In the first chapter I give a
summary of the results in my PhD over the first two years, then I
outline a brief plan for the remaining time of my PhD. In hte second
chapter I present a literature review specifically focusing on cancer
stems cell modeling. In the third chapter I present my preprint
``Virtual Clinical Trials of BMP4 Differentiation Therapy: Digital Twins
to Aid Glioblastoma Trial Design'', this constitutes the largest part of
my that will for a significant chapter of my thesis.

\bookmarksetup{startatroot}

\chapter{Summary}\label{summary}

This is the summary of main results and plan for thesis.

what thesis is about Summary first of what done in last year, then
current year IMO, data-thon, Mayo U54 trip, SMB then thesis

Plan for thesis: Thesis by papers?

possible papers: - virtual clinical trials for BMP4. - More math bio
journal analysis of the model (some of the stuff Markus already stated
look at with the model; nullclines, model reduction, also think looking
at pde and ideas of tumour growth paradox). - More digital twin PI
modeling using data from patient view/Mayo. Incorporating uncertainty in
DT model. corresponds with Turing institute placement. - Stuff with
monocle, cell cycle\\
- Frontiers of young minds paper.

\bookmarksetup{startatroot}

\chapter{Literature review}\label{sec-lit-review}

In this chapter we present a brief literature review of stem cells in
cancer and the modeling approaches that have been used.

\section{Introduction}\label{sec-introduction}

Stem cells are defined as cells that have the ability to perpetuate
themselves through self-renewal and to generate mature cells of a
particular tissue through differentiation (\citeproc{ref-reya2001}{Reya
et al. 2001}). Stem cells are fundamental to tissue maintenance and
repair; they also play a critical role in cancer development and in
determining the outcomes of cancer treatment
(\citeproc{ref-weiss2017a}{Weiss, Komarova, and Rodriguez-Brenes 2017}).

\section{Stem cells in cancer}\label{sec-stem-cells-in-cancer}

Perhaps the most important and useful property of stem cells is that of
self-renewal. Self-renewal is crucial to stem cell function, because it
is required by the majority of stem cells to persist for the lifetime of
the animal. Moreover, whereas stem cells from different organs may vary
in their developmental potential, all stem cells must self-renew and
regulate the relative balance between self-renewal and differentiation.
Understanding the regulation of normal stem cell self-renewal is also
fundamental to understanding the regulation of cancer cell
proliferation, because cancer can be considered to be a disease of
unregulated self-renewal (\citeproc{ref-reya2001}{Reya et al. 2001}).
Another distinguishing hallmark of stem cells is the ability to undergo
asymmetric division, during which stem cells give rise to daughter cells
of different fates, proliferative potential, size or other
characteristics (\citeproc{ref-majumdar2020}{Majumdar and Liu 2020};
\citeproc{ref-hitomi2021}{Hitomi et al. 2021}). CSCs generate such
diverse progeny by executing multiple modes of cell division, lineage
tracing experiments in glioma cells revealed that CSC undergo three main
types of cell division. 1) Symmetric CSC self-renewing division; 2)
symmetric differentiating division; 3) asymmetric differentiation,
additionally less than 1\% of cell divisions resulted in cell death
(\citeproc{ref-lathia2011}{Lathia et al. 2011}). The types of CSC cell
division are summarized in FIGURE ??

Numerous arguments suggest a stem-cell origin for human cancers. First,
it is worth noting that stem cells possess many of the features that
constitute the tumour phenotype, including self-renewal and essentially
unlimited replicative potential (\citeproc{ref-hanahan2000}{Hanahan and
Weinberg 2000}). Second, the mutations that initiate tumour formation
seem to accumulate in cells that persist throughout life, as suggested
by the exponential increase of cancer incidence with age
(\citeproc{ref-meza2008}{Meza et al. 2008}). This is thought to reflect
a requirement for between four and seven mutations in a single cell to
effect malignant transformation (\citeproc{ref-hanahan2000}{Hanahan and
Weinberg 2000}). Similarly, cancer formation from cells that persist
throughout life is suggested by an increased incidence in adults of skin
tumours such as melanoma after higher childhood exposure to a mutagenic
agent such as ultraviolet solar radiation (\citeproc{ref-balk2011}{Balk
2011}). Normal somatic stem cells are strong candidates for such
persistent cells. An alternative explanation would be that mutation
within a more differentiated cell might break the normal
growth-regulatory mechanisms that limit its proliferative capacity and
result in a persistent clone of proliferating cells. However, this seems
less likely because non-stem cells are generally destined for terminal
differentiation within a time window too short for acquisition of
sequential mutations that must affect two copies of a wild-type tumour
suppressor (\citeproc{ref-reya2001}{Reya et al. 2001}).

The identification of a stem cell origin for human cancers was first
identified in leukaemia's, perhaps because the hight fraction of stem
cells in haematopoietic system, when it was discovered that some, but
not all, cancer cells where able to initiate tumours of the blood
(\citeproc{ref-taipale2001}{Taipale and Beachy 2001};
\citeproc{ref-lapidot1994}{Lapidot et al. 1994};
\citeproc{ref-bonnet1997}{Bonnet and Dick 1997}). More recently CSCs
have been identified in many solid tumours including breast, colon and
brain (\citeproc{ref-al-hajj2003}{Al-Hajj et al. 2003};
\citeproc{ref-ricci-vitiani2007}{Ricci-Vitiani et al. 2007};
\citeproc{ref-ignatova2002}{Ignatova et al. 2002};
\citeproc{ref-hemmati2003}{Hemmati et al. 2003};
\citeproc{ref-singh2004}{Singh et al. 2004};
\citeproc{ref-galli2004}{Galli et al. 2004}). In GBM, cells expressing
the CD133 cell surface protein marker (also found on neural stem cells)
have been identified as having stem cell properties \emph{in vitro}
(\citeproc{ref-singh2003}{Singh et al. 2003}). Furthermore, when tested
using a xenograft assay, it was found that injection of as few as 100
CD133+ cells produced a tumour that could be serially transplanted and
was phenotypically similar to the patients original tumour, while
injection of \(10^5\) CD133- cells engrafted but did not cause a tumour
(\citeproc{ref-singh2004}{Singh et al. 2004}). This provides strong
evidence that there is a small subpopulation of glioma stem cells that
have the unique ability to initiate tumours, while the majority of cells
cannot.

\section{Cancer stem cells and treatment
resistance}\label{sec-cancer-stem-cells-and-treatment-resistance}

Radiation therapy is the most common form of treatment across all
cancers, with around 50\% of all cancer patients receiving radiotherapy
at some point in their treatment (\citeproc{ref-baskar2012}{Baskar et
al. 2012}). However, in addition to being tumor initiating, CSCs are
highly resistant to both radio- and chemo-therapy through preferential
activation of the DNA damage checkpoint response and an increase in DNA
repair capacity (\citeproc{ref-bao2006}{Bao et al. 2006};
\citeproc{ref-tang2021}{Tang et al. 2021}; \citeproc{ref-rich2007}{Rich
2007}; \citeproc{ref-schonberg2014}{Schonberg et al. 2014}). In glioma,
experimental results have shown that both in culture and mouse models
CD133-expressing stem cells survive radiation in larger proportions than
the majority of tumour cells which lack CD133 expression; these results
suggest that CSC confer radio-resistance in GBM and ultimately are the
source of tumour recurrence after radiation (\citeproc{ref-bao2006}{Bao
et al. 2006}).

In addition to being resistant to treatment CSCs also engage in a
synergistic relationship with the surrounding tumor microenvironment
(TME) to promote angiogenesis, proliferation, migration, tumor survival,
and immune evasion (\citeproc{ref-ma2018}{Ma et al. 2018};
\citeproc{ref-rich2007}{Rich 2007}). Taken together this highlights the
important role CSC play in determining tumour response to therapy. There
is a desperate need for targeted therapies that either directly kill
CSCs or sensitize CSCs to cytotoxic therapies in order to improve
treatment outcomes.

\section{Mathematical models of cancer stem cell
dynamics}\label{sec-mathematical-models-of-cancer-stem-cell-dynamics}

Many different mathematical models have been developed to model stem
cell dynamics. Understanding CSC kinetics and interaction with their
non-stem counterparts is still spares and theoretic/mathematical models
may help elucidate their role in cancer progression and treatment
response. Here we focus on a small subset of models that are used in the
literature that cover a wide range of modelling techniques.

Many of the following models use slightly different terminology to refer
to the non-stem cell population such as cancer cell, progenitor cells or
tumour cells, for clarity we will refer to non-stem cells always as
tumour cells (TCs) throughout this review.

\subsection{Agent-based model}\label{sec-agent-based-model}

in (\citeproc{ref-enderling2009}{Enderling et al. 2009}) and
(\citeproc{ref-gao2013}{Gao et al. 2013}) the authors develop an
agent-based model (ABM) to study the dynamics of CSCs and TCs in a
tumour. It is assumed that tumours are a heterogenous mix of CSCs and
TCs. Cells are considered as individual entities with a cell cycle and
limited proliferation capacity \(\rho = [0,\rho_\text{max}]\). CSCs have
unlimited self-renewal, hence \(\rho_\text{max} = \infty\). At each cell
division CSCs can undergo symmetric division with probability \(\delta\)
or asymmetric division with probability \(1-\delta\). The proliferation
capacity \(\rho\) is decremented at each TC division and inherited by
both daughter cells.

Simulations of the ABM model revealed the following key results

\begin{itemize}
\tightlist
\item
  Tumours developing solely from TCs will inevitably die out, due to
  their limited proliferation capacity. Hence, CSCs are necessary for
  malignant tumour growth. This is consistent with experimental results
  showing only CSCs can initiate tumours
  (\citeproc{ref-lapidot1994}{Lapidot et al. 1994};
  \citeproc{ref-singh2004}{Singh et al. 2004}).
\item
  Tumour started without CSCs could still persist for a long time as
  long-term dormant lesions, but due to space limited growth remain
  small well below the potential maximum size of
  \(2^{\rho_\text{max}}\). This is consistent with the observation that
  many tumours remain dormant for many years before they start to grow
  (\citeproc{ref-sweeney1995}{Sweeney 1995};
  \citeproc{ref-neves-e-castro2006}{Neves-E-Castro 2006};
  \citeproc{ref-folkman2004}{Folkman and Kalluri 2004}).
\item
  A high rate of spontaneous death of TCs actually enables room for
  sufficient stem cell divisions to enrich the stem cell pool and drive
  tumour growth. This lead to what they call the ``tumour growth
  paradox'', where counterintuitively while an increase in the death
  rate of TCs decreases the total number of cancer cells in the short
  term, in the long run it leads to an increase in the total tumour size
  as the tumour contains more CSCs.
\end{itemize}

Mathematically the tumour growth paradox is defined as follows.

\begin{definition}[Tumour growth
paradox]\protect\hypertarget{def-tumor-growth-paradox}{}\label{def-tumor-growth-paradox}

Let \(N_\alpha (t)\) denote a total tumour population with death rate
\(\alpha\) for TCs. The population exhibits a tumour growth paradox if
there exist death rates \(\alpha_1 < \alpha_2\) and times \(t_1,t_2\)
and \(T_0\) such that

\begin{equation}\phantomsection\label{eq-tumour-growth-paradox}{
\begin{aligned}
N_{\alpha_1}(t_1) = N_{\alpha_2}(t_2)& \quad \text{and} \quad N_{\alpha_1}(t_1 + T) < N_{\alpha_2}(t_2 + T) \\
&\text{for} \quad (0<T<T_0) 
\end{aligned}
}\end{equation}

\end{definition}

\subsection{Integro-differential
model}\label{sec-integro-differential-model}

Following on from the ABM developed in
(\citeproc{ref-enderling2009}{Enderling et al. 2009};
\citeproc{ref-gao2013}{Gao et al. 2013}) in
(\citeproc{ref-hillen2013}{Hillen, Enderling, and Hahnfeldt 2013}) the
authors develop an integro-differential equation version of the model.
The model is based on the same assumptions as in
(\citeproc{ref-enderling2009}{Enderling et al. 2009};
\citeproc{ref-gao2013}{Gao et al. 2013}), but this time takes the form
of an integro-differential equation model. Let \(u(x,t)\) denote the
density (in cells per unit space i.e., the fraction of the interval
\((x,x+dx)\) physically occupied by cells), and let \(v(x,t)\) denote
the density of TCs. Hence, the total tumor density is denoted
\(N(x,t) = u(x,t) + v(x,t)\). For this analysis cells are assumed to be
very small compared to the size of the tissue domain \(\Omega\) (which
we take without loss of generality to have unit volume), and are small
even compared to integration increments \(dx\) and \(dy\). It is also
assumed that cells cannot pile on top of each other so there is a
maximum density of one cell per unit space, this implies
\(N(x,t) \leq 1\). Cells can only proliferate if there is space to place
the daughter cell, otherwise reproduction is inhibited (cellular
quiescence). To model the spatial search for space, they define a
nonlinear integral term, and inline with the ABM
(\citeproc{ref-enderling2009}{Enderling et al. 2009};
\citeproc{ref-gao2013}{Gao et al. 2013}) they assume that all cells can
migrate randomly, which is model by simple diffusion. These assumptions
lead to the following system of equations to describe CSC and TC
dynamics:

\begin{equation}\phantomsection\label{eq-integro-differential-CSC-model}{
\begin{aligned}
    \underbrace{\frac{\partial u(x,t)}{\partial t}}_\text{ROC CSCs} &= \underbrace{D_u \nabla^2 u}_\text{Diffusion of CSCs} + \underbrace{\delta \gamma \int_{\Omega} k(x,y,N(x,t))u(y,t) dy}_\text{Self-renewal of CSCs} \\
    \underbrace{\frac{\partial v(x,t)}{\partial t}}_\text{ROC TCs} &= \underbrace{D_v \nabla^2 v}_\text{Diffusion of TCs} + \underbrace{(1-\delta) \gamma \int_{\Omega} k(x,y,N(x,t))u(y,t) dy}_\text{Differentiation of CSCs} + \\
    & \underbrace{\rho \int_{\Omega} k(x,y,N(x,t))v(y,t) dy}_\text{Proliferation of TCs} - \underbrace{\alpha v}_\text{Apoptosis of TCs}.
\end{aligned}
}\end{equation}

The spatial distribution kernel \(k(x,y,N)\) describes the rate of
progeny contribution to location x for a cell at location y per ``cell
cycle time'' i.e., the defined period between divisions of a freely
cycling cell. Since greater density at \(x\) would be expected to hinder
progeny occupation it is assumed that \(k\) is monotonically decreasing
in \(N\), with \(k(x,y,N(x,t))=0\) at \(N=1\). The number of cell cycle
times per unit time of CSCs and TCs are denoted by \(\gamma\) and
\(\rho\), respectively and for simplicity it is assumed that
\(\gamma = \rho = 1\) throughout. The parameter \(\delta\) with
\(0 \leq \delta \leq 1\) denotes the fraction of CSC divisions that are
symmetric, while \(1-\delta\) is the fraction of asymmetric divisions.
The parameter \(\alpha\) denotes the spontaneous death rate of TCs.
Background cell motility is modelled by the diffusion coefficients
\(D_u\) and \(D_v\) for CSCs and TCs, respectively. The system is
considered to hold in a smooth bounded domain \(\Omega\), with
homogenous Neumann or Dirichlet boundary conditions.

Homogeneous Neumann boundary conditions correspond to a boundary that is
impenetrable by cells this could for example represent, a tissue
surrounded by membranes, smooth muscle tissue, or bone, and are given by

\begin{equation}\phantomsection\label{eq-homogeneous-neumann-boundary-conditions}{
\frac{\partial u}{\partial n} = 0, \quad \frac{\partial v}{\partial n} = 0 \quad \text{on} \quad \partial \Omega,
}\end{equation}

where \(\partial / \partial n\) is the normal derivative at the
boundary. The redistribution kernel can only redistribute cells within
this domain \(\Omega\), hence we impose

\begin{equation}\phantomsection\label{eq-redistribution-kernel-neumann-BC}{
k(x,y,N) = 0 \quad \text{for} \quad x \notin \Omega. 
}\end{equation}

Homogeneous Dirichlet boundary conditions correspond to a boundary that
cells can freely leave but not reenter again, for example this could
represent intravasation into adjacent blood vessels, and are given by

\begin{equation}\phantomsection\label{eq-homogeneous-dirichlet-boundary-conditions}{
u = 0, \quad v = 0 \quad \text{on} \quad \partial \Omega.
}\end{equation}

The redistribution kernel describes transport of cells out of the domain
but does not allow entering from the outside if, hence

\begin{equation}\phantomsection\label{eq-redistribution-kernel-dirichlet-BC}{
k(x,y,N) = 0 \quad \text{for} \quad y \notin \Omega.
}\end{equation}

Based on these two boundary conditions we can model any combination of
domains such as partially covered by membranes, partially permeable
membranes and adjacent blood vessels.

\subsection{ODE model reduction}\label{sec-ode-model-reduction}

In order to analyses this model analytically the authors reduce the
system of integro-differential equations
(Equation~\ref{eq-integro-differential-CSC-model}) to a system of
ordinary differential equations in the following way.

\textbf{Reduction 1: Progeny placement depends only on the density at
the destination}\\
In this case \(k(x,y,N(x,t)) = k(N(x,t))\). Introducing mean densities
which given that the domain \(\Omega\) has unit volume, can be written
as

\begin{equation}\phantomsection\label{eq-density-mean}{
\bar{u}(t) = \int_{\Omega} u(y,t) dy, \quad \bar{v}(t) = \int_{\Omega} v(y,t) dy, \quad \bar{N}(t) = \bar{u}(t) + \bar{v}(t).
}\end{equation}

Then Equation~\ref{eq-integro-differential-CSC-model} becomes:

\begin{equation}\phantomsection\label{eq-first-reduction-ODE-CSC-model}{
\begin{aligned}
u_t(x,t) &= D_u \nabla^2 u(x,t) + \delta   k(N(x,t))\bar{u}(t), \\
v_t(x,t) &= D_v \nabla^2 v(x,t) + (1-\delta)   k(N(x,t))\bar{u}(t) + k(N(x,t))\bar{v}(t) - \alpha v(x,t).
\end{aligned}
}\end{equation}

\textbf{Reduction 2: Density is uniform across the domain}\\
If tumour growth is assumed uniform across the domain then,
\(k(N(x,t)) = k(\bar{N}(t))\) and \(u(x,t)\) and \(v(x,t)\) can be
replaced with their spatially averaged values (\(\bar{u}(t)\) and
\(\bar{v}(t)\)) and diffusion is zero everywhere. Therefore,
Equation~\ref{eq-first-reduction-ODE-CSC-model} becomes:

\begin{equation}\phantomsection\label{eq-ODE-CSC-model}{
\begin{aligned}
    \frac{d \bar{u}}{dt} &= \delta  k(\bar{N}(t)) \bar{u}, \\
    \frac{d \bar{v}}{dt} &= (1-\delta)  k(\bar{N}(t)) \bar{u} + k(\bar{N}(t)) \bar{v} - \alpha \bar{v}(t),
\end{aligned}
}\end{equation}

where the volume filling constraint \(k(\bar{N})\) is taken to be

\begin{equation}\phantomsection\label{eq-volume-filling-constraint}{
k(\bar{N}) = \text{max} \left\{0, 1 - \bar{N}^\sigma \right\}. \quad \text{for} \quad \sigma > 1.
}\end{equation}

An exponent of \(\sigma = 1\) corresponds to a linearly decreasing rate
of occupancy for newborn cells as the total density \(\bar{N}\)
increases. Since cells are nonrigid, deformable and able to squeeze into
available spaces the authors argue that \(\sigma > 1\) is appropriate
and take it to be \(\sigma = 4\), in all their simulations.

Without a CSCs population \(\bar{u}(t)\), the density of TCs
\(\bar{v}(t)\) satisfies the equation

\begin{equation}\phantomsection\label{eq-ODE-TC-only-model}{
\frac{d\bar{v}}{dt}  =K(\bar{v}(t))\bar{v}(t) - \alpha \bar{v}(t).
}\end{equation}

Since \(K(\bar{v}(t))\) is a decreasing function of \(\bar{v}(t)\) the
TC population will die out when \(\alpha > k(0)\).

This simpler ODE model allows for analytical analysis of the steady
states, from which it can be shown that the pure stem cell steady state
\((u,v)= (1,0)\) is a global attractor. Therefore, this model predicts
that for long times the tumour will consist of only stem cells.
Intermediate tumor consistency and steady state time are dependent on
cell death rate \(\alpha\). The convergence to \((1,0)\) is somewhat
surprising as typically the CSC compartment is considered small
comprising only 1-3\% of the tumour (\citeproc{ref-bao2006}{Bao et al.
2006}). However, the authors argue that this does not interfere with our
analysis, since we are not interested in the long time dynamics
(\(t \rightarrow \infty\)), but rather in the intermediate time dynamics
of the tumor.

\subsection{Stochastic model of CSC
dynamics}\label{sec-stochastic-model}

In (\citeproc{ref-turner2009}{Turner et al. 2009}), the authors first
develop a stochastic model for the dynamics of CSCs and TCs,
particularly for the case of brain cancer. This stochastic model is
particular appropriate for situations in which small numbers of cells
are present such as \emph{in vitro} or in the early stages of tumour
formation. In these cases stochastic fluctuations may have significant
effects and cannot be neglected. To study larger populations the authors
then derive a deterministic ODE model, based on the stochastic master
equation, that describes the average number of CSCs and TCs.

The model assumptions on CSCs and TCs are largely similar as those given
previously in Section~\ref{sec-agent-based-model}. However, it is
assumed that CSCs are not immortal so have some probability of death.
Defining \(p(n_s, n_p,t)\) as the probability that there are exactly
\(n_s\) CSCs and \(n_p\) TCs at time \(t\), The stochastic master
equation is given by

\begin{equation}\phantomsection\label{eq-stochastic-master-equation}{
\begin{aligned}
\frac{dp(n_s,n_p,t)}{dt} = & \ \rho_s \left[r_1 (n_s - 1) p(n_s - 1, n_p, t) \right. \\
& + r_2 n_p p(n_s, n_p - 1, t) \\
& + r_3 (n_s + 1) p(n_s + 1, n_p - 2, t) \\
& \left. - n_s p(n_s, n_p, t) \right] \\
& + \Gamma_s \left[ (n_s + 1) p(n_s + 1, n_p, t) - n_s p(n_s, n_p, t) \right] \\
& + \Gamma_p \left[ (n_p + 1) p(n_s, n_p + 1, t) - n_p p(n_s, n_p, t) \right].
\end{aligned}
}\end{equation}

The parameters \(\Gamma_s\) and \(\Gamma_p\) represent the rate of
apoptosis for CSCs and TCs respectively. Due to the models stochastic
nature, and the inclusion of a death rate for CSCs, it can be shown that
The model predictions that the occurrence of a single cancer stem cell
will not necessarily result in a tumour, even if the probability of
self-renewal is greater than the probability of differentiation. This is
in contrast to the previous models discussed
(\citeproc{ref-enderling2009}{Enderling et al. 2009};
\citeproc{ref-hillen2013}{Hillen, Enderling, and Hahnfeldt 2013}) and to
a deterministic version of this model
(Equation~\ref{eq-deterministic-master-equation}) that would predict
exponential growth of the tumour from a single CSC.

For larger cellular populations it becomes more challenging to simulate
such a stochastic model and it becomes pertinent to consider the
equations for the average number of CSCs and TCs. Defining the mean
cellular populations \(S = <n_s>\), \(P = <n_p>\) and \(r = r_1 - r_3\),
the deterministic model is given by

\begin{equation}\phantomsection\label{eq-deterministic-master-equation}{
\begin{aligned}
\frac{dS}{dt} &= \rho_s r S - \Gamma_s S, \\
\frac{dP}{dt} &= \rho_s (1-r) S - \Gamma_p P.
\end{aligned}
}\end{equation}

This model is largely similar to the ODE model presented in
(\citeproc{ref-hillen2013}{Hillen, Enderling, and Hahnfeldt 2013}), the
slight differences that CSCs have a spontaneous death rate and TCs
cannot themselves proliferate reflect the slightly different underlying
assumptions of CSC dynamics between the two models.

\subsection{A multispecies model of cell
lineages}\label{sec-multispecies-model-of-cell-lineages}

In (\citeproc{ref-youssefpour2012}{Youssefpour et al. 2012}) the authors
develop a multispecies PDE model for CSCs lineage dynamics, for a review
of general multispecies models for modelling tumour dynamics see
(\citeproc{ref-lowengrub2010}{Lowengrub et al. 2010}). This is the first
model we have looked at that considers more than two cells types CSCs
and TCs. Here the model is more complex and accounts for CSCs, committed
progenitor cells, terminal cells, and dead cells. As with the previous
models we have looked at it is assumed that differentiation and feedback
processes link the cells lineage through self-renewal fractions and
mitosis rates. However, in contrast to the previous models it contains
endogenous production of differentiation promoter, denoted \(T\), which
is produced by the terminal cells and taken up by the CSCs.
Additionally, endogenous production of self-renewal promoter, denoted
\(W\), and an its inhibitor, denoted \(WI\), are assumed to be produced
by all viable tumour cells (CSCs, progenitor cells, terminal cells).
This allows for feedback mechanisms to that maintain the CSC population.
The dependent variables in the model are the local volume fractions of
the cell species
\(\phi_\text{CSC}, \phi_\text{CP}, \phi_\text{TC}, \phi_\text{DC}, as well as healthy cells and water \phi_\text{H}, \phi_\text{W}\).
Assuming there are no voids the sum of the volume fractions equals 1 and
each cell type follows a conservation equation of the form

\begin{equation}\phantomsection\label{eq-youssefpour-multispecies}{
\frac{\partial \phi}{\partial t} = - \nabla \cdot J - \nabla \cdot (u_s \phi) + S
}\end{equation}

where \(\psi\) denotes the volume fraction of the cell type, \(J\) is
generalized diffusion, \(u_s\) is the mass-averaged velocity of the
solid components, \(S\) denotes the mass-exchange terms.

\section{Models of differentiation
therapy}\label{sec-models-of-differentiation-therapy}

If, as with normal tissues, cellular phenotypic heterogeneity within
tumors can be explained by a hierarchy of differentiation, with only a
subset of stem-like cells capable of long-term self-renewal, this raises
the prospect that signals promoting differentiation could be effective
at driving malignant cells to a less aggressive and ideally post-mitotic
differentiated state (\citeproc{ref-caruxe9n2016}{Carén, Beck, and
Pollard 2016}). This differentiation therapy approach has seen success
in acute promyelocytic leukemia (APL) where all-trans-retinoic acid
(ATRA) can promote differentiation of CSCs and lead to complete
remission (\citeproc{ref-yan2016}{Yan and Liu 2016};
\citeproc{ref-dethuxe92018}{De Thé 2018}). In GBM, bone morphogenetic
protein 4 (BMP4), a member of the TGF-\(\beta\) superfamily, has shown
potential as a differentiation therapeutic agent. BMP4 has been shown to
drive differentiation of GSCs towards a predominantly glial (astrocytic)
fate, to reduce GBM tumor burden in vivo and to improve survival in a
mouse model of GBM (\citeproc{ref-nayak2020}{Nayak et al. 2020};
\citeproc{ref-piccirillo2006}{Piccirillo et al. 2006}). Despite its
potential as a treatment option relatively few mathematical models have
considered its possible effects on tumour growth
(\citeproc{ref-youssefpour2012}{Youssefpour et al. 2012};
\citeproc{ref-bachman2013}{Bachman and Hillen 2013};
\citeproc{ref-turner2009}{Turner et al. 2009}).

\subsection{Differentiation promoter and self-renewal
promoter}\label{differentiation-promoter-and-self-renewal-promoter}

In (\citeproc{ref-youssefpour2012}{Youssefpour et al. 2012}) they follow
(\citeproc{ref-lander2009}{Lander et al. 2009}) and assume that the
proliferation and differentiation of CSCs are regulated by factors in
the tumour microenvironemnt that feedback on self-renewal fractions and
mitosis rates. In particular, they denote the differentiation promoter
\(T\) (for TGF-beta superfamily members) that reduces the probability of
self-renewal for CSCs. They also account for self-renewal promoter \(W\)
which increases the probability of self-renewal of CSCs, as well as an
inhibitor of \(W\) denoted \(WI\), possible self-renewal promoters
include WNTs, Notch and Shh (\citeproc{ref-pannuti2010}{Pannuti et al.
2010}; \citeproc{ref-bailey2007}{Bailey, Singh, and Hollingsworth
2007}), they therefore define the CSC self-renewal fraction as

\begin{equation}\phantomsection\label{eq-youssefpour-Ps}{
P_s = P_\text{min} + (P_\text{max} - P_\text{min}) \left( \frac{\xi C_w}{1 + \xi C_W} \right)\left( \frac{1}{1 + \psi C_T} \right),
}\end{equation}

where \(P_\text{min}\) and \(P_\text{max}\) are the minimum and maximum
probability of CSC self-renewal, taken to be \(0.2\) and \(1\)
respectively. \(C_W\) and \(C_T\) represent the concentrations of the
self-renewal promoter and differentiation promoter respectively. The
parameters \(\xi\) and \(\psi\) quantify the sensitivity of CSCs to the
regulating proteins.

The concentration of differentiation promoter and self-renewal promoter
are then modeled as follows. It is assumed that \(T\) is more diffuse
than either \(W\) or \(WI\). Therefore, on the time scale of cellular
proliferation they make the quasi-steady-state assumption that time
derivatives and advection of \(T\) can be neglected. Thus, the
quasi-steady reaction-diffusion equation for \(T\) is given by

\begin{equation}\phantomsection\label{eq-youssefpour-T}{
0 = \nabla \cdot (D_T \nabla C_T) - \left( \nu_{UT} \phi + \nu_{DT} \right) C_T + \nu_{PT}\phi_{TC}
}\end{equation}

where \(D_T\) is the diffusion coefficient, \(\nu_{UT}\), \(\nu_{DT}\)
and \(\nu_\text{PT}\) are the uptake rate by CSCs, the rate of decay and
the rate of production by the TCs, respectively.

To model the self-renewal promoter \(W\) and its inhibitor \(WI\) a
generalized Gierer-Meinhard-Turing system of reaction diffusion
equations is used, given by

\begin{equation}\phantomsection\label{eq-youssefpour-W-WI}{
\begin{aligned}
&\frac{\partial C_W}{\partial t} + \nabla \cdot (u_s C_W) = \nabla \cdot (D_W \nabla C_W) + f(C_W, C_{WI}), \\
&\frac{\partial C_{WI}}{\partial t} + \nabla \cdot (u_s C_{WI}) = \nabla \cdot (D_{WI} \nabla C_{WI}) + g(C_W, C_{WI}),
\end{aligned}
}\end{equation}

where

\begin{equation}\phantomsection\label{eq-youssefpour-f-g}{
\begin{aligned}
f(C_W, C_{WI}) &= \nu_{PW} \frac{C^2_W}{C_{WI}}C_0 \phi_\text{CSC} - \nu_{DW} C_W + u_0C_0 (\phi_\text{CSC} + \phi_\text{CP} + \phi_\text{TC}), \\
g(C_W, C_{WI}) &= \nu_\text{PWI}C^2_W C_0 \phi_\text{CSC} - \nu_\text{DWI} C_{WI}.
\end{aligned}
}\end{equation}

The parameters \(D_W\) and \(D_{WI}\) are the difusion coefficeints,
\(\nu_\text{PW}\), \(\nu_\text{DW}\) and \(\nu_\text{PWI}\),
\(\nu_\text{DWI}\) are the production and decay rates of \(W\) and
\(WI\) respectively. The parameter \(u_0\) represnts a low-level source
of \(W\) from all the tumour cells.

To model differentiation therapy they fix \(\psi = 0.5\) and introduce
an external source of \(T\) i.e., a constant source term is added to
Equation~\ref{eq-youssefpour-T}.

\subsection{No self-renewal promoter}\label{no-self-renewal-promoter}

In (\citeproc{ref-bachman2013}{Bachman and Hillen 2013}) they follow
(\citeproc{ref-youssefpour2012}{Youssefpour et al. 2012}) and model
differentiation therapy through a relationship between the average level
of differentiation promoter, which they denote \(C_F\) and the
probability of CSC self-renewal \(P_s\). But hey do not include the
effects of a CSC self-renewal promoting factor, thus

\begin{equation}\phantomsection\label{eq-bachman-Ps}{
P_s(t) = P_\text{min} + (P_\text{max} - P_\text{min}) \left( \frac{1}{1 + \psi C_F(t)} \right),
}\end{equation}

where the notation is the same as in Equation~\ref{eq-youssefpour-Ps}.
Since (\citeproc{ref-bachman2013}{Bachman and Hillen 2013}) do not model
endogenous production of differentiation promoter, \(C_F\) solely
represents the level of differentiation promoter prescribed during
differentiation therpay. To address this lack of endogenous
differentiation promoters they take \(P_\text{max} = 0.505\) (which is
equivalent to setting \(\delta = 0.01\) as was done in
(\citeproc{ref-hillen2013}{Hillen, Enderling, and Hahnfeldt 2013})) and
\(P_\text{min} = 0.2\), as is done in
(\citeproc{ref-youssefpour2012}{Youssefpour et al. 2012}).

As (\citeproc{ref-bachman2013}{Bachman and Hillen 2013}) use the ODE
model developed in (\citeproc{ref-hillen2013}{Hillen, Enderling, and
Hahnfeldt 2013}) they must also develop a submodel for the average level
of differentiation promoter. As it is an ODE model they consider the
average level of differentiation promoter within a spatially homogeous
tumour \(C_F(t)\). It is assumed that the tumour resides within a
spherical region of tissue and that differentiation promoter enters this
area through the boundary. The differentaition promoter enters the
region from the boundary and will diffuse very quickly and attain a
steady state disttibution over this region. To compute the value of
\(C_F(t)\) they solve the problem of diffusion over a sphere of radius
\(R\) and avergae the solution over the volume of the sphere. We use the
lower case letter to describ the radial symmetric solution \(c_F(r,t)\)
of the follwoing boundary value probalem

\begin{equation}\phantomsection\label{eq-bachman-CF}{
\begin{aligned}
\frac{\partial c_F}{\partial t} &= \omega \left( \frac{\partial}{\partial r} \left(\frac{\partial c_F}{\partial r} \right) + \frac{2}{r} \frac{\partial c_f}{\partial r}   \right) \\
c_F(R,t) &= C_{F0}(t).
\end{aligned}
}\end{equation}

Where \(\omega\) is the effective diffusivity of the differentiation
promoter. Before differentation therapy begins \(C_{F0}(t)\) is 0, when
differnetiation therapy begins the boundary condition on the sphere is
set to \(C_{F0}(t) = 1\), and the promoter diffuses into the sphere.
When differentaition therapy ends, the bounayry condition is simply set
to 0 and the promoter diffusies out of the sphere. They then set

\begin{equation}\phantomsection\label{eq-bachman-CF2}{
C_F(t) = \frac{3}{R^3} = \int^R_0 c_F (r,t) r^2 dr.
}\end{equation}

\subsection{BMP4 in glioma}\label{bmp4-in-glioma}

In the previous models (\citeproc{ref-youssefpour2012}{Youssefpour et
al. 2012}; \citeproc{ref-hillen2013}{Hillen, Enderling, and Hahnfeldt
2013}) they considered a gneral differentaition promoter, in
(\citeproc{ref-turner2009}{Turner et al. 2009}) they consider a the
specific differentaiton promter, BMP4 in GBM. As in the case for the
general differentaiton promoter they interprest the effects of BMP4 as
decrreasing the net symmetric division rate \(r\) (following the
notatiomn used in Section~\ref{sec-stochastic-model}). Based on
(\citeproc{ref-piccirillo2006}{Piccirillo et al. 2006}) they estimate
that from a pre-treatment value of \(r = 0.1\) the effect of BMP4 is to
reduce \(r\) to \(-0.1\), note that following the notation used in
Section~\ref{sec-stochastic-model} \(r\) is defined as \(r = r_1-r_3\)
so a change of \(r\) to negative represnts a both an increaese in the
proportion of symmetric differentaiton divisions and a decrease in
symetric slef-renewal divisions. To model differentaiton therapy the
parameter value \(r\) is simply swtiched beteen these two values for the
duratio nof BMP4 exposure.

\subsection{Summary of differentaition therpay
results}\label{summary-of-differentaition-therpay-results}

All the models compare 3 different tretament cases radiation alone,
differentaition therapy alone and combination therpay. Importantly, as
has been shown, all models assume that CSCs are less senstavie to
radiation than TCs (\citeproc{ref-bao2006}{Bao et al. 2006}). Despite
the slight differences in implementation of the models, all models find
similar result. Radiotherpy alone fails as some CSCs survive and are
able to repopulate the tumour. In fact all models show an extention of
the ``tumour growth paradox'' which we term the ``tumour treatment
paradox''. When treating with radiation alone the fraction of CSCs
increases and since the TCs die there is more room for CSCs this allows
much more rapid re-grwoth of the tumour. This suggests that current
standard of care treatment selects for the more resistant and
turmourogenic CSCs thus treatment often fasicilitates more rapid and
agressive tumour recurrace. Diffenrentiation therpy alone can succesfuly
iradicate the tumour, however, given all models assume that CSCs can
only transform into TCs through (asymmetric) cellualr divison, rather
than direct transition, large intermediate values of total tumour size
may be reached using this approach. Combination of differentiation and
radiothery out perfomred eith single therapies often showing that the
tumour can be driven to much smaller sizes and potentially extinction.
This is because the differentiation agent induces CSCs to turn into TCs
which then can be killed by traditional radiation therapy. This
combination therapy can be considered a new class of strategies for
cancer therapy known as evolutionary steeringvappraoches. Rather than
reactively altering treatment as resitance is aqured we proactively
slecet our treamtent to minimise resistantce and increase chance of
extinction (\citeproc{ref-enriquez-navas2015}{Enriquez-Navas,
Wojtkowiak, and Gatenby 2015}; \citeproc{ref-acar2020}{Acar et al.
2020}).

\bookmarksetup{startatroot}

\chapter{Conclusion}\label{conclusion}

None of the models attempt to parameterise senstaitvit of cell lines to
differentiation promoter. None model a specific feasible delivery
strategy for differntiation promoter. None consider effects on cohorts
of virtual populations and interactions between other parameters and
differentiation therapy. In all cases it was assumed that CSCs can only
transform into TCs through asymetric divison rather than any direct
traistions from CSCs to TCs.

\bookmarksetup{startatroot}

\chapter{Preprint}\label{preprint}

This is the preprint it constitutes a substantial piece of work that
will contribute to my thesis. The preprint can be found here
\url{https://www.biorxiv.org/content/10.1101/2024.08.22.609156v1.full}
on bioR\(\chi\)iv.

\bookmarksetup{startatroot}

\chapter*{References}\label{references}
\addcontentsline{toc}{chapter}{References}

\markboth{References}{References}

\phantomsection\label{refs}
\begin{CSLReferences}{1}{0}
\bibitem[\citeproctext]{ref-acar2020}
Acar, Ahmet, Daniel Nichol, Javier Fernandez-Mateos, George D.
Cresswell, Iros Barozzi, Sung Pil Hong, Nicholas Trahearn, et al. 2020.
{``Exploiting Evolutionary Steering to Induce Collateral Drug
Sensitivity in Cancer.''} \emph{Nature Communications} 11 (1): 1923.
\url{https://doi.org/10.1038/s41467-020-15596-z}.

\bibitem[\citeproctext]{ref-al-hajj2003}
Al-Hajj, Muhammad, Max S. Wicha, Adalberto Benito-Hernandez, Sean J.
Morrison, and Michael F. Clarke. 2003. {``Prospective Identification of
Tumorigenic Breast Cancer Cells.''} \emph{Proceedings of the National
Academy of Sciences} 100 (7): 3983--88.
\url{https://doi.org/10.1073/pnas.0530291100}.

\bibitem[\citeproctext]{ref-bachman2013}
Bachman, Jeff W. N., and Thomas Hillen. 2013. {``Mathematical
Optimization of the Combination of Radiation and Differentiation
Therapies for Cancer.''} \emph{Frontiers in Oncology} 3.
\url{https://doi.org/10.3389/fonc.2013.00052}.

\bibitem[\citeproctext]{ref-bailey2007}
Bailey, Jennifer M., Pankaj K. Singh, and Michael A. Hollingsworth.
2007. {``Cancer Metastasis Facilitated by Developmental Pathways: Sonic
Hedgehog, Notch, and Bone Morphogenic Proteins.''} \emph{Journal of
Cellular Biochemistry} 102 (4): 829--39.
\url{https://doi.org/10.1002/jcb.21509}.

\bibitem[\citeproctext]{ref-balk2011}
Balk, Sophie J. 2011. {``Ultraviolet Radiation: A Hazard to Children and
Adolescents.''} \emph{Pediatrics} 127 (3): e791--817.
\url{https://doi.org/10.1542/peds.2010-3502}.

\bibitem[\citeproctext]{ref-bao2006}
Bao, Shideng, Qiulian Wu, Roger E. McLendon, Yueling Hao, Qing Shi,
Anita B. Hjelmeland, Mark W. Dewhirst, Darell D. Bigner, and Jeremy N.
Rich. 2006. {``Glioma Stem Cells Promote Radioresistance by Preferential
Activation of the DNA Damage Response.''} \emph{Nature} 444 (7120):
756--60. \url{https://doi.org/10.1038/nature05236}.

\bibitem[\citeproctext]{ref-baskar2012}
Baskar, Rajamanickam, Kuo Ann Lee, Richard Yeo, and Kheng-Wei Yeoh.
2012. {``Cancer and Radiation Therapy: Current Advances and Future
Directions.''} \emph{International Journal of Medical Sciences} 9 (3):
193--99. \url{https://doi.org/10.7150/ijms.3635}.

\bibitem[\citeproctext]{ref-bonnet1997}
Bonnet, Dominique, and John E. Dick. 1997. {``Human Acute Myeloid
Leukemia Is Organized as a Hierarchy That Originates from a Primitive
Hematopoietic Cell.''} \emph{Nature Medicine} 3 (7): 730--37.
\url{https://doi.org/10.1038/nm0797-730}.

\bibitem[\citeproctext]{ref-caruxe9n2016}
Carén, Helena, Stephan Beck, and Steven M. Pollard. 2016.
{``Differentiation Therapy for Glioblastoma {\textendash} Too Many
Obstacles?''} \emph{Molecular \& Cellular Oncology} 3 (2): e1124174.
\url{https://doi.org/10.1080/23723556.2015.1124174}.

\bibitem[\citeproctext]{ref-dethuxe92018}
De Thé, Hugues. 2018. {``Differentiation Therapy Revisited.''}
\emph{Nature Reviews Cancer} 18 (2): 117--27.
\url{https://doi.org/10.1038/nrc.2017.103}.

\bibitem[\citeproctext]{ref-enderling2009}
Enderling, Heiko, Alexander R. A. Anderson, Mark A. J. Chaplain, Afshin
Beheshti, Lynn Hlatky, and Philip Hahnfeldt. 2009. {``Paradoxical
Dependencies of Tumor Dormancy and Progression on Basic Cell
Kinetics.''} \emph{Cancer Research} 69 (22): 8814--21.
\url{https://doi.org/10.1158/0008-5472.CAN-09-2115}.

\bibitem[\citeproctext]{ref-enriquez-navas2015}
Enriquez-Navas, Pedro M., Jonathan W. Wojtkowiak, and Robert A. Gatenby.
2015. {``Application of Evolutionary Principles to Cancer Therapy.''}
\emph{Cancer Research} 75 (22): 4675--80.
\url{https://doi.org/10.1158/0008-5472.CAN-15-1337}.

\bibitem[\citeproctext]{ref-folkman2004}
Folkman, Judah, and Raghu Kalluri. 2004. {``Cancer Without Disease.''}
\emph{Nature} 427 (6977): 787--87.
\url{https://doi.org/10.1038/427787a}.

\bibitem[\citeproctext]{ref-galli2004}
Galli, Rossella, Elena Binda, Ugo Orfanelli, Barbara Cipelletti, Angela
Gritti, Simona De Vitis, Roberta Fiocco, Chiara Foroni, Francesco
Dimeco, and Angelo Vescovi. 2004. {``Isolation and Characterization of
Tumorigenic, Stem-Like Neural Precursors from Human Glioblastoma.''}
\emph{Cancer Research} 64 (19): 7011--21.
\url{https://doi.org/10.1158/0008-5472.CAN-04-1364}.

\bibitem[\citeproctext]{ref-gao2013}
Gao, Xuefeng, J. Tyson McDonald, Lynn Hlatky, and Heiko Enderling. 2013.
{``Acute and Fractionated Irradiation Differentially Modulate Glioma
Stem Cell Division Kinetics.''} \emph{Cancer Research} 73 (5): 1481--90.
\url{https://doi.org/10.1158/0008-5472.CAN-12-3429}.

\bibitem[\citeproctext]{ref-hanahan2000}
Hanahan, Douglas, and Robert A Weinberg. 2000. {``The Hallmarks of
Cancer.''} \emph{Cell} 100 (1): 57--70.
\url{https://doi.org/10.1016/S0092-8674(00)81683-9}.

\bibitem[\citeproctext]{ref-hemmati2003}
Hemmati, Houman D., Ichiro Nakano, Jorge A. Lazareff, Michael
Masterman-Smith, Daniel H. Geschwind, Marianne Bronner-Fraser, and
Harley I. Kornblum. 2003. {``Cancerous Stem Cells Can Arise from
Pediatric Brain Tumors.''} \emph{Proceedings of the National Academy of
Sciences} 100 (25): 15178--83.
\url{https://doi.org/10.1073/pnas.2036535100}.

\bibitem[\citeproctext]{ref-hillen2013}
Hillen, Thomas, Heiko Enderling, and Philip Hahnfeldt. 2013. {``The
Tumor Growth Paradox and Immune System-Mediated Selection for Cancer
Stem Cells.''} \emph{Bulletin of Mathematical Biology} 75 (1): 161--84.
\url{https://doi.org/10.1007/s11538-012-9798-x}.

\bibitem[\citeproctext]{ref-hitomi2021}
Hitomi, Masahiro, Anastasia P. Chumakova, Daniel J. Silver, Arnon M.
Knudsen, W. Dean Pontius, Stephanie Murphy, Neha Anand, Bjarne W.
Kristensen, and Justin D. Lathia. 2021. {``Asymmetric Cell Division
Promotes Therapeutic Resistance in Glioblastoma Stem Cells.''} \emph{JCI
Insight} 6 (3): e130510.
\url{https://doi.org/10.1172/jci.insight.130510}.

\bibitem[\citeproctext]{ref-ignatova2002}
Ignatova, Tatyana N., Valery G. Kukekov, Eric D. Laywell, Oleg N.
Suslov, Frank D. Vrionis, and Dennis A. Steindler. 2002. {``Human
Cortical Glial Tumors Contain Neural Stem{-}Like Cells Expressing
Astroglial and Neuronal Markers in Vitro.''} \emph{Glia} 39 (3):
193--206. \url{https://doi.org/10.1002/glia.10094}.

\bibitem[\citeproctext]{ref-lander2009}
Lander, Arthur D, Kimberly K Gokoffski, Frederic Y. M Wan, Qing Nie, and
Anne L Calof. 2009. {``Cell Lineages and the Logic of Proliferative
Control.''} Edited by Charles F Stevens. \emph{PLoS Biology} 7 (1):
e1000015. \url{https://doi.org/10.1371/journal.pbio.1000015}.

\bibitem[\citeproctext]{ref-lapidot1994}
Lapidot, Tsvee, Christian Sirard, Josef Vormoor, Barbara Murdoch, Trang
Hoang, Julio Caceres-Cortes, Mark Minden, Bruce Paterson, Michael A.
Caligiuri, and John E. Dick. 1994. {``A Cell Initiating Human Acute
Myeloid Leukaemia After Transplantation into SCID Mice.''} \emph{Nature}
367 (6464): 645--48. \url{https://doi.org/10.1038/367645a0}.

\bibitem[\citeproctext]{ref-lathia2011}
Lathia, J D, M Hitomi, J Gallagher, S P Gadani, J Adkins, A Vasanji, L
Liu, et al. 2011. {``Distribution of CD133 Reveals Glioma Stem Cells
Self-Renew Through Symmetric and Asymmetric Cell Divisions.''}
\emph{Cell Death \& Disease} 2 (9): e200--200.
\url{https://doi.org/10.1038/cddis.2011.80}.

\bibitem[\citeproctext]{ref-lowengrub2010}
Lowengrub, J S, H B Frieboes, F Jin, Y-L Chuang, X Li, P Macklin, S M
Wise, and V Cristini. 2010. {``Nonlinear Modelling of Cancer: Bridging
the Gap Between Cells and Tumours.''} \emph{Nonlinearity} 23 (1):
R1--91. \url{https://doi.org/10.1088/0951-7715/23/1/R01}.

\bibitem[\citeproctext]{ref-ma2018}
Ma, Qianquan, Wenyong Long, Changsheng Xing, Junjun Chu, Mei Luo, Helen
Y. Wang, Qing Liu, and Rong-Fu Wang. 2018. {``Cancer Stem Cells and
Immunosuppressive Microenvironment in Glioma.''} \emph{Frontiers in
Immunology} 9 (December): 2924.
\url{https://doi.org/10.3389/fimmu.2018.02924}.

\bibitem[\citeproctext]{ref-majumdar2020}
Majumdar, Sreemita, and Song-Tao Liu. 2020. {``Cell Division Symmetry
Control and Cancer Stem Cells.''} \emph{AIMS Molecular Science} 7 (2):
82--101. \url{https://doi.org/10.3934/molsci.2020006}.

\bibitem[\citeproctext]{ref-meza2008}
Meza, Rafael, Jihyoun Jeon, Suresh H. Moolgavkar, and E. Georg Luebeck.
2008. {``Age-Specific Incidence of Cancer: Phases, Transitions, and
Biological Implications.''} \emph{Proceedings of the National Academy of
Sciences} 105 (42): 16284--89.
\url{https://doi.org/10.1073/pnas.0801151105}.

\bibitem[\citeproctext]{ref-nayak2020}
Nayak, Sonali, Ashorne Mahenthiran, Yongyong Yang, Mark McClendon,
Barbara Mania-Farnell, Charles David James, John A. Kessler, et al.
2020. {``Bone Morphogenetic Protein 4 Targeting Glioma Stem-Like Cells
for Malignant Glioma Treatment: Latest Advances and Implications for
Clinical Application.''} \emph{Cancers} 12 (2): 516.
\url{https://doi.org/10.3390/cancers12020516}.

\bibitem[\citeproctext]{ref-neves-e-castro2006}
Neves-E-Castro, Manuel. 2006. {``Why Do Some Breast Cancer Cells Remain
Dormant?''} \emph{Gynecological Endocrinology} 22 (4): 190--97.
\url{https://doi.org/10.1080/09513590600624374}.

\bibitem[\citeproctext]{ref-pannuti2010}
Pannuti, Antonio, Kimberly Foreman, Paola Rizzo, Clodia Osipo, Todd
Golde, Barbara Osborne, and Lucio Miele. 2010. {``Targeting Notch to
Target Cancer Stem Cells.''} \emph{Clinical Cancer Research} 16 (12):
3141--52. \url{https://doi.org/10.1158/1078-0432.CCR-09-2823}.

\bibitem[\citeproctext]{ref-piccirillo2006}
Piccirillo, S. G. M., B. A. Reynolds, N. Zanetti, G. Lamorte, E. Binda,
G. Broggi, H. Brem, A. Olivi, F. Dimeco, and A. L. Vescovi. 2006.
{``Bone Morphogenetic Proteins Inhibit the Tumorigenic Potential of
Human Brain Tumour-Initiating Cells.''} \emph{Nature} 444 (7120):
761--65. \url{https://doi.org/10.1038/nature05349}.

\bibitem[\citeproctext]{ref-reya2001}
Reya, Tannishtha, Sean J. Morrison, Michael F. Clarke, and Irving L.
Weissman. 2001. {``Stem Cells, Cancer, and Cancer Stem Cells.''}
\emph{Nature} 414 (6859): 105--11.
\url{https://doi.org/10.1038/35102167}.

\bibitem[\citeproctext]{ref-ricci-vitiani2007}
Ricci-Vitiani, Lucia, Dario G. Lombardi, Emanuela Pilozzi, Mauro
Biffoni, Matilde Todaro, Cesare Peschle, and Ruggero De Maria. 2007.
{``Identification and Expansion of Human Colon-Cancer-Initiating
Cells.''} \emph{Nature} 445 (7123): 111--15.
\url{https://doi.org/10.1038/nature05384}.

\bibitem[\citeproctext]{ref-rich2007}
Rich, Jeremy N. 2007. {``Cancer Stem Cells in Radiation Resistance.''}
\emph{Cancer Research} 67 (19): 8980--84.
\url{https://doi.org/10.1158/0008-5472.CAN-07-0895}.

\bibitem[\citeproctext]{ref-schonberg2014}
Schonberg, David L., Daniel Lubelski, Tyler E. Miller, and Jeremy N.
Rich. 2014. {``Brain Tumor Stem Cells: Molecular Characteristics and
Their Impact on Therapy.''} \emph{Molecular Aspects of Medicine} 39
(October): 82--101. \url{https://doi.org/10.1016/j.mam.2013.06.004}.

\bibitem[\citeproctext]{ref-singh2003}
Singh, Sheila K., Ian D. Clarke, Mizuhiko Terasaki, Victoria E. Bonn,
Cynthia Hawkins, Jeremy Squire, and Peter B. Dirks. 2003.
{``Identification of a Cancer Stem Cell in Human Brain Tumors.''}
\emph{Cancer Research} 63 (18): 5821--28.

\bibitem[\citeproctext]{ref-singh2004}
Singh, Sheila K., Cynthia Hawkins, Ian D. Clarke, Jeremy A. Squire, Jane
Bayani, Takuichiro Hide, R. Mark Henkelman, Michael D. Cusimano, and
Peter B. Dirks. 2004. {``Identification of Human Brain Tumour Initiating
Cells.''} \emph{Nature} 432 (7015): 396--401.
\url{https://doi.org/10.1038/nature03128}.

\bibitem[\citeproctext]{ref-sweeney1995}
Sweeney, Eamon. 1995. {``Dormant Cells in Columnar Cell Carcinoma of the
Thyroid.''} \emph{Human Pathology} 26 (6): 691--92.
\url{https://doi.org/10.1016/0046-8177(95)90180-9}.

\bibitem[\citeproctext]{ref-taipale2001}
Taipale, Jussi, and Philip A. Beachy. 2001. {``The Hedgehog and Wnt
Signalling Pathways in Cancer.''} \emph{Nature} 411 (6835): 349--54.
\url{https://doi.org/10.1038/35077219}.

\bibitem[\citeproctext]{ref-tang2021}
Tang, Xuejia, Chenghai Zuo, Pengchao Fang, Guojing Liu, Yongyi Qiu, Yi
Huang, and Rongrui Tang. 2021. {``Targeting Glioblastoma Stem Cells: A
Review on Biomarkers, Signal Pathways and Targeted Therapy.''}
\emph{Frontiers in Oncology} 11 (July): 701291.
\url{https://doi.org/10.3389/fonc.2021.701291}.

\bibitem[\citeproctext]{ref-turner2009}
Turner, C., A. R. Stinchcombe, M. Kohandel, S. Singh, and S.
Sivaloganathan. 2009. {``Characterization of Brain Cancer Stem Cells: A
Mathematical Approach.''} \emph{Cell Proliferation} 42 (4): 529--40.
\url{https://doi.org/10.1111/j.1365-2184.2009.00619.x}.

\bibitem[\citeproctext]{ref-weiss2017a}
Weiss, Lora D., Natalia L. Komarova, and Ignacio A. Rodriguez-Brenes.
2017. {``Mathematical Modeling of Normal and Cancer Stem Cells.''}
\emph{Current Stem Cell Reports} 3 (3): 232--39.
\url{https://doi.org/10.1007/s40778-017-0094-4}.

\bibitem[\citeproctext]{ref-yan2016}
Yan, Min, and Quentin Liu. 2016. {``Differentiation Therapy: A Promising
Strategy for Cancer Treatment.''} \emph{Chinese Journal of Cancer} 35
(1): 3. \url{https://doi.org/10.1186/s40880-015-0059-x}.

\bibitem[\citeproctext]{ref-youssefpour2012}
Youssefpour, H., X. Li, A. D. Lander, and J. S. Lowengrub. 2012.
{``Multispecies Model of Cell Lineages and Feedback Control in Solid
Tumors.''} \emph{Journal of Theoretical Biology} 304 (July): 39--59.
\url{https://doi.org/10.1016/j.jtbi.2012.02.030}.

\end{CSLReferences}




\end{document}
